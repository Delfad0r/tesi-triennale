\chapter{Successioni spettrali}

\section{Prime definizioni}

\begin{definition}
Un gruppo abeliano $A$ si dice graduato se si scrive come somma diretta di sottogruppi
$$
A=\Dirsum_{n\in\NN}\pre{n}{A}.
$$
Gli elementi non nulli di $\pre{n}{A}$ si dicono omogenei di grado $n$. Se $x\in A\setminus0$ è omogeneo di grado $n$ scriviamo $\deg{x}=n$.
\end{definition}
\begin{definition}
Sia $A$ un gruppo graduato. Un'applicazione $\map{d}{A}{A}$ si dice differenziale di grado $k$ se $dd=0$ e per ogni $n\in\NN$ vale $d(\pre{n}{A})\subs\pre{n+k}{A}$.
\end{definition}
A ogni gruppo graduato $A$ dotato di un differenziale $d$ di grado $-1$ è associato in modo naturale un complesso di gruppi abeliani $A_\bullet$ dove l'$n$-esimo gruppo è $\pre{n}{A}$ e le mappe di bordo sono $d_n=d|_{\pre{n}{A}}$. Possiamo allora definire i gruppi di omologia di $A$ riconducendoci al complesso associato: $H_n(A)=H_n(A_\bullet)$. In altri termini:
$$
H_n(A)=\frac{\ker d\cap\pre{n}{A}}{\im d\cap\pre{n}{A}}.
$$
\begin{definition}
Sia $A$ un gruppo graduato dotato di un differenziale $d$ di grado $-1$. Si dice filtrazione crescente una successione di sottogruppi $A^p$ indicizzata da $p\in\NN$ che soddisfi le seguenti proprietà.
\begin{enumerate}
\item $\bigcup_{p\in\NN}A^p=A$.
\item Per ogni $p$ vale $A^p\subs A^{p+1}$.
\item Ogni $A^p$ è somma diretta delle sue componenti omogenee $\pre{n}{A}\cap A^p$.
\item\label{filtering-degree-total-degree} Per ogni $n$ vale $\pre{n}{A}\subs A^n$.
\item Ogni $A^p$ è stabile per $d$, ossia $d(A^p)\subs A^p$.
\end{enumerate}
\end{definition}
In questa situazione, poniamo per comodità $A^p=0$ per $p < 0$ e $\pre{n}{A}=0$ per $n<0$. Definiamo inoltre, per ogni $r\in\ZZ$, i sottogruppi di $A^p$:
\begin{itemize}
\item $C^p_r=d^{-1}(A^{p-r})\cap A^p$;
\item $B^p_r=d(A^{p+r})\cap A^p$;
\item $C^p_\infty=d^{-1}(0)\cap A^p$;
\item $B^p_\infty=d(A)\cap A^p$.
\end{itemize}
Valgono le seguenti relazioni di inclusione:
$$
\ldots\subs B^p_{r-1}\subs B^p_r\subs\ldots\subs B^p_\infty\subs C^p_\infty\subs\ldots\subs C^p_r\subs C^p_{r-1}\subs\ldots\subs C^p_0=A^p.
$$
È facile convincersi che questi sottogruppi sono ancora gruppi graduati, ossia si scrivono come somma diretta delle loro componenti omogenee (secondo il grado di A). Per motivi che risulteranno evidenti in seguito, conviene definire, per $p,q\in\ZZ$, $A^{p,q}=\pre{p+q}{A}\cap A^p$ (gli elementi omogenei di $A^p$ di grado $p+q$). Denotiamo con $C^{p,q}_r=C^p_r\cap A^{p,q}$ gli elementi omogenei di grado $p+q$ di $C^p_r$; definiamo analogamente $B^{p,q}_r,C^{p,q}_\infty,B^{p,q}_\infty$.

Possiamo ora costruire la successione spettrale associata ad $A$. Dati $p,q\in\ZZ,r\in\NN$ definiamo
$$
E^{p,q}_r=\frac{C^{p,q}_r}{B^{p,q}_{r-1}+C^{p-1,q+1}_{r-1}}.
$$
In $E^{p,q}_r$, $p$ è detto grado filtrante, $q$ grado complementare, $p+q$ grado totale (quest'ultimo corrisponde al grado in $A$).
Osserviamo che 
\begin{align*}
d(C^{p,q}_r)&=d(d^{-1}(A^{p-r})\cap A^{p,q})\\
&\subs A^{p-r}\cap d(A^{p,q})\\
&=A^{p-r,q+r-1}\cap d(A^p)\\
&=B^{p-r,q+r-1}_r\\
&\subs C^{p-r,q+r-1}_r
\end{align*}
e
\begin{align*}
d(B^{p,q}_{r-1}+C^{p-1,q+1}_{r-1})&=d(C^{p-1,q+1}_{r-1})\\
&\subs B^{p-r,q+r-1}_{r-1}\\
&\subs B^{p-r,q+r-1}_{r-1}+C^{p-r-1,q+r}_{r-1},
\end{align*}
dunque il differenziale $d$ passa al quoziente $\map{d^{p,q}_r}{E^{p,q}_r}{E^{p-r,q+r-1}_r}$. Come si vede immediatamente, il differenziale:
\begin{itemize}
\item diminuisce di $r$ il grado filtrante,
\item aumenta di $r-1$ il grado complementare,
\item diminuisce di $1$ il grado totale.
\end{itemize}
Per concludere, definiamo i gruppi terminali
$$
E^{p,q}_\infty=\frac{C^{p,q}_\infty}{B^{p,q}_\infty+C^{p-1,q+1}_\infty}.
$$
\begin{proposition}\thlabel{spectral-sequence-first-properties}
La successione spettrale $E$ soddisfa le seguenti proprietà.
\begin{enumerate}
\item $E^{p,q}_{r+1}=\ker d^{p,q}_r/\im d^{p+r,q-r+1}_r$.
\item $E^{p,q}_r=0$ per $p<0$ o $q<0$.
\item Se $r>p$, allora $E^{p,q}_{r+1}$ è un quoziente di $E^{p,q}_r$.
\item Se $r>q+1$, allora $E^{p,q}_{r+1}$ è un sottogruppo di $E^{p,q}_r$.
\item $E^{p,q}_r=E^{p,q}_\infty$ per $r$ sufficientemente grande (in particolare, per $r>\max\{p,q+1\}$).
\end{enumerate}
\end{proposition}
\begin{proof}
\leavevmode
\begin{enumerate}
\item \todo{Fare i conti}
\item Osserviamo che $E^{p,q}_0=A^{p,q}/A^{p-1,q+1}$. Se $p<0$ chiaramente $E^{p,q}_0=0$ (ricordiamo che abbiamo posto $A^p=0$ per $p<0$). Se $q<0$, allora $A^{p,q}=\pre{p+q}{A}\cap A^p\subs\pre{p+1}{A}\cap A^{p-1}=A^{p-1,q+1}$, dunque $E^{p,q}_0=0$ anche in questo caso. La proprietà è pertanto vera per $r=0$; ma $E^{p,q}_{r+1}$ è un quoziente di un sottogruppo di $E^{p,q}_r$, da cui per induzione $E^{p,q}_r=0$ per ogni $r\ge 0$.
\item Se $r>p$, allora $E^{p-r,q+r-1}_r=0$, da cui $d^{p,q}_r=0$. Segue che $E^{p,q}_{r+1}=E^{p,q}_r/\im d^{p+r,q-r+1}_r$ è un quoziente di $E^{p,q}_r$.
\item Se $r>q+1$, allora $E^{p+r,q+r-1}_r=0$, da cui $d^{p+r,q+r-1}=0$. Segue che $E^{p,q}_{r+1}=\ker d^{p,q}_r/0$ è un sottogruppo di $E^{p,q}_r$.
\item Se $r>p$, allora $C^p_r=C^p_\infty$ e $C^{p-1}_{r-1}=C^{p-1}_\infty$. Se $r>q+1$, allora
\begin{align*}
B^{p,q}_\infty&=d(A)\cap A^{p,q}=d(\pre{p+q+1}{A})\cap A^{p,q}\\
&\subs d(A^{p+q+1})\cap A^{p,q}\subs d(A^{p+r})\cap A^{p,q}=B^{p,q}_r.
\end{align*}
Pertanto, se $r>\max\{p,q+1\}$, allora $E^{p,q}_r=E^{p,q}_\infty$.
\end{enumerate}
\end{proof}