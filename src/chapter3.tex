\chapter{Successione spettrale omologica di uno spazio fibrato}

\section{Costruzione della successione spettrale}
Consideriamo una fibrazione $\fibration{F}{E}{B}$. Il nostro obiettivo è mettere in relazione i gruppi di omologia di $B$, $E$ e $F$; lo strumento che adopereremo sono le successioni spettrali. In questa sezione ci capiterà di richiedere l'esistenza di costruzioni (analoghe a quella del \thref{fibration-H-construction}) che associno cubi ad altri cubi, soddisfacendo determinate proprietà. Per separare nettamente la parte algebrica da quella topologica e per evitare di allungare eccessivamente le dimostrazioni, già piuttosto estese, abbiamo preferito rimandare le suddette dimostrazioni alla fine del capitolo (sezione \ref{spectral-sequence-of-fibration:proofs}).

\subsection{Filtrazione del complesso singolare}
Per applicare i risultati della sezione \ref{sec:spectral-sequence-filtered-complex}, è necessario definire una filtrazione crescente sul complesso singolare $C_\bullet(E)$ (d'ora in poi ometteremo la $E$ fra parentesi). Ciò che faremo sarà filtrare il complesso $Q_\bullet$ con dei sottocomplessi $Q^p_\bullet$ e prenderne le immagini nel quoziente $C_\bullet$.

Sia dunque $Q^p_n$ il sottogruppo di $Q_n$ generato dai cubi $u\in Q_n$ tali che $p\circ u$ dipende solo dalle prime $p$ coordinate (e $Q^p_n=Q_n$ se $p>n$). Si vede immediatamente che i $Q^p_\bullet$ sono sottocomplessi di $Q_\bullet$ e che soddisfano le proprietà di una filtrazione crescente. Dunque lo stesso vale per $C^p_\bullet=(Q^p_\bullet+D_\bullet)/D_\bullet$, che definiscono una filtrazione crescente per $C_\bullet$. Applicando la \thref{spectral-sequence-of-filtered-complex} otteniamo una successione spettrale $E^{p,q}_r\converges H_{p+q}(E)$. Come vedremo, è possibile calcolare esplicitamente i termini $E^{p,q}_2$ della successione spettrale in funzione dei gruppi di omologia di $B$ e di $F$.

\section{Il termine \texorpdfstring{$E_0$}{E0}}
Costruiamo due applicazioni $B^p$ e $F^p$ definite sui cubi di $Q^p_\bullet$ (ed estese per $\ZZ$-linearità a tutto $Q^p_\bullet$). Se $u\in\ Q^p_n$ è un cubo di dimensione $n$ con $n\ge p$, posto $q=n-p$, definiamo
\begin{align*}
(B^pu)(t_1,\ldots,t_p)&=pu(t_1,\ldots,t_p,0,\ldots,0);\\
(F^pu)(t_1,\ldots,t_q)&=u(0,\ldots,0,t_1,\ldots,t_q).
\end{align*}
$B^pu$ è un cubo di $B$ di dimensione $p$; notiamo che, poiché $u\in Q^p_n$, possiamo sostituire gli zeri nella definizione con qualunque altra $q$-upla di reali fra $0$ e $1$. $F^pu$ è invece un cubo di $F$ di dimensione $q$; la sua immagine è contenuta in $F$ poiché
$$
p(F^pu)(t_1,\ldots,t_q)=pu(0,\ldots,0,t_1,\ldots,t_q)=pu(0,\ldots,0)=b.
$$
Le seguenti proprietà sono di verifica immediata.
\begin{enumerate}
\item\label{spectral-sequence-of-fibration:pr1} $B^pu$ è degenere se e solo se $u\in Q^{p-1}_n$.
\item\label{spectral-sequence-of-fibration:pr2} Se $u$ è degenere e $q>0$ allora $F^pu$ è degenere; se $u$ è degenere e $q=0$ allora $B^pu$ è degenere.
\item\label{spectral-sequence-of-fibration:pr3} Se $i>p,\epsilon\in\{0,1\}$ allora $B^p\lambda^\epsilon_iu=B^pu$ e $F^p\lambda^\epsilon_iu=\lambda^\epsilon_{i-p}F^pu$.
\end{enumerate}

Ricordiamo che $E^{p,q}_0=C^p_{p+q}/C^{p-1}_{p+q}$ e che la mappa $\map{d^{p,q}_0}{E^{p,q}_0}{E^{p,q-1}_0}$ si ottiene dalla mappa di bordo di $C_\bullet$ per passaggio al quoziente. Possiamo dunque raggruppare i termini della successione spettrale aventi $r=0$ in complessi $E^{p,\bullet}_0$, con mappe di bordo $d^p_0$. Osserviamo che
$$
d^p_0u=\sum_{i>p}\sum_\epsilon(-1)^{i+\epsilon}\lambda^\epsilon_iu,
$$
in quanto per $i\le p$ vale $\lambda^\epsilon_iu\in Q^{p-1}_\bullet$.

Consideriamo il complesso $J^p_\bullet=C_p(B)\tensor C_\bullet(F)$ con mappe di bordo $d_J(b\tensor f)=(-1)^pb\tensor df$, e sia $\varphi^p$ il morfismo di complessi
\Map{\varphi^p}{E^{p,\bullet}_0}{J^p_\bullet}{u}{B^pu\tensor F^pu}
Le proprietà \ref{spectral-sequence-of-fibration:pr1} e \ref{spectral-sequence-of-fibration:pr2} garantiscono che si tratta di una buona definizione (ossia che è compatibile con il quoziente che definisce $E^{p,\bullet}_0$), mentre dalla \ref{spectral-sequence-of-fibration:pr3} segue che $\varphi^p$ è effettivamente un morfismo di complessi:
\begin{align*}
\varphi^pd^p_0u&=\varphi^p\sum_{i>p}\sum_\epsilon(-1)^{i+\epsilon}\lambda^\epsilon_iu\\
&=\sum_{i>p}\sum_\epsilon(-1)^{i+\epsilon}B^p\lambda^\epsilon_iu\tensor F^p\lambda^0_iu\\
&=\sum_{i>p}\sum_\epsilon(-1)^pB^pu\tensor(-1)^{p-i+\epsilon}\lambda^\epsilon_{i-p}F^pu\\
&=(-1)^pB^pu\tensor dF^pu\\
&=d_J\varphi^pu.
\end{align*}

Ci proponiamo di mostrare che $\varphi^p$ è in realtà un'equivalenza omotopica. Sfrutteremo il seguente lemma per costruire un'inversa omotopica.

\begin{restatable}{lemma}{spectralsequenceoffibrationKconstruction}\thlabel{spectral-sequence-of-fibration-K-construction}
Esiste un'applicazione $K$ che associa a ogni coppia di cubi $(u,v)$ con $u\in Q_p(B)$ e $v\in Q_q(F)$ un cubo $K(u,v)\in Q^p_{p+q}$ e che soddisfa le seguenti proprietà:
\begin{enumerate}
\item\label{spectral-sequence-of-fibration-K-construction:pr1} $B^pK(u,v)=u$ e $F^pK(u,v)=v$;
\item\label{spectral-sequence-of-fibration-K-construction:pr2} per ogni $0<i\le q,\epsilon\in\{0,1\}$ vale $K(u,\lambda^\epsilon_iv)=\lambda^\epsilon_{i+p}K(u,v)$;
\item\label{spectral-sequence-of-fibration-K-construction:pr3} se $v$ è degenere allora $K(u,v)$ è degenere.
\end{enumerate}
\end{restatable}
\begin{proof}
La dimostrazione è rimandata alla fine del capitolo (sezione \ref{spectral-sequence-of-fibration:proofs}).
\end{proof}

Definiamo allora il morfismo di complessi
\Map{\psi^p}{J^p_\bullet}{E^{p,\bullet}_0}{u\tensor v}{K(u,v)}
Osserviamo che $\psi^p$ è ben definito (ossia è compatibile con i quozienti che definiscono $J^p_\bullet$): infatti se $u$ è degenere allora $B^pK(u,v)$ è degenere per la \ref{spectral-sequence-of-fibration-K-construction:pr1}, dunque $K(u,v)\in Q^{p-1}_\bullet$. Se invece $v$ è degenere allora $K(u,v)$ è degenere per la \ref{spectral-sequence-of-fibration-K-construction:pr3}. Inoltre $\psi^p$ è effettivamente un morfismo di complessi, come mostra un semplice verifica:
\begin{align*}
\psi^pd_J(u\tensor v)&=(-1)^p\psi^p(u\tensor dv)\\
&=(-1)^p\psi^p\left(u\tensor\sum_{i=1}^q\sum_\epsilon(-1)^{i+\epsilon}\lambda^\epsilon_iv\right)\\
&=\sum_{i=1}^q\sum_\epsilon(-1)^{p+i+\epsilon}K(u,\lambda^\epsilon_iv)\\
&=\sum_{i=p+1}^n\sum_\epsilon(-1)^{i+\epsilon}\lambda^\epsilon_iK(u,v)\\
&=d^p_0\psi^p(u\tensor v).
\end{align*}

\section{Il termine \texorpdfstring{$E_1$}{E1}}

\begin{proposition}\thlabel{spectral-sequence-of-fibration-homotopy-equivalence}
Per ogni $p$ la mappa $\map{\varphi^p}{E^{p,\bullet}_0}{J^p_\bullet}$ è un'equivalenza omotopica, di cui $\map{\psi^p}{J^p_\bullet}{E^{p,\bullet}_0}$ è un'inversa omotopica.
\end{proposition}
\begin{proof}
Si vede immediatamente che $\varphi^p\psi^p=\1$. Per dimostrare che $\psi^p\varphi^p$ è omotopo all'identità ci serviremo del seguente lemma.
\begin{restatable}{lemma}{spectralsequenceoffibrationSconstruction}\thlabel{spectral-sequence-of-fibration-S-construction}
Esiste un'applicazione $S$ che associa a ogni cubo $u\in Q^p_n$ un cubo $Su\in Q^p_{n+1}$ e che soddisfa le seguenti proprietà:
\begin{enumerate}
\item\label{spectral-sequence-of-fibration-S-construction:pr1} $B^pSu=B^pu$;
\item\label{spectral-sequence-of-fibration-S-construction:pr2} $\lambda^0_{p+1}Su=u$ e $\lambda^1_{p+1}Su=K(B^pu,F^pu)$;
\item\label{spectral-sequence-of-fibration-S-construction:pr3} se $i>p$ e $\epsilon\in\{0,1\}$ vale $S\lambda^\epsilon_iu=\lambda^\epsilon_{i+1}Su$;
\item\label{spectral-sequence-of-fibration-S-construction:pr4} se $q>0$ e $u$ è degenere, allora $Su$ è degenere.
\end{enumerate}
\end{restatable}
\begin{proof}
La dimostrazione è rimandata alla fine del capitolo (sezione \ref{spectral-sequence-of-fibration:proofs}).
\end{proof}
Dato un cubo $u\in E^{p,q}_0$, definiamo $hu=(-1)^pSu$, ed estendiamo $h$ a una mappa
$\map{h}{E^{p,q}_0}{E^{p,q+1}_0}$ per $\ZZ$-linearità. Osserviamo che $h$ è ben definita (ossia è compatibile con il quoziente che definisce $E^{p,q}_0$): infatti se $u\in Q^{p-1}_n(E)$ allora $Su\in Q^{p-1}_{n+1}(E)$ per la proprietà \ref{spectral-sequence-of-fibration-S-construction:pr1}; se $u$ è degenere e $q>0$ allora $Su$ è degenere (proprietà \ref{spectral-sequence-of-fibration-S-construction:pr4}), mentre se $u$ è degenere e $q=0$ allora $B^pSu$ è degenere (proprietà \ref{spectral-sequence-of-fibration-S-construction:pr1}), dunque $Su\in Q^{p-1}_{n+1}(E)$.

Mostriamo ora che $h$ definisce un'omotopia fra $\psi^p\varphi^p$ e l'identità. Calcoliamo
\begin{align*}
d^p_0hu+hd^p_0u&=(-1)^p(d^p_0Su+Sd^p_0u)\\
&=(-1)^p\left(\sum_{i=p+1}^{n+1}\sum_\epsilon(-1)^{i+\epsilon}\lambda^\epsilon_iSu+\sum_{i=p+1}^n\sum_\epsilon(-1)^{i+\epsilon}S\lambda^\epsilon_iu\right)\\
&=(-1)^p(-1)^{p+1}(u-K(B^pu,F^pu))+\\
&\phantomop(-1)^p\sum_{i=p+2}^{n+1}\sum_\epsilon(-1)^{i+\epsilon}S\lambda^\epsilon_{i-1}u+\\
&\phantomop(-1)^p\sum_{i=p+1}^n\sum_\epsilon(-1)^{i+\epsilon}S\lambda^\epsilon_iu\\
&=K(B^pu,F^pu)-u\\
&=\psi^p\varphi^pu-u.
\end{align*}
Questo conclude la dimostrazione.
\end{proof}

Da questa proposizione segue che le mappe indotte in omologia $\map{\varphi_*}{H_q(E^{p,\bullet}_0)}{H_q(J^p_\bullet)}$ sono isomorfismi. Osserviamo però che $H_q(E^{p,\bullet}_0)=E^{p,q}_1$, mentre dal fatto che $C_p(B)$ è un gruppo abeliano libero segue
$$
H_q(J^p_\bullet)=H_q(C_p(B)\tensor C_\bullet(F))=C_p(B)\tensor H_q(F)=C_p(B;H_q(F)).
$$
Abbiamo dunque dimostrato la seguente.
\begin{proposition}\thlabel{spectral-sequence-of-fibration-E-1}
I morfismi di complessi $\varphi^p$ inducono isomorfismi $\map{\varphi_*}{E^{p,q}_1}{C_p(B;H_q(F))}$.
\end{proposition}

\section{Il termine \texorpdfstring{$E_2$}{E2}}
Per calcolare i termini $E^{p,q}_2$ della successione spettrale, è necessario studiare come si trasformano le mappe di bordo $d^{p,q}_1$ mediante l'isomorfismo $\varphi_*$.
\begin{proposition}\thlabel{spectral-sequence-of-fibration-boundary-map}
Supponiamo che il gruppo fondamentale di $B$ agisca banalmente sui gruppi di omologia di $F$. Allora il seguente diagramma commuta
\begin{diagram}
E^{p,q}_1\rar{d^{p,q}_1}\dar{\varphi_*}&E^{p-1,q}_1\dar{\varphi_*}\\
C_p(B;H_q(F))\rar{d}&C_{p-1}(B;H_q(F))
\end{diagram}
dove abbiamo denotato con $d$ la mappa di bordo canonica del complesso singolare.
\end{proposition}
\begin{proof}
Consideriamo un elemento $x=b\tensor h\in C_p(B;H_q(F))$, dove $b$ è un cubo di $B$ di dimensione $p$; sia $m\in C_q(F)$ un ciclo della classe di omologia di $h$. Ci proponiamo di calcolare $\varphi^{p-1}d\psi^p(b\tensor m)$; per passaggio al quoziente otterremo $\varphi_*d^{p,q}_1\varphi_*^{-1}x$. Scriviamo $m$ come somma di cubi: $m=\sum_{\alpha}g_\alpha u_\alpha$ con $g_\alpha\in\ZZ$. Calcoliamo
$$
d\psi^p(b\tensor m)=\sum_\alpha\sum_{i=1}^n\sum_\epsilon(-1)^{i+\epsilon}g_\alpha\lambda^\epsilon_iK(b,u_\alpha).
$$
Osserviamo però che
\begin{align*}
\sum_\alpha\sum_{i=p+1}^n\sum_\epsilon(-1)^{i+\epsilon}g_\alpha\lambda^\epsilon_iK(b,u_\alpha)&=\sum_{\alpha}\sum_{i=p+1}^n\sum_\epsilon(-1)^{i+\epsilon}g_\alpha K(b,\lambda^\epsilon_{i-p}u_\alpha)\\
&=\sum_\alpha(-1)^pg_\alpha\psi^p(b\tensor du_\alpha)\\
&=(-1)^p\psi^p(b\tensor dm)\\
&=0,
\end{align*}
da cui
$$
d\psi^p(b\tensor m)=\sum_\alpha\sum_{i=1}^p\epsilon(-1)^{i+\epsilon}g_\alpha\lambda^\epsilon_iK(b,u_\alpha).
$$
Poiché $K(b,u_\alpha)\in Q^p_{p+q}$, abbiamo che $\lambda^\epsilon_iK(b,u_\alpha)\in Q^{p-1}_{p+q-1}$, dunque possiamo applicare $\varphi^{p-1}$ ai singoli termini della sommatoria, ottenendo
$$
\varphi^{p-1}d\psi^p(b\tensor m)=\sum_\alpha\sum_{i=1}^p\sum_\epsilon(-1)^{i+\epsilon}g_\alpha B^{p-1}\lambda^\epsilon_iK(b,u_\alpha)\tensor F^{p-1}\lambda^\epsilon_iK(b,u_\alpha).
$$
Si vede facilmente che $B^{p-1}\lambda^{\epsilon}_iK(b,u_\alpha)=\lambda^\epsilon_i B^pK(b,u_\alpha)=\lambda^\epsilon_ib$. Per quanto riguarda $F^{p-1}\lambda^\epsilon_iK(b,u_\alpha)$ abbiamo invece
$$
(F^{p-1}\lambda^\epsilon_iK(b,u_\alpha))(x_1,\ldots,x_q)=K(b,u_\alpha)(0,\ldots,0,\epsilon,0,\ldots,0,x_1,\ldots,x_q)
$$
dove $\epsilon$ si trova in posizione $i$. Consideriamo l'applicazione $T$ che a ogni cubo $u$ di $F$ di dimensione $q$ associa un cubo $Tu$ di $E$ di dimensione $q+1$ definito da
$$
(Tu)(t,x_1,\ldots,x_q)=K(b,u)(0,\ldots,0,t\epsilon,0,\ldots,0,x_1,\ldots,x_q)
$$
dove $t\epsilon$ si trova di nuovo in posizione $i$. Si verifica facilmente che $T$ è una costruzione subordinata al cammino $t\mapsto b(0,\ldots,0,t\epsilon,0,\ldots,0)$. È inoltre evidente che $S_Tu=F^{p-1}\lambda^\epsilon_iK(b,u)$. Se denotiamo con $S_{i,\epsilon,b}$ il morfismo $S_T$ associato all'applicazione $T$ determinata dai valori fissati di $i,\epsilon$ e $b$, possiamo scrivere
\begin{align*}
\varphi^{p-1}d\psi^p(b\tensor m)&=\sum_\alpha\sum_{i=1}^p\sum_\epsilon(-1)^{i+\epsilon}g_\alpha\lambda^\epsilon_ib\tensor S_{i,\epsilon,b}u_\alpha\\
&=\sum_{i=1}^p\sum_\epsilon(-1)^{i+\epsilon}\lambda^\epsilon_ib\tensor S_{i,\epsilon,b}m.
\end{align*}
Passando al quoziente, ricordando che $S_{i,\epsilon, b}$ induce l'identità in omologia, otteniamo infine
$$
\varphi_*d^{p,q}_1\varphi_*^{-1}x\varphi_*d^{p,q}_1=\varphi_*^{-1}(b\tensor h)=db\tensor h.
$$
\end{proof}
Da questa proposizione segue immediatamente il risultato conclusivo della sezione.
\begin{proposition}\thlabel{spectral-sequence-of-fibration-E-2}
Supponiamo che il gruppo fondamentale di $B$ agisca banalmente sui gruppi di omologia di $F$. Allora $\varphi_*$ induce un isomorfismo $E^{p,q}_2\iso H_p(B;H_q(F))$.
\end{proposition}

Osserviamo che i risultati ottenuti fino a questo punto rimangono validi, con le medesime dimostrazioni, se consideriamo l'omologia a coefficienti in un gruppo abeliano $G$ invece che in $\ZZ$: a ogni spazio fibrato si può associare una successione spettrale $E^{p,q}_r$ (derivante dall'omologia a coefficienti in $G$) i cui termini con $r=2$ sono isomorfi a $H_p(B;H_q(F;G))$.

\section{Applicazioni}
Presentiamo ora due risultati le cui dimostrazioni fanno uso degli strumenti sviluppati in questo capitolo.

\begin{proposition}\thlabel{fibration-finitely-generated-homology-groups}
Sia $\fibration{F}{E}{B}$ una fibrazione in cui il gruppo fondamentale di $B$ agisce banalmente sui gruppi di omologia di $F$. Supponiamo che i gruppi $H_i(B)$ e $H_i(E)$ siano finitamente generati. Allora anche i gruppi $H_i(F)$ lo sono.
\end{proposition}
\begin{proof}
La dimostrazione segue esattamente lo schema di quella della \thref{homology-of-universal-covering-finitely-generated}, utilizzando la successione spettrale associata alla fibrazione in luogo di quella del rivestimento universale.
\end{proof}

\begin{proposition}\thlabel{fibration-homology-exact-sequence}
Sia $G$ un PID, $\fibration{F}{E}{B}$ una fibrazione in cui il gruppo fondamentale di $B$ agisce banalmente sui moduli di omologia di $F$. Supponiamo che $H_i(B;G)=0$ per $0<i<p$ e che $H_i(F;G)=0$ per $0<i<q$. Allora esiste una successione esatta
\begin{diagram}
H_{p+q-1}(F;G)\rar&H_{p+q-1}(E;G)\rar&H_{p+q-1}(B;G)\rar{d_{p+q-1}}&H_{p+q-2}(F;G)\rar&\ldots\rar&H_2(B;G)\rar{d_2}&H_1(F;G)\rar&H_1(E;G)\rar&H_1(B;G)\rar&0
\end{diagram}
\end{proposition}
\begin{proof}
Consideriamo la successione spettrale $E^{p,q}_r$ associata alla fibrazione. Per il teorema dei coefficienti universali vale
$$
E^{i,j}_2=(H_i(B;G)\tensor H_j(F;G))\dirsum\Tor(H_{i-1}(B;G),H_j(F;G)),
$$
da cui $E^{i,j}_2=0$ se $i,j>0$ e $i+j\le p+q-1$. Pertanto, se $1\le n\le p+q-1$, ci sono al più due termini $E^{i,j}_2$ non nulli con $i+j=n$ (ossia $(i,j)=(0,n)$ e $(i,j)=(n,0)$). Possiamo dunque applicare la \thref{spectral-sequence-exact-sequence} con $(a_n',b_n')=(0,n)$ e $(a_n'',b_n'')=(n,0)$ (si vede facilmente che le altre condizioni sono verificate). La successione esatta che otteniamo, ricordando che $E^{0,n}_2=H_n(F;G)$ e $E^{n,0}=H_n(B;G)$, è precisamente quella della tesi.
\end{proof}

\section{Dimostrazione dei Lemmi \ref{spectral-sequence-of-fibration-K-construction} e \ref{spectral-sequence-of-fibration-S-construction}}\label{spectral-sequence-of-fibration:proofs}
Poiché le dimostrazioni dei due lemmi non si discostano molto da quella del \thref{fibration-H-construction} e le verifiche non presentano sostanziali difficoltà, ometteremo alcuni dettagli.

\spectralsequenceoffibrationKconstruction*
\begin{proof}
Costruiamo $K$ per induzione su $q$.
\paragraph{Costruzione per $q=0$.} In questo caso $v$ è semplicemente il punto $e$, e il problema si riduce a trovare un cubo $\map{w}{I^p}{E}$ avente tutti i vertici in $e$ tale che $pw=u$. Denotiamo con $s$ il punto $(0,\ldots,0)\in I^p$. Consideriamo il diagramma
\begin{diagram}
\{s\}\rar{g}\dar{i}&E\dar{p}\\
I^p\rar{u}\ar[ru,dashed,"w'"]&B
\end{diagram}
dove $g(s)=e$, e l'esistenza di $w'$ è garantita dalla \thref{homotopy-lifting-plus}. Per ottenere un cubo $w$ con i vertici in $e$ procediamo come segue. Siano $\{s_\alpha\}_\alpha$ i vertici di $I^p$; sappiamo che $w'(s_\alpha)\in F$. Poiché $F$ è connesso per archi esistono cammini $\gamma_\alpha$ con $\gamma_\alpha(0)=w'(s_\alpha)$ e $\gamma_\alpha(1)=e$. Consideriamo allora il diagramma
\begin{diagram}
A\dar{i}\rar{g}&E\dar{p}\\
X\rar{f}\ar[ur,dashed,"h"]&B
\end{diagram}
dove
\begin{itemize}
\item $X=I^p\times I$;
\item $A=(I^p\times\{0\})\cup(\{s_\alpha\}\times I)$;
\item $f(x_1,\ldots,x_p,t)=u(x_1,\ldots,x_p)$;
\item $g$ è così definita:
$$
\begin{cases}
g(x_1,\ldots,x_p,0)=w'(x_1,\ldots,x_p)\\
g(s_\alpha,t)=\gamma_\alpha(t)
\end{cases};
$$
\item l'esistenza di $h$ è garantita dalla \thref{homotopy-lifting-plus}.
\end{itemize}
Allora il cubo $w(x_1,\ldots,x_n)=h(1,x_1,\ldots,x_n)$ soddisfa le proprietà richieste.
\paragraph{Costruzione per $q>0$ e $v$ degenere.} Se $v$ è degenere, poniamo
$$
K(u,v)(x_1,\ldots,x_n)=K(u,\lambda^0_q)(x_1,\ldots,x_{n-1}).
$$
Si verifica facilmente che $K(u,v)$ soddisfa le proprietà richieste.
\paragraph{Costruzione per $q>0$ e $v$ non degenere.} Consideriamo il diagramma
\begin{diagram}
A\dar{i}\rar{g}&E\dar{p}\\
X\rar{f}\ar[ur,dashed,"w"]&B
\end{diagram}
dove
\begin{itemize}
\item $X=I^p\times I^q$;
\item $A=(\{s\}\times I^q)\cup(I^p\times\del I^q)$;
\item $f(x_1,\ldots,x_p,y_1,\ldots,y_q)=u(x_1,\ldots,x_p)$;
\item $g$ è così definita
$$
\begin{cases}
g(0,\ldots,0,y_1,\ldots,y_q)=v(y_1,\ldots,y_q)\\
g(x_1,\ldots,x_p,y_1,\ldots,y_{i-1},\epsilon,y_i,\ldots,y_{q-1})=K(u,\lambda^\epsilon_iv)(x_1,\ldots,x_p,y_1,\ldots,y_{q-1})&\epsilon\in\{0,1\}
\end{cases};
$$
\item l'esistenza di $w$ è garantita dalla \thref{homotopy-lifting-plus}, a patto di dimostrare che $g$ è ben definita sulle intersezioni e che $pg=fi$; si tratta in entrambi i casi di facili verifiche, sfruttando l'ipotesi induttiva.
\end{itemize}
Poniamo $K(u,v)=w$; si verifica senza difficoltà alcuna che le proprietà richieste sono soddisfatte.
\end{proof}

\spectralsequenceoffibrationSconstruction*
\begin{proof}
Costruiamo $S$ per induzione su $q$, richiedendo inoltre che soddisfi la seguente proprietà:
\begin{enumerate}
\setcounter{enumi}{4}
\item $(Su)(0,\ldots,0,t,x_1,\ldots,x_n)=u(0,\ldots,0,x_1,\ldots,x_q)$.
\end{enumerate}
\paragraph{Costruzione per $q=0$.} Consideriamo il diagramma
\begin{diagram}
A\dar{i}\rar{g}&E\dar{p}\\
X\rar{f}\ar[ur,dashed,"v"]&B
\end{diagram}
dove
\begin{itemize}
\item $X=I^p\times I$;
\item $A=(I^p\times\{0,1\})\cup(\{s\}\times I)$ (continuiamo a denotare con $s$ il punto $(0,\ldots,0)\in I^p$);
\item $f(x_1,\ldots,x_p,1)=pu(x)$;
\item $g$ è così definita
$$
\begin{cases}
g(x_1,\ldots,x_p,0)=u(x_1,\ldots,x_p)\\
g(x_1,\ldots,x_p,1)=K(B^pu,F^pu)(x_1,\ldots,x_p)\\
g(0,\ldots,0,t)=e
\end{cases};
$$
\item l'esistenza di $w$ è garantita dalla \thref{homotopy-lifting-plus}.
\end{itemize}
Poniamo $Su=v$; si verifica facilmente che le proprierà richieste sono soddisfatte.
\paragraph{Costruzione per $q>0$ e $u$ degenere.} Se $u$ è degenere, poniamo
$$
(Su)(x_1,\ldots,x_{n+1})=(S\lambda^0_nu)(x_1,\ldots,x_n).
$$
Si verifiche facilmente che $Su$ soddisfa le proprietà richieste.
\paragraph{Costruzione per $q>0$ e $u$ non degenere.} Consideriamo il digramma
\begin{diagram}
A\dar{i}\rar{g}&E\dar{p}\\
X\rar{f}\ar[ur,dashed,"v"]&B
\end{diagram}
dove
\begin{itemize}
\item $X=I^p\times I\times I^q$;
\item $A=(I^p\times\{0,1\}\times I^q)\cup(I^p\times I\times\del I^q)\cup(\{s\}\times I\times I^q)$;
\item $f(x_1,\ldots,x_p,t,y_1,\ldots,y_q)=(B^pu)(x_1,\ldots,x_p)$;
\item $g$ è così definita
$$
\begin{cases}
g(x_1,\ldots,x_p,0,y_1,\ldots,y_q)=u(x_1,\ldots,x_p,y_1,\ldots,y_q)\\
g(x_1,\ldots,x_p,1,y_1,\ldots,y_q)=K(B^pu,F^pu)(x_1,\ldots,x_p,y_1,\ldots,y_q)\\
g(x_1,\ldots,x_p,t,y_1,\ldots,y_{i-1},\epsilon,y_i,\ldots,y_{q-1})=(S\lambda^\epsilon_{p+i}u)(x_1,\ldots,x_p,t,y_1,\ldots,y_{q-1})\\
g(0,\ldots,0,t,y_1,\ldots,y_q)=u(0,\ldots,0,y_1,\ldots,y_q)
\end{cases};
$$
\item l'esistenza di $v$ è garantita dalla \thref{homotopy-lifting-plus}, a patto di dimostrare che $g$ è ben definita sulle intersezioni e che $pg=fi$; si tratta in entrambi i casi di facili verifiche, sfruttando l'ipotesi induttiva.
\end{itemize}
Poniamo $Su=v$; si verifica senza difficoltà alcuna che le proprietà richieste sono soddisfatte.
\end{proof}