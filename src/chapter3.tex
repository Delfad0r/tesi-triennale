\chapter{Lemmi a caso}

\todo{Rendere questo capitolo un po' più serio (o inglobarlo altrove)}

\section{Primo lemma}
\begin{proposition}\thlabel{finitely-generated-homology-groups}
Sia $A$ un PID, $\fibration{F}{E}{B}$ una fibrazione in cui $\pi_1(B)$ agisce banalmente sui moduli di omologia di $F$. Supponiamo che tutti i moduli di omologia di $E$ e di $B$ a coefficienti in $A$ siano $A$-moduli finitamente generati. Allora lo stesso vale per $F$.
\end{proposition}
\begin{proof}
Sia $E^{p,q}_r$ la successione spettrale associata alla fibrazione. Mostriamo per induzione su $i$ che $H_i(F;A)$ è un $A$-modulo finitamente generato. Per $i=0$ è ovvio, essendo $F$ connesso per archi. Sia ora $i>0$. Supponiamo per assurdo che $H_i(F;A)=E^{0,i}_2$ non sia finitamente generato. Allora nemmeno $E^{0,i}_3$ è finitamente generato: infatti $E^{0,i}_3$ è il quoziente di $E^{0,i}_2$ per l'immagine del differenziale $\map{d^{2,i-1}_2}{E^{2,i-1}_2}{E^{0,i}_2}$, la quale è finitamente generata in quanto
$$
E^{2,i-1}_2=(H_2(B;A)\tensor H_{i-1}(F;A))\dirsum\Tor(H_1(B;A),H_{i-1}(F;A))
$$
\todo{$\Tor$ di moduli f.g. è f.g.? Sì, se $A$ è PID}
è finitamente generato per ipotesi induttiva. Procedendo allo stesso modo si trova che $E^{0,i}_r$ non è finitamente generato per alcun $r\ge 2$. Ma ciò è assurdo, poiché per $r$ sufficientemente grande $E^{0,i}_r=E^{0,i}_\infty$ è un sottomodulo del modulo graduato associato a $H_i(E;A)$, e quest'ultimo è finitamente generato per ipotesi.
\end{proof}

\section{Una successione esatta}
\begin{proposition}\thlabel{fibration-homology-exact-sequence}
Sia $A$ un PID, $\fibration{F}{E}{B}$ una fibrazione in cui $\pi_1(B)$ agisce banalmente sui moduli di omologia di $F$. Supponiamo che $H_i(B;A)=0$ per $0<i<p$ e che $H_i(F;A)=0$ per $0<i<q$.
\begin{enumerate}
\item Esiste una successione esatta
\begin{diagram}
H_{p+q-1}(F;A)\rar&H_{p+q-1}(E;A)\rar&H_{p+q-1}(B;A)\rar&H_{p+q-2}(F;A)\rar&\ldots\rar&H_2(B;A)\rar&H_1(F;A)\rar&H_1(E;A)\rar&H_1(B;A)\rar&0.
\end{diagram}
\item L'applicazione $\map{p_*}{H_i(E,F;A)}{H_i(B;A)}$ indotta da $p$ è un isomorfismo per $2\le i<p+q$ ed è suriettiva per $i=p+q$.
\end{enumerate}
\end{proposition}
\begin{proof}\leavevmode
\begin{enumerate}
\item Per il teorema dei coefficienti universali vale
$$
E^{i,j}_2=(H_i(B;A)\tensor H_j(F;A))\dirsum\Tor(H_{i-1}(B;A),H_j(F;A)),
$$
da cui $E^{i,j}_2=0$ se $i,j>0$ e $i+j\le p+q-1$. Pertanto, se $0\le n\le p+q-1$, ci sono al più due termini $E^{i,j}_2$ non nulli con $i+j=n$ (ossia $(i,j)=(0,n)$ e $(i,j)=(n,0)$); è inoltre evidente che le altre condizioni di \missing{} sono soddisfatte, dunque possiamo applicarl* (ricordando che $E^{0,n}_2=H_n(F;A)$ e $E^{n,0}=H_n(B;A)$) ottenendo la successione esatta della tesi.
\item Sia $2\le i\le p+q$. Abbiamo visto (\missing{}) che l'immagine di $p_\ast$ in $H_i(B;A)$ è $E^{i,0}_i$. Notiamo che per $2\le r<i$ il differenziale $\map{d^{i,0}_r}{E^{i,0}_r}{E^{i-r,r-1}_r}$ è nullo, in quanto $E^{i-r,r-1}_r$ è nullo (infatti $i-r>0,r-1>0$ e $i-r+r-1=i-1<p+q$). Deduciamo che
$$
H_i(B;A)=E^{i,0}_2=E^{i,0}_3=\ldots=E^{i,0}_i=\im p_\ast,
$$
pertanto $p_\ast$ è suriettiva. Sia ora $2\le i<p+q$; consideriamo il diagramma commutativo
\begin{diagram}
H_i(F;A)\rar\dar{\1}&H_i(E;A)\rar\dar{1}&H_i(E,F;A)\rar\dar{p_\ast}&H_{i-1}(F;A)\rar\dar{\1}&H_{i-1}(E;A)\dar{\1}\\
H_i(F;A)\rar&H_i(E;A)\rar&H_i(B;A)\rar&H_{i-1}(F;A)\rar&H_{i-1}(E;A)
\end{diagram}
La prima riga è la sequenza esatta lunga della coppia $(E,F)$, la seconda deriva dalla prima parte della proposizione, e la commutatività segue da \missing{}. La tesi segue allora dal lemma dei cinque.\todo{Rivedere dopo aver fatto il diagramma del capitolo 1.}
\end{enumerate}
\end{proof}
Naturalmente vale un teorema analogo per i moduli di coomologia.
\begin{corollary}\thlabel{fibration-suspension-isomorphism}
Supponiamo che $H_i(E;A)=0$ per ogni $i>0$ e che $H_i(B;A)=0$ per $0<i<p$. Allora la sospensione $\map{\Sigma}{H_i(F;A)}{H_{i+1}(B;A)}$ è un isomorfismo per $0<i<2p-2$ ed è suriettiva per $i=2p-2$.
\end{corollary}
\begin{proof}
Dalla prima parte della \thref{fibration-homology-exact-sequence} (applicata con $q=1$) segue che $H_i(F;A)=0$ per $0<i<p-1$. Dalla seconda parte (applicata con $q=p-1$) segue immediatamente la tesi, ricordando che $\Sigma=p_\ast\circ\del^{-1}$ (dove $\map{\del}{H_{i+1}(E,F;A)}{H_i(F;A)}$ denota il morfismo di bordo).
\end{proof}