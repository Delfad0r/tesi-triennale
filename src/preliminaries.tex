\chapter*{Notazioni e richiami}
\section*{Topologia generale}
\subsection*{Applicazioni continue}
Se $X,Y$ sono spazi topologici e $A,B$ sono sottospazi rispettivamente di $X,Y$, scriviamo $\map{f}{(X,A)}{(Y,B)}$ per indicare che $f$ è un'applicazione continua da $X$ in $Y$ tale che $f(A)\subs B$.
\begin{definition*}
Siano $X$ uno spazio topologico, $A$ un sottospazio di $X$. Si dice retrazione di $X$ su $A$ un'applicazione $\map{r}{X}{A}$ tale che $r|_A$ è l'identità su $A$.
\end{definition*}
\subsection*{Omotopie}
Indichiamo con $I$ l'intervallo chiuso $[0,1]\subs\RR$ con l'usuale topologia euclidea.
\begin{definition*}
Siano $X,Y$ spazi topologici. Si dice omotopia un'applicazione $\map{F}{X\times I}{Y}$. Data un'omotopia $F$, denotiamo con $F_t$ l'applicazione $F(-,t)$ con $t\in[0,1]$. Si dice che $F$ è un'omotopia fra $F_0$ e $F_1$, e che le applicazione $F_0$ e $F_1$ sono omotope (mediante $F$).
\end{definition*}

\subsection*{Rivestimenti}
\begin{definition*}
\cite[Sezione 1.3]{hatcher}
Si dice rivestimento un'applicazione $\map{p}{T}{X}$ che soddisfa la seguente proprietà: per ogni $x\in X$ esiste un intorno aperto $U\subs X$ di $x$ tale che $p^{-1}(U)$ è unione disgiunta di aperti di $T$, ciascuno dei quali è omeomorfo a $U$ mediante la restrizione di $p$. Un intorno $U$ che soddisfa questa proprietà si dice ben rivestito.
\end{definition*}
\begin{proposition*}
\cite[Proposizione 1.30]{hatcher}
Siano $\map{p}{T}{X}$ un rivestimento, $\map{f}{Y\times I}{X}$ un'omotopia, $\map{\tilde{f}_0}{Y}{T}$ un sollevamento di $f_0$ (ossia $p\tilde{f}_0=f_0$). Allora esiste un'omotopia $\map{\tilde{f}}{Y\times I}{T}$ che solleva $f$.
\begin{diagram}
Y\times\{0\}\dar\rar{\tilde{f}_0}&T\dar{p}\\
Y\times I\rar{f}\ar[ur,dashed,"\tilde{f}"]&X
\end{diagram}
\end{proposition*}
\begin{proposition*}
\cite[Proposizione 1.34]{hatcher}
Siano $\map{p}{T}{X}$ un rivestimento, $Y$ uno spazio topologico connesso, $\map{f}{Y}{X}$ un'applicazione, $\map{\tilde{f},\tilde{f}'}{Y}{T}$ due sollevamenti di $f$. Se $\tilde{f}(y)=\tilde{f}'(y)$ per un qualche punto $y\in Y$, allora $\tilde{f}=\tilde{f}'$.
\end{proposition*}
\begin{definition*}
Un rivestimento $\map{p}{T}{X}$ si dice universale se $T$ è semplicemente connesso.
\end{definition*}
\begin{definition*}
Uno spazio topologico $X$ si dice semilocalmente semplicemente connesso se ogni punto $x\in X$ ammette un intorno $U$ tale che l'inclusione $\map{i}{U}{X}$ induce la mappa banale $\map{i_*}{\pi_1(U,x)}{\pi_1(X,x)}$.
\end{definition*}
\begin{proposition*}
\cite[Sezione 1.3]{hatcher}
Sia $X$ uno spazio topologico connesso e localmente connesso per archi. Allora $X$ ammette un rivestimento universale se e solo se è semilocalmente semplicemente connesso.
\end{proposition*}

\section*{Topologia algebrica}
\subsection{Gruppi di omotopia}
