\chapter{Omologia e coomologia degli spazi fibrati}
\label{chapter-fibration}

\section{Omologia singolare cubica}

L'omologia singolare classica utilizza i simplessi singolari come oggetti fondamentali. Per la teoria degli spazi fibrati dovremo introdurre la nozione di omologia singolare cubica, che impiega cubi in luogo dei simplessi. Come è lecito aspettarsi, i cubi si prestano meglio allo studio degli spazi prodotto, e anche a quello degli spazi fibrati, che ne sono, in un certo senso, una generalizzazione.

Nel seguito indicheremo con $I$ l'intervallo $[0,1]$ con l'usuale topologia euclidea. Sia inoltre $X$ uno spazio topologico.

\begin{definition}
Sia $n\in\NN$. Si dice cubo singolare (o più semplicemente cubo) di dimensione $n$ un'applicazione continua $\map{u}{I^n}{X}$. Un cubo di dimensione $n\ge 1$ si dice degenere se non dipende dall'ultima coordinata, ossia se $u(x_1,\ldots,x_{n-1},x_n)=u(x_1,\ldots,x_{n-1},x_n')$ per ogni $x_1,\ldots,x_n,x_n'\in I$.
\end{definition}

Denotiamo con $Q_n(X)$ il gruppo abeliano libero avente per base l'insieme dei cubi singolari di dimensione $n$, con $D_n(X)$ il gruppo abeliano libero avente per base l'insieme dei cubi degeneri di dimensione $n$. Per definire il complesso $Q_\bullet(X)$ è necessario costruire mappe di bordo $\map{d_n}{Q_n(X)}{Q_{n-1}(X)}$.

Sia $u$ un cubo di dimensione $n$, $p,q\in\NN$ con $p+q=n$, $H$ un sottoinsieme di $\{1,\ldots,n\}$ di cardinalità $p$, $K$ il complementare di $H$, $\varphi_K$ l'unica applicazione strettamente crescente da $K$ in $\{1,\ldots,q\}$; sia inoltre $\epsilon\in\{0,1\}$. Definiamo allora il cubo singolare $\lambda^\epsilon_Hu$ di dimensione $q$:
$$
\lambda^\epsilon_Hu(x_1,\ldots,x_n)=u(y_1,\ldots,y_n)\qquad\text{, dove }y_i=\begin{cases}\epsilon&\text{se $i\in H$}\\x_{\varphi_K(i)}&\text{se $i\in K$}\end{cases}.
$$
Per snellire la notazione, se $H=\{i\}$ (ossia se $p=1$), scriviamo $\lambda^\epsilon_i$ in luogo di $\lambda^\epsilon_{\{i\}}$. Dato un cubo $u$ di dimensione $n$, definiamo dunque
$$
d_nu=\sum_{i=1}^n(\lambda^0_iu-\lambda^1_iu),
$$
estendendola per $\ZZ$-linearità a tutto $Q_n(X)$. È immediato verificare che $\lambda^\epsilon_i\lambda^{\epsilon'}_j=\lambda^{\epsilon'}_{j-1}\lambda^\epsilon_i$; un semplice conto mostra allora che $d_nd_{n+1}=0$. Abbiamo così definito il complesso $Q_\bullet(X)$. Si vede inoltre che $D_\bullet(X)$ è un sottocomplesso di $Q_\bullet(X)$: se $u$ è un cubo degenere di dimensione $n$, allora anche $\lambda^\epsilon_iu$ è degenere per $0\le i<n$, mentre $\lambda^0_nu=\lambda^1_nu$, pertanto $du$ è degenere.
\begin{definition}
Si dice complesso singolare (cubico) di $X$ il complesso $C_\bullet(X)=Q_\bullet(X)/D_\bullet(X)$. I suoi gruppi di omologia e coomologia a coefficienti in un gruppo abeliano $G$ si dicono gruppi di omologia e coomologia singolare (cubica) di $X$ a coefficienti in $G$.
\end{definition}
Poiché nel seguito faremo uso esclusivamente dell'omologia singolare cubica, impiegheremo le notazioni classiche dell'omologia singolare: 
\begin{align*}
C_\bullet(X;G)&=C_\bullet(X)\tensor G\\
H_n(X;G)&=H_n(C_\bullet(X;G))\\
C^\bullet(X;G)&=\Hom(C_\bullet(X),G)\\
H^n(X;G)&=H^n(C^\bullet(X;G)).
\end{align*}
Osserviamo che $C^n(X;G)$ può essere interpretato come il gruppo delle funzioni dai cubi di dimensione $n$ in $G$ che sono nulle sui cubi degeneri.

Siano inoltre $H(X;G)=\Dirsum_{n\ge 0}H_n(X;G)$, $H^*(X;G)=\Dirsum_{n\ge 0}H^n(X;G)$. Esattamente come nel caso della teoria singolare classica, se $G$ è un anello, $H^*(X;G)$ acquisisce una struttura di anello graduato. Si definisce il \cupproduct{} come segue: se $u$ è un cubo di dimensione $p+q$ e $f,g$ sono cocatene di dimensione $p,q$ rispettivamente, allora
$$
(f\cupp g)u=\sum_H\rho_{H,K}f(\lambda^0_Ku)\cdot g(\lambda^1_Hu),
$$
dove $H$ varia fra i sottoinsiemi di $\{1,\ldots,p+q\}$ di cardinalità $p$, $K$ è il complementare di $H$ e $\rho_{H,K}=(-1)^\nu$ ($\nu$ indica il numero di coppie $(i,j)\in H\times K$ con $j<i$). Il \cupproduct{} è ben definito: poiché $f$ e $g$ sono nulle sui cubi degeneri, si vede anche $f\cupp g$ soddisfa la stessa proprietà (se $u$ è degenere, allora anche uno fra $\lambda^0_Ku$ e $\lambda^1_Hu$ lo è). Si verifica poi che $\cupp$ è associativo, e che
$$
d(f\cupp g)=df\cupp g+(-1)^pf\cupp dg,
$$
da cui segue che il \cupproduct{} passa al quoziente, definendo un prodotto in coomologia $H^*(X;G)\times H^*(X;G)\to H^*(X;G)$.

Si può dimostrare che l'approccio dell'omologia cubica conduce ai medesimi risultati dell'omologia singolare classica.

\begin{proposition}\thlabel{cubic-homology-isomorphic-classical-homology}
Denotiamo con $H_\Delta(X;G),H^*_\Delta(X;G)$ l'omologia e la coomologia singolare standard. Allora $H(X;G)\iso H_\Delta(X;G)$ come gruppi graduati, e $H^*(X;G)\iso H_\Delta^*(X;G)$ come anelli graduati.
\end{proposition}
\begin{corollary}\thlabel{cohomology-ring-anticommutative}
Siano $f,g\in H^*(X;G)$ rispettivamente di grado $p$ e $q$. Allora $f\cdot g=(-1)^{pq}g\cdot f$.
\end{corollary}
Studiando più esplicitamente l'isomorfismo fra l'omologia (e la coomologia) cubica e quella singolare classica si può dimostrare quanto segue.
\begin{proposition}\thlabel{cubic-homology-single-vertex}
Supponiamo che $X$ sia connesso per archi; sia $x\in X$ un punto fissato. Allora i gruppi di omologia e coomologia (cubica)  di $X$ rimangono inalterati se ci si limita a considerare cubi singolari aventi tutti i vertici in $x$.
\end{proposition}
\todo{Ma serve davvero?}

\section{Spazi fibrati}
\begin{definition}
Un'applicazione continua suriettiva $\map{p}{E}{B}$ si dice fibrazione se soddisfa la seguente proprietà (sollevamento dell'omotopia per poliedri finiti): dati un poliedro finito $P$ e due applicazioni continue $\map{f}{I\times P}{B},\map{g}{P}{E}$ tali che $pg=fi$ (dove $i$ denota l'inclusione $\map{i}{P}{I\times P}$ definita da $i(x)=(0,x)$), esiste un'applicazione continua $\map{h}{I\times P}{E}$ tale che $ph=f$ e $hi=g$.
\begin{diagram}
P\rar{g}\dar{i}&E\dar{p}\\
I\times P\rar{f}\ar[ru,dashed,"h"]&B
\end{diagram}
\end{definition}
Se $\map{p}{E}{B}$ è una fibrazione, chiameremo $E$ spazio totale e $B$ spazio base. Inoltre la terna $(p, E, B)$ è detta spazio fibrato; con tale espressione potremo anche indicare, con lieve abuso di notazione, il solo spazio $E$.

In realtà il sollevamento dell'omotopia per poliedri finiti implica una proprietà più forte.
\begin{proposition}\thlabel{homotopy-lifting-plus}
Sia $\map{p}{E}{B}$ una fibrazione, $A\subs X$ due poliedri finiti; indichiamo con $\map{i}{A}{X}$ l'inclusione. Siano $\map{f}{X}{B},\map{g}{A}{E}$ applicazioni continue tali che $pg=fi$. Allora esiste un'applicazione continua $\map{h}{X}{E}$ tale che $ph=f$ e $hi=g$.
\begin{diagram}
A\rar{g}\dar{i}&E\dar{p}\\
X\rar{f}\ar[ru,dashed,"h"]&B
\end{diagram}
\end{proposition}
\begin{proof}
\begin{lemma}
La proposizione è vera se $X=A\times I^n$ per un qualche $n\ge 0$ e $i(x)=(x,0)$.
\end{lemma}
\begin{proof}

\end{proof}
\end{proof}
\begin{proposition}\thlabel{fibration-homotopy-exact-sequence}
Sia $\map{p}{E}{B}$ una fibrazione, $e\in E,b=p(e),F=p^{-1}(e)$.
\begin{enumerate}
\item La mappa $p$ induce un isomorfismo $\map{p_*}{\pi_i(E,F,e)}{\pi_i(B,b)}$ per ogni $i\ge 1$.
\item Esiste una successione esatta lunga di gruppi
\begin{diagram}
\ldots\rar&\pi_{i+1}(E,e)\rar&\pi_{i+1}(B,b)\rar&\pi_i(F,e)\rar&\pi_i(E,e)\rar&\ldots\rar&\pi_1(E,e)\rar&\pi_1(B,b)
\end{diagram}
\end{enumerate}
\end{proposition}
\begin{proof}

\end{proof}

\begin{proposition}\thlabel{fibration-path-connected-base-and-fiber}
Sia $\map{p}{E}{B}$ una fibrazione, $e\in E,b=p(e),F=p^{-1}(e)$. Supponiamo che $B$ e $F$ siano connessi per archi. Allora anche $E$ e tutte le altre fibre sono connessi per archi.
\end{proposition}
\begin{proof}

\end{proof}
D'ora in poi considereremo solo fibrazioni con spazio base e fibre connessi per archi. Per la \thref{cubic-homology-single-vertex} possiamo limitarci a considerare cubi con vertici in un singolo punto fissato nello studio dell'omologia e della coomologia. Nel seguito supporremo dunque implicitamente che i cubi in $F$ ed $E$ abbiano tutti i vertici in un punto fissato $e$, e che i cubi in $B$ abbiano tutti i vertici in $b=p(e)$.

\section{Azione del gruppo fondamentale di \texorpdfstring{$B$}{B} sull'omologia della fibra}

Ci proponiamo ora di mostrare come il gruppo fondamentale di $B$ agisca sui gruppi di omologia e coomologia di $F$.

\begin{definition}
Sia $\gamma$ un cammino chiuso in $B$ con estremi in $b$, $T$ un'applicazione che a ogni cubo $u$ di dimensione $n$ di $F$ associa un cubo $Tu$ di $E$ di dimensione $n+1$. $T$ si dice costruzione subordinata a $\gamma$ se soddisfa le seguenti proprietà per ogni cubo $u$ di dimensione $n$:
\begin{enumerate}
\item $\lambda^0_1 Tu=u$;
\item $(p\circ Tu)(t,t_1,\ldots,t_n)=\gamma(t)$ per ogni $t_1,\ldots,t_n\in I$;
\item $T\lambda^\epsilon_iu=\lambda^\epsilon_{i+1}Tu$ per $0\le i\le n,\epsilon\in\{0,1\}$;
\item se $u$ è degenere, allora anche $Tu$ lo è.
\end{enumerate}
\end{definition}
Ogni costruzione $T$ induce un morfismo di complessi $\map{S_T}{C_\bullet(F)}{C_\bullet(F)}$ definito da $(S_Tu)(t)=(Tu)(1,t)$. Le proprietà delle costruzioni garantiscono che $S_Tu$ è effettivamente un cubo di $F$, che cubi degeneri vengono mandati in cubi degeneri e che $S_T$ commuta con la mappa di bordo. A sua volta, $S_T$ induce endomorfismi dei gruppi di omologia e coomologia di $F$.
\begin{proposition}\thlabel{subordinate-construction-homotopy-equivalence}
Siano $\gamma_1,\gamma_2$ cammini chiusi in $B$ con estremi in $b$, $T_1,T_2$ costruzioni subordinate rispettivamente a $\gamma_1,\gamma_2$. Supponiamo che $\gamma_1,\gamma_2$ rappresentino lo stesso elemento del gruppo fondamentale. Allora i morfismi di complessi $S_{T_1}$ e $S_{T_2}$ sono omotopi.
\end{proposition}
\begin{proof}

\end{proof}
Si potrebbe dimostrare che per ogni cammino $\gamma$ esiste una costruzione subordinata a $\gamma$, e che l'applicazione $\pi_1(B,b)\to\Aut(H_n(F))$ è un omomorfismo di gruppi, ma non utilizzeremo questi risultati. \todo{Serre lo fa però.} 
Ci limitiamo a dimostrare quanto segue.
\begin{proposition}\thlabel{trivial-path-trivial-action}
Sia $\gamma$ un cammino chiuso in $B$ con estremi in $b$, $T$ una costruzione subordinata a $\gamma$. Supponiamo che $\gamma$ sia omotopicamente banale. Allora $S_T$ induce l'identità in omologia e in coomologia.
\end{proposition}
\begin{proof}

\end{proof}
Motivati dalla proposizione precedente, ci limiteremo spesso a studiare fibrazioni in cui l'azione di $\pi_1(B)$ sui gruppi di omologia e coomologia di $F$ è banale (con questa espressione intendiamo che per ogni costruzione $T$ subordinata a un qualche cammino il morfismo $S_T$ induce l'identità in omologia e in coomologia).
\begin{corollary}\thlabel{simply-connected-trivial-action}
Se $B$ è semplicemente connesso, allora $\pi_1(B)$ agisce banalmente sui gruppi di omologia e coomologia di $F$.
\end{corollary}

\section{Successione spettrale di uno spazio fibrato}
Per applicare i risultati della Sezione \ref{sec:spectral-sequence-filtered-complex}, è necessario definire una filtrazione crescente sul complesso singolare $C_\bullet(E)$ (d'ora in poi ometteremo la $E$ dell'argomento). Ciò che faremo sarà filtrare il complesso $Q_\bullet$ con dei sottocomplessi $Q^p_\bullet$ e prenderne le immagini nel quoziente $C_\bullet$. Sia dunque $Q^p_n$ il sottogruppo di $Q_n$ generato dai cubi $u\in Q_n$ tali che $p\circ u$ dipende solo dalle prime $p$ coordinate (e $Q^p_n=Q_n$ se $p>n$). Si vede immediatamente che i $Q^p_\bullet$ sono sottocomplessi di $Q_\bullet$ e che soddisfano le proprietà di una filtrazione crescente. Dunque lo stesso vale per $C^p_\bullet=(Q^p_\bullet+D_\bullet)/D_\bullet$, che definiscono una filtrazione crescente per $C_\bullet$. Applicando la \thref{spectral-sequence-of-filtered-complex} otteniamo una successione spettrale $E^{p,q}_r\converges H_{p+q}(E)$. Come vedremo, è possibile calcolare esplicitamente i termini $E^{p,q}_2$ della successione spettrale in funzione dei gruppi di omologia di $B$ e di $F$.

Costruiamo due applicazioni $B^p$ e $F^p$ definite sui cubi di $Q^p_\bullet$ (ed estese per $\ZZ$-linearità a tutto $Q^p_\bullet$). Se $u\in\ Q^p_n$ è un cubo di dimensione $n$ con $n\ge p$, posto $q=n-p$, definiamo
\begin{align*}
(B^pu)(t_1,\ldots,t_p)&=pu(t_1,\ldots,t_p,0,\ldots,0);\\
(F^pu)(t_1,\ldots,t_q)&=u(0,\ldots,0,t_1,\ldots,t_q).
\end{align*}
$B^pu$ è un cubo di $B$ di dimensione $p$; notiamo che, poiché $u\in Q^p_n$, possiamo sostituire gli zeri nella definizione con qualunque altra $q$-upla di reali fra $0$ e $1$. $F^pu$ è invece un cubo di $F$ di dimensione $q$; la sua immagine è contenuta in $F$ poiché
$$
p(F^pu)(t_1,\ldots,t_q)=pu(0,\ldots,0,t_1,\ldots,t_q)=pu(0,\ldots,0)=b.
$$
Le seguenti proprietà sono di verifica immediata.
\begin{enumerate}
\item\label{spectral-sequence-of-fibration:pr1} $B^pu$ è degenere se e solo se $u\in Q^{p-1}_n$.
\item\label{spectral-sequence-of-fibration:pr2} Se $u$ è degenere e $q>0$ allora $F^pu$ è degenere; se $u$ è degenere e $q=0$ allora $B^pu$ è degenere.
\item\label{spectral-sequence-of-fibration:pr3} Se $i>p,\epsilon\in\{0,1\}$ allora $B^p\lambda^\epsilon_iu=B^pu$ e $F^p\lambda^\epsilon_iu=\lambda^\epsilon_{i-p}F^pu$.
\end{enumerate}

Ricordiamo che $E^{p,q}_0=C^p_{p+q}/C^{p-1}_{p+q}$ e che la mappa $\map{d^{p,q}_0}{E^{p,q}_0}{E^{p,q-1}_0}$ si ottiene dalla mappa di bordo di $C_\bullet$ per passaggio al quoziente. Possiamo dunque raggruppare i termini della successione spettrale aventi $r=0$ in complessi $E^{p,\bullet}_0$, con mappe di bordo $d^p_0$. Osserviamo che
$$
d^p_0u=\sum_{i>p}(-1)^i(\lambda^0_iu-\lambda^1_iu)
$$
in quanto per $i\le p$ vale $\lambda^\epsilon_iu\in Q^p_\bullet$.

Consideriamo il complesso $J^p_\bullet=C_p(B)\tensor C_\bullet(F)$ con mappe di bordo $d_J(b\tensor f)=(-1)^pb\tensor df$, e sia $\varphi^p$ il morfismo di complessi
\Map{\varphi^p}{E^{p,\bullet}_0}{J^p_\bullet}{u}{B^pu\tensor F^pu}
Le proprietà \ref{spectral-sequence-of-fibration:pr1} e \ref{spectral-sequence-of-fibration:pr2} garantiscono che si tratta di una buona definizione (ossia che è compatibile con il quoziente che definisce $E^{p,\bullet}_0$), mentre dalla \ref{spectral-sequence-of-fibration:pr3} segue che $\varphi^p$ è effettivamente un morfismo di complessi:
\begin{align*}
\varphi^pd^p_0u&=\varphi^p\sum_{i>p}(-1)^i(\lambda^0_iu-\lambda^1_iu)\\
&=\sum_{i>p}(-1)^i(B^p\lambda^0_iu\tensor F^p\lambda^0_iu-B^p\lambda^1_iu\tensor F^p\lambda^1_iu)\\
&=\sum_{i>p}(-1)^pB^pu\tensor(-1)^{p-i}(\lambda^0_{i-p}F^pu-\lambda^1_{i-p}F^pu)\\
&=(-1)^pB^pu\tensor dF^pu\\
&=d_J\varphi^pu.
\end{align*}

Ci proponiamo di mostrare che $\varphi^p$ è in realtà un'equivalenza omotopica. Sfrutteremo il seguente lemma per costruire un'inversa omotopica.

\begin{lemma}\thlabel{spectral-sequence-of-fibration-psi-construction}
Esiste un'applicazione $K$ che associa a ogni coppia di cubi $(u,v)$ con $u\in Q_p(B)$ e $v\in Q_q(F)$ un cubo $K(u,v)\in Q^p_{p+q}(E)$ e che soddisfa le seguenti proprietà:
\begin{enumerate}
\item\label{spectral-sequence-of-fibration-psi-construction:pr1} $B^pK(u,v)=u$ e $F^pK(u,v)=v$;
\item\label{spectral-sequence-of-fibration-psi-construction:pr2} per ogni $i\le q,\epsilon\in\{0,1\}$ vale $K(u,\lambda^\epsilon_iv)=\lambda^\epsilon_{i+p}K(u,v)$;
\item\label{spectral-sequence-of-fibration-psi-construction:pr3} se $v$ è degenere allora $K(u,v)$ è degenere.
\end{enumerate}
\end{lemma}
\begin{proof}

\end{proof}

Definiamo allora il morfismo di complessi
\Map{\psi^p}{J^p_\bullet}{E^{p,\bullet}_0}{u\tensor v}{K(u,v)}
Osserviamo che $\psi^p$ è ben definito (ossia è compatibile con i quozienti che definiscono $J^p_\bullet$): infatti se $u$ è degenere allora $B^pK(u,v)$ è degenere per la \ref{spectral-sequence-of-fibration-psi-construction:pr1}, dunque $K(u,v)\in Q^{p-1}_\bullet$. Se invece $v$ è degenere allora $K(u,v)$ è degenere per la \ref{spectral-sequence-of-fibration-psi-construction:pr3}. Inoltre $\psi^p$ è effettivamente un morfismo di complessi, come mostra un semplice verifica:
\begin{align*}
\psi^pd_J(u\tensor v)&=(-1)^p\psi^p(u\tensor dv)\\
&=(-1)^p\psi^p\left(u\tensor\sum_{i=1}^q(-1)^i(\lambda^0_iv-\lambda^1_iv)\right)\\
&=\sum_{i=1}^q(-1)^{p+i}(K(u,\lambda^0_iv)-K(u,\lambda^1_iv))\\
&=\sum_{i=p+1}^n(-1)^i(\lambda^0_iK(u,v)-\lambda^1_iK(u,v))\\
&=d^p_0\psi^p(u\tensor v).
\end{align*}

\begin{proposition}\thlabel{spectral-sequence-of-fibration-homotopy-equivalence}
Per ogni $p$ la mappa $\map{\varphi^p}{E^{p,\bullet}_0}{J^p_\bullet}$ è un'equivalenza omotopica, di cui $\map{\psi^p}{J^p_\bullet}{E^{p,\bullet}_0}$ è un'inversa omotopica.
\end{proposition}
\begin{proof}
Si vede immediatamente che $\varphi^p\psi^p=\1$. Per dimostrare che $\psi^p\varphi^p$ è omotopo all'identità ci serviremo del seguente lemma.
\begin{lemma*}
Esiste un'applicazione $S$ \todo{Forse $S$ non è il nome più adatto?} che associa a ogni cubo $u\in Q^p_n(E)$ un cubo $Su\in Q^p_{n+1}$ e che soddisfa le seguenti proprietà:
\begin{enumerate}
\item\label{spectral-sequence-of-fibration-homotopy-equivalence:pr1} $B^pSu=B^pu$;
\item\label{spectral-sequence-of-fibration-homotopy-equivalence:pr2} $\lambda^0_{p+1}Su=u$ e $\lambda^1_{p+1}Su=K(B^pu,F^pu)$;
\item\label{spectral-sequence-of-fibration-homotopy-equivalence:pr3} se $i>p$ vale $S\lambda^\epsilon_iu=\lambda^\epsilon_{i+1}Su$;
\item\label{spectral-sequence-of-fibration-homotopy-equivalence:pr4} se $q>0$ e $u$ è degenere, allora $Su$ è degenere.
%\item $Su(0,\ldots,0,t,x_1,\ldots,x_q)=u(0,\ldots,0,x_1,\ldots,x_q)$.
\end{enumerate}
\end{lemma*}
\begin{proof}

\end{proof}
Dato un cubo $u\in E^{p,q}_0$, definiamo $hu=(-1)^pSu$, ed estendiamo $h$ a una mappa
$\map{h}{E^{p,q}_0}{E^{p,q+1}_0}$ per $\ZZ$-linearità. Osserviamo che $h$ è ben definita (ossia è compatibile con il quoziente che definisce $E^{p,q}_0$): infatti se $u\in Q^{p-1}_n(E)$ allora $Su\in Q^{p-1}_{n+1}(E)$ per la proprietà \ref{spectral-sequence-of-fibration-homotopy-equivalence:pr1}; se $u$ è degenere e $q>0$ allora $Su$ è degenere (proprietà \ref{spectral-sequence-of-fibration-homotopy-equivalence:pr4}), mentre se $u$ è degenere e $q=0$ allora $B^pSu$ è degenere (proprietà \ref{spectral-sequence-of-fibration-homotopy-equivalence:pr1}), dunque $Su\in Q^{p-1}_{n+1}(E)$.

Mostriamo ora che $h$ definisce un'omotopia fra $\psi^p\varphi^p$ e l'identità. Calcoliamo
\begin{align*}
d^p_0hu+hd^p_0u&=(-1)^p(d^p_0Su+Sd^p_0u)\\
&=(-1)^p\left(\sum_{i=p+1}^{n+1}(-1)^i(\lambda^0_iSu-\lambda^1_iSu)+\sum_{i=p+1}^n(-1)^i(S\lambda^0_iu-\lambda^1_iu)\right)\\
&=(-1)^p(-1)^{p+1}(u-K(B^pu,F^pu))+\\
&\phantomop(-1)^p\sum_{i=p+2}^{n+1}(-1)^i(S\lambda^0_{i-1}u-S\lambda^1_{i-1}u)+\\
&\phantomop(-1)^p\sum_{i=p+1}^n(-1)^i(S\lambda^0_iu-\lambda^1_iu)\\
&=K(B^pu,F^pu)-u\\
&=\psi^p\varphi^pu-u.
\end{align*}
Questo conclude la dimostrazione.
\end{proof}

Da questa proposizione segue che le mappe indotte in omologia $\map{\varphi_*}{H_q(E^{p,\bullet}_0)}{H_q(J^p_\bullet)}$ sono isomorfismi. Osserviamo però che $H_q(E^{p,\bullet}_0)=E^{p,q}_1$, mentre dal fatto che $C_p(B)$ è un gruppo abeliano libero segue
$$
H_q(J^p_\bullet)=H_q(C_p(B)\tensor C_\bullet(F))=C_p(B)\tensor H_q(F)=C_p(B;H_q(F)).
$$
Abbiamo dunque dimostrato la seguente.
\begin{proposition}\thlabel{spectral-sequence-of-fibration-E-1}
I morfismi di complessi $\varphi^p$ inducono isomorfismi $\map{\varphi_*}{E^{p,q}_1}{C_p(B;H_q(F))}$.
\end{proposition}

Per calcolare i termini $E^{p,q}_2$ della successione spettrale, è necessario studiare come si trasformano le mappe di bordo $d^{p,q}_1$ mediante l'isomorfismo $\varphi_*$.
\begin{proposition}\thlabel{spectral-sequence-of-fibration-boundary-map}
Supponiamo che il gruppo fondamentale di $B$ agisca banalmente sui gruppi di omologia di $F$. Allora il seguente diagramma commuta
\begin{diagram}
E^{p,q}_1\rar{d^{p,q}_1}\dar{\varphi_*}&E^{p-1,q}_1\dar{\varphi_*}\\
C_p(B;H_q(F))\rar{d}&C_{p-1}(B;H_q(F))
\end{diagram}
dove abbiamo denotato con $d$ la mappa di bordo canonica del complesso singolare.
\end{proposition}
\begin{proof}
Consideriamo un elemento $x=b\tensor h\in C_p(B;H_q(F))$, dove $b$ è un cubo di $B$ di dimensione $p$; sia $m\in C_q(F)$ un ciclo della classe di omologia di $h$. Ci proponiamo di calcolare $\varphi^{p-1}d\psi^p(b\tensor m)$; per passaggio al quoziente otterremo $\varphi_*d^{p,q}_1\varphi_*^{-1}x$. Scriviamo $m$ come somma di cubi: $m=\sum_{\alpha}g_\alpha u_\alpha$ con $g_\alpha\in\ZZ$. Calcoliamo
$$
d\psi^p(b\tensor m)=\sum_\alpha\sum_{i=1}^n(-1)^ig_\alpha(\lambda^0_iK(b,u_\alpha)-\lambda^1_iK(b,u_\alpha)).
$$
Osserviamo però che
\begin{align*}
\sum_{\alpha}\sum_{i=p+1}^n(-1)^ig_\alpha(\lambda^0_iK(b,u_\alpha)-\lambda^1_iK(b,u_\alpha))&=\sum_{\alpha}\sum_{i=p+1}^n(-1)^ig_\alpha(K(b,\lambda^0_{i-p}u_\alpha)-K(b,\lambda^1_{i-p}u_\alpha))\\
&=\sum_\alpha(-1)^pg_\alpha\psi^p(b\tensor du_\alpha)\\
&=(-1)^p\psi^p(b\tensor dm)\\
&=0,
\end{align*}
da cui
$$
d\psi^p(b\tensor m)=\sum_\alpha\sum_{i=1}^p(-1)^ig_\alpha(\lambda^0_iK(b,u_\alpha)-\lambda^1_iK(b,u_\alpha)).
$$
Poiché $K(b,u_\alpha)\in Q^p_{p+q}$, abbiamo che $\lambda^\epsilon_iK(b,u_\alpha)\in Q^{p-1}_{p+q-1}$, dunque possiamo applicare $\varphi^{p-1}$ ai singoli termini della sommatoria, ottenendo
$$
\varphi^{p-1}d\psi^p(b\tensor m)=\sum_\alpha\sum_{i=1}^p\sum_{\epsilon=0}^1(-1)^{i+\epsilon}g_\alpha B^{p-1}\lambda^\epsilon_iK(b,u_\alpha)\tensor F^{p-1}\lambda^\epsilon_iK(b,u_\alpha).
$$
Si vede facilmente che $B^{p-1}\lambda^{\epsilon}_iK(b,u_\alpha)=\lambda^\epsilon_i B^pK(b,u_\alpha)=\lambda^\epsilon_ib$. Per quanto riguarda $F^{p-1}\lambda^\epsilon_iK(b,u_\alpha)$ abbiamo invece
$$
(F^{p-1}\lambda^\epsilon_iK(b,u_\alpha))(x_1,\ldots,x_q)=K(b,u_\alpha)(0,\ldots,0,\epsilon,0,\ldots,0,x_1,\ldots,x_q)
$$
dove $\epsilon$ si trova in posizione $i$. Consideriamo l'applicazione $T$ che a ogni cubo $u$ di $F$ di dimensione $q$ associa un cubo $Tu$ di $E$ di dimensione $q+1$ definito da
$$
Tu(t,x_1,\ldots,x_q)=K(b,u)(0,\ldots,0,t\epsilon,0,\ldots,0,x_1,\ldots,x_q)
$$
dove $t\epsilon$ si trova di nuovo in posizione $i$. Si verifica facilmente che $T$ è una costruzione subordinata al cammino $t\mapsto b(0,\ldots,0,t\epsilon,0,\ldots,0)$. È inoltre evidente che $S_Tu=F^{p-1}\lambda^\epsilon_iK(b,u)$. Se denotiamo con $S_{i,\epsilon,b}$ il morfismo $S_T$ associato all'applicazione $T$ determinata dai valori fissati di $i,\epsilon$ e $b$, possiamo scrivere
\begin{align*}
\varphi^{p-1}d\psi^p(b\tensor m)&=\sum_\alpha\sum_{i=1}^p(-1)^{i}g_\alpha(\lambda^0_ib\tensor S_{i,\epsilon,b}u_\alpha-\lambda^1_ib\tensor S_{i,\epsilon,b}u_\alpha)\\
&=\sum_{i=1}^p(-1)^i(\lambda^0_ib\tensor S_{i,\epsilon,b}m-\lambda^1_ib\tensor S_{i,\epsilon,b}m).
\end{align*}
Passando al quoziente, ricordando che $S_{i,\epsilon, b}$ induce l'identità in omologia, otteniamo infine
$$
\varphi_*d^{p,q}_1\varphi_*^{-1}x\varphi_*d^{p,q}_1=\varphi_*^{-1}(b\tensor h)=db\tensor h.
$$
\end{proof}
Da questa proposizione segue immediatamente il risultato conclusivo della sezione.
\begin{proposition}\thlabel{spectral-sequence-of-fibration-E-2}
Supponiamo che il gruppo fondamentale di $B$ agisca banalmente sui gruppi di omologia di $F$. Allora $\varphi_*$ induce un isomorfismo $E^{p,q}_2\iso H_p(B;H_q(F))$.
\end{proposition}

Osserviamo che i risultati ottenuti fino a questo punto rimangono validi se consideriamo l'omologia a coefficienti in un gruppo abeliano $G$: a ogni spazio fibrato si può associare una successione spettrale $E^{p,q}_r$ (derivante dall'omologia a coefficienti in $G$) i cui termini con $r=2$ sono isomorfi a $H_p(B;H_q(F;G))$.

\section{Successione spettrale coomologica di uno spazio fibrato}
Come è lecito aspettarsi, dato uno spazio fibrato è possibile associarvi anche una successione spettrale coomologica. Sia $G$ un gruppo abeliano. Partendo dal complesso filtrato $C_\bullet$ procediamo come mostrato in \missing{}, costruendo il cocomplesso $C^\bullet=\Hom(C_\bullet, G)$ e filtrandolo mediante i sottocomplessi $C^\bullet_p=\Ann C^{p-1}_\bullet$. Otteniamo così una successione spettrale coomologica $E^{\ast p,q}_r \converges H^{p+q}(E)$.

Osserviamo che
$$
E^{\ast p,q}_0=\frac{\Ann C^{p-1}_n}{\Ann C^p_n}=\frac{\Hom(C_n/C^{p-1}_n,G)}{\Hom(C_n/C^p_n,G)},
$$
ma, poiché la successione esatta
\begin{diagram}
0\rar&C^p_n/C^{p-1}_n\rar&C_n/C^{p-1}_n\rar&C_n/C^p_n\rar&0
\end{diagram}
spezza, otteniamo
$$
E^{\ast p,q}_0=\Hom(C^p_n/C^{p-1}_n,G)=\Hom(E^{p,q}_0,G).
$$
Possiamo quindi interpretare gli elementi di $E^{\ast p,q}_0$ come funzioni che associano a ogni cubo di $Q^p_n$ un elemento di $G$ e che sono nulle sui cubi degeneri e su quelli di $Q^{p-1}_n$.

Ripercorriamo brevemente i passi della sezione precedente, applicati questa volta a $E^{\ast p,q}_r$. Per ogni $p$ definiamo il complesso $J^\bullet_p=\Hom(J^p_\bullet, G)$ e i morfismi $\map{\varphi_p}{J^\bullet_p}{E^{\ast p,\bullet}_0}$, $\map{\psi_p}{E^{\ast p,\bullet}_0}{J^\bullet_p}$, trasposti di $\varphi^p$ e $\psi^p$. Naturalmente $\varphi_p$ è un'equivalenza omotopica e $\psi_p$ è una sua inversa omotopica, pertanto il morfismo indotto $\varphi_*$ è un isomorfismo fra $E^{\ast p,q}_1$ e $H^q(J^\bullet_p)$. Poiché $C_p(B)$ è un gruppo abeliano libero, vale
$$
H^q(J^\bullet_p)=H^q(\Hom(C_p(B),\Hom(C_\bullet(F),G)))=\Hom(C_p(B),H^q(F;G))=C^p(B;H^q(F;G)).
$$
Si mostra poi, in modo del tutto analogo a quanto fatto per la successione spettrale omologica, che le mappe di bordo $d^{\ast p,q}_1$ vengono mandate da $\psi^*$ nei morfismi di bordo standard di $C^\bullet(B;H^q(F;G))$, da cui si ottiene il risultato "duale" della \thref{spectral-sequence-of-fibration-E-2}.
\begin{proposition}\thlabel{cohomological-spectral-sequence-of-fibration-E-2}
Supponiamo che il gruppo fondamentale di $B$ agisca banalmente sui gruppi di coomologia di $F$. Allora $\psi^*$ induce un isomorfismo $E^{\ast p,q}_2\iso H^p(B;H^q(F;G))$.
\end{proposition}
Supponiamo ora che $G$ sia anche un anello. Allora i gruppi di coomologia acquistano una struttura di anello, e lo stesso accade per i gruppi $E^{\ast p,q}_r$. Infatti il \cupproduct{} definisce una struttura di anello su $C^\bullet(E;G)$; si verifica facilmente che tale prodotto soddisfa la condizione $C^n_p\cupp C^m_q\subs C^{m+n}_{p+q}$, dunque come mostrato in \missing{} i termini della successione spettrale $E^{\ast p,q}_r$ ereditano una struttura di anello per la quale le mappe di bordo sono antiderivazioni. È naturale domandarsi come questa struttura si trasformi mediante l'isomorfismo della \thref{cohomological-spectral-sequence-of-fibration-E-2}. Osserviamo che anche su $H^p(B;H^q(F;G))$ è definita una struttura di anello: infatti $H^*(F;G)$ è un anello (con il prodotto dato dal \cupproduct{} su $F$), dunque su $H^*(B;H^*(F;G))$ è definito il prodotto dato dal \cupproduct{} su $B$.

\begin{proposition}\thlabel{cohomological-spectral-sequence-of-fibration-ring}
Siano $g\in E^{\ast p,q}_2,g'\in E^{\ast p',q'}_2$. Allora $\psi^*(g\cdot g')=(-1)^{p'q}\psi^*(g)\cdot\psi^*(g')$.
\end{proposition}
\begin{proof}
Cerchiamo di studiare più approfonditamente la struttura di anello indotta dal \cupproduct{} su $C^\bullet(B;H^*(F;G))$. Innanzitutto, poiché $C^p(B;H^*(F;G))=H^*(J^\bullet_p)$, è sufficiente calcolare il prodotto fra elementi di $H^q(J^\bullet_p)$. Siano dunque $\bar{f}\in H^q(J^\bullet_p),\bar{f}'\in H^{q'}(J^\bullet_{p'})$, e consideriamo $f\in J^q_p,f'\in J^{q'}_{p'}$ rappresentanti delle classi di coomologia rispettivamente di $\bar{f},\bar{f}'$. Possiamo interpretare $f$ come una funzione che a ogni coppia di cubi $u\in C_p(B),v\in C_q(F)$ associa un elemento $f(u,v)\in G$; inoltre $f(u,v)=0$ ogniqualvolta uno fra $u$ e $v$ è degenere. Allora il prodotto $f\cupp f'$ appartiene a $J^{q+q'}_{p+p'}$, e sviluppando il \cupproduct{} su $B$ e su $F$ si ottiene che
$$
(f\cupp f')(u,v)=\sum_{H,M}\rho_{H,L}\rho_{M,N}f(\lambda^0_Lu,\lambda^0_Nv)\cdot f'(\lambda^1_Hu,\lambda^1_Mv),
$$
dove $H$ varia fra i sottoinsiemi di $\{1,\ldots,p+q\}$ di cardinalità $p$, $M$ fra quelli di $\{1,\ldots,p'+q'\}$ di cardinalità $p'$, $L$ e $N$ sono i complementari di $H$ e $M$, e $\rho_{H,L},\rho_{M,N}$ sono stati definiti insieme al \cupproduct{} in \missing{}. Calcoliamo allora
\begin{align*}
\psi_{p+p'}(\varphi_p(f)\cupp\varphi_{p'}(f'))(u,v)&=(\varphi_p(f)\cupp\varphi_{p'}(f'))K(u,v)\\
&=\sum_P\rho_{P,Q}\varphi_p(f)(\lambda^0_QK(u,v))\cdot\varphi_{p'}(f')(\lambda^1_PK(u,v)),
\end{align*}
dove $P$ varia fra i sottoinsiemi di $\{1,\ldots,p+p'+q+q'\}$ di cardinalità $p+q$ e $Q$ è il complementare di $P$. Ricordiamo che $\varphi_p(f)$, come ogni elemento di $E^{p,q}_0$, è una funzione che a ogni cubo di $Q^p_n$ associa un elemento di $G$, ed è nulla sui cubi degeneri e su quelli che appartengono a $Q^{p-1}_n$, dove al solito $n=p+q$. Poiché $K(u,v)\in Q^{p+p'}_{n+n'}$, se $P\cap\{1,\ldots,p+p'\}$ ha meno di $p$ elementi allora $\lambda^0_QK(u,v)\in Q^{p-1}_{n}$, dunque $\varphi_{p}(f)(\lambda^0_QK(u,v))=0$. Analogamente se $P\cap\{1,\ldots,p+p'\}$ ha più di $p$ elementi allora $\varphi_{p'}(f')(\lambda^1_PK(u,v))=0$. Pertanto possiamo limitarci a sommare sui $P$ della forma $P=H\cup(p+p'+M)$, dove $H\subs\{1,\ldots,p+p'\}$ ha cardinalità $p$ e $M\subs\{1,\ldots,q+q'\}$ ha cardinalità $q$. Denotiamo con $L,N$ i complementari di $H,M$. Le seguenti formule sono di verifica immediata ($w$ denota un qualunque cubo di $Q^{p+p'}_{n+n'}$):
\begin{itemize}
\item $B^p\lambda^0_Qw=\lambda^0_LB^{p+p'}w$;
\item $B^{p'}\lambda^1_Pw=\lambda^1_HB^{p+p'}w$;
\item $F^p\lambda^0_Qw=\lambda^0_NF^{p+p'}w$;
\item $F^{p'}\lambda^1_Pw=\lambda^{0,1}_{L,H}\lambda^1_{p+p'+M}w$, dove l'operazione $\lambda^{0,1}_{L,H}$ consiste nel rimpiazzare le coordinate con indice in $L$ con $0$ e quelle con indice in $H$ con $1$.
\end{itemize}
Inoltre, osservando che
\begin{align*}
\rho_{P,Q}&=\rho_{H,L}\cdot\rho_{H,p+p'+N}\cdot\rho_{p+p'+M,L}\cdot\rho_{p+p'+M,p+p'+N}\\
&=\rho_{H,L}\cdot (-1)^0\cdot(-1)^{qp'}\cdot\rho_{M,N}\\
&=(-1)^{p'q}\rho_{H,L}\rho_{M,N} 
\end{align*}
Otteniamo dunque
\begin{align*}
&\phantomop\psi_{p+p'}(\varphi_p(f)\cupp\varphi_{p'}(f'))(u,v)\\
&=\sum_{H,M}(-1)^{p'q}\rho_{H,L}\rho_{M,N} f(B^p\lambda^0_QK(u,v),F^p\lambda^0_QK(u,v))\cdot f'(B^{p'}\lambda^1_PK(u,v),F^{p'}\lambda^1_PK(u,v))\\
&=\sum_{H,M}(-1)^{p'q}\rho_{H,L}\rho_{M,N}f(\lambda^0_Lu,\lambda^0_Nv)\cdot f'(\lambda^1_Hu,\lambda^{0,1}_{L,H}K(u,\lambda^1_Mv))
\end{align*}
\end{proof}