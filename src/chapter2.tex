\chapter{Omologia e coomologia degli spazi fibrati}

\section{Omologia singolare cubica}

L'omologia singolare classica utilizza i simplessi singolari come oggetti fondamentali. Per la teoria degli spazi fibrati dovremo introdurre la nozione di omologia singolare cubica, che impiega cubi in luogo dei simplessi. Tuttavia, come vedremo nel paragrafo \ref{cubic-homology:comparison}, dal punto di vista dell'omologia e della coomologia non vi è alcuna differenza.

Consideriamo uno spazio topologico $X$.

\begin{definition}
Sia $n\in\NN$. Si dice cubo singolare (o più semplicemente cubo) di dimensione $n$ un'applicazione continua $\map{u}{I^n}{X}$. Un cubo di dimensione $n\ge 1$ si dice degenere se non dipende dall'ultima coordinata, ossia se $u(x_1,\ldots,x_{n-1},x_n)=u(x_1,\ldots,x_{n-1},x_n')$ per ogni $x_1,\ldots,x_n,x_n'\in I$.
\end{definition}

Denotiamo con $Q_n(X)$ il gruppo abeliano libero avente per base l'insieme dei cubi singolari di dimensione $n$, con $D_n(X)\subs Q_n(X)$ il sottogruppo generato dai cubi degeneri di dimensione $n$. Per definire il complesso $Q_\bullet(X)$ è necessario costruire mappe di bordo $\map{d_n}{Q_n(X)}{Q_{n-1}(X)}$.

Siano $u$ un cubo di dimensione $n$, $p,q\in\NN$ con $p+q=n$, $H$ un sottoinsieme di $\{1,\ldots,n\}$ di cardinalità $p$, $K$ il complementare di $H$, $\varphi_K$ l'unica applicazione strettamente crescente da $K$ in $\{1,\ldots,q\}$; sia inoltre $\epsilon\in\{0,1\}$. Definiamo allora il cubo singolare $\lambda^\epsilon_Hu$ di dimensione $q$:
$$
\lambda^\epsilon_Hu(x_1,\ldots,x_n)=u(y_1,\ldots,y_n)\qquad\text{, dove }y_i=\begin{cases}\epsilon&i\in H\\x_{\varphi_K(i)}&i\in K\end{cases}.
$$
Per snellire la notazione, se $H=\{i\}$ (ossia se $p=1$), scriviamo $\lambda^\epsilon_i$ in luogo di $\lambda^\epsilon_{\{i\}}$. Dato un cubo $u$ di dimensione $n$, definiamo dunque
$$
d_nu=\sum_{i=1}^n\sum_{\epsilon=0}^1(-1)^{i+\epsilon}\lambda^\epsilon_iu,
$$
estendendola per $\ZZ$-linearità a tutto $Q_n(X)$. È immediato verificare che $\lambda^\epsilon_i\lambda^{\epsilon'}_j=\lambda^{\epsilon'}_{j-1}\lambda^\epsilon_i$; un semplice conto mostra allora che $d_nd_{n+1}=0$. Abbiamo così definito il complesso $Q_\bullet(X)$. Si vede inoltre che $D_\bullet(X)$ è un sottocomplesso di $Q_\bullet(X)$: se $u$ è un cubo degenere di dimensione $n$, allora anche $\lambda^\epsilon_iu$ è degenere per $0\le i<n$, mentre $\lambda^0_nu=\lambda^1_nu$, pertanto $du$ è degenere.
\begin{definition}
Si dice complesso singolare (cubico) di $X$ il complesso $C_\bullet(X)=Q_\bullet(X)/D_\bullet(X)$. I suoi gruppi di omologia e coomologia a coefficienti in un gruppo abeliano $G$ si dicono gruppi di omologia e coomologia singolare (cubica) di $X$ a coefficienti in $G$.
\end{definition}
Poiché nel seguito faremo uso esclusivamente dell'omologia singolare cubica, impiegheremo le notazioni classiche dell'omologia singolare: 
\begin{align*}
C_\bullet(X;G)&=C_\bullet(X)\tensor G\\
H_n(X;G)&=H_n(C_\bullet(X;G))\\
C^\bullet(X;G)&=\Hom(C_\bullet(X),G)\\
H^n(X;G)&=H^n(C^\bullet(X;G)).
\end{align*}
Osserviamo che $C^n(X;G)$ può essere interpretato come il gruppo abeliano delle funzioni dai cubi di dimensione $n$ in $G$ che sono nulle sui cubi degeneri. Definiamo inoltre $H_*(X;G)=\Dirsum_{n\ge 0}H_n(X;G)$, $H^*(X;G)=\Dirsum_{n\ge 0}H^n(X;G)$.

\subsection{Prodotto cup}\label{cubic-homology:cup-product}
Esattamente come nel caso della teoria singolare classica, se $G$ è un anello $H^*(X;G)$ acquisisce una struttura di $G$-algebra graduata. Si definisce il \cupproduct{} come segue: se $u$ è un cubo di dimensione $p+q$ e $f,g$ sono cocatene di dimensione $p,q$ rispettivamente, allora
$$
(f\cupp g)u=\sum_H\rho_{H,K}f(\lambda^0_Ku)\cdot g(\lambda^1_Hu),
$$
dove $H$ varia fra i sottoinsiemi di $\{1,\ldots,p+q\}$ di cardinalità $p$, $K$ è il complementare di $H$ e $\rho_{H,K}=(-1)^\nu$ ($\nu$ indica il numero di coppie $(i,j)\in H\times K$ con $j<i$). Il \cupproduct{} è ben definito: poiché $f$ e $g$ sono nulle sui cubi degeneri, si vede anche $f\cupp g$ soddisfa la stessa proprietà (se $u$ è degenere, allora anche uno fra $\lambda^0_Ku$ e $\lambda^1_Hu$ lo è). Si verifica poi che $\cupp$ è associativo, e che
$$
d(f\cupp g)=df\cupp g+(-1)^pf\cupp dg,
$$
da cui segue che il \cupproduct{} passa al quoziente, definendo un prodotto in coomologia $H^*(X;G)\times H^*(X;G)\to H^*(X;G)$.

\subsection{Confronto con l'omologia singolare classica}\label{cubic-homology:comparison}
Come mostrato in \citechapter{eilenberg-maclane1953}{8} l'approccio dell'omologia cubica conduce ai medesimi risultati dell'omologia singolare classica.

\begin{proposition}\thlabel{cubic-homology-isomorphic-classical-homology}
Denotiamo con $H_*^\Delta(X;G)$ e $H^*_\Delta(X;G)$ l'omologia e la coomologia singolare classiche. Allora $H_i(X;G)\iso H_i^\Delta(X;G)$ e $H^i(X;G)\iso H_\Delta^i(X;G)$. Inoltre, se $G$ è anche un anello, allora l'isomorfismo in coomologia preserva la struttura moltiplicativa.
\end{proposition}
\begin{corollary}\thlabel{cohomology-ring-anticommutative}
Supponiamo che $G$ sia un anello. Siano $f,g\in H^*(X;G)$ rispettivamente di grado $p$ e $q$. Allora $f\cdot g=(-1)^{pq}g\cdot f$.
\end{corollary}
Studiando più esplicitamente l'isomorfismo fra l'omologia (e la coomologia) cubica e quella singolare classica si può dimostrare quanto segue.
\begin{proposition}\thlabel{cubic-homology-single-vertex}
Supponiamo che $X$ sia connesso per archi; sia $x\in X$ un punto fissato. Allora i gruppi di omologia e coomologia (cubica)  di $X$ rimangono inalterati se ci si limita a considerare cubi singolari aventi tutti i vertici in $x$.
\end{proposition}

\section{Spazi fibrati}
\begin{definition}
Un'applicazione continua suriettiva $\map{p}{E}{B}$ si dice fibrazione se soddisfa la seguente proprietà (sollevamento dell'omotopia per poliedri finiti): dati un poliedro finito $P$ e due applicazioni continue $\map{f}{P\times I}{B}$, $\map{g}{P}{E}$ tali che $pg=fi$ (dove $i$ denota l'inclusione $\map{i}{P}{P\times I}$ definita da $i(x)=(0,x)$), esiste un'applicazione continua $\map{h}{P\times I}{E}$ tale che $ph=f$ e $hi=g$.
\begin{diagram}
P\rar{g}\dar{i}&E\dar{p}\\
P\times I\rar{f}\ar[ru,dashed,"h"]&B
\end{diagram}
\end{definition}
Se $\map{p}{E}{B}$ è una fibrazione, chiameremo $E$ spazio totale e $B$ spazio base. Inoltre la terna $(p, E, B)$ è detta spazio fibrato.

\begin{proposition}\thlabel{homotopy-lifting-plus}
Siano $\map{p}{E}{B}$ una fibrazione, $A\subs X$ due poliedri finiti contrattili; indichiamo con $\map{i}{A}{X}$ l'inclusione. Siano $\map{f}{X}{B}$, $\map{g}{A}{E}$ applicazioni continue tali che $pg=fi$. Allora esiste un'applicazione continua $\map{h}{X}{E}$ tale che $ph=f$ e $hi=g$.
\begin{diagram}
A\rar{g}\dar{i}&E\dar{p}\\
X\rar{f}\ar[ru,dashed,"h"]&B
\end{diagram}
\end{proposition}
\begin{proof}
Mostriamo innanzitutto un lemma preliminare.
\begin{lemma*}
La proposizione è vera se $X=A\times I^n$ per un qualche $n\ge 0$ e $i(x)=(x,s_1,\ldots,s_n)$ per un qualche $(s_1,\ldots,s_n)\in I^n$.
\end{lemma*}
\begin{proof}
La tesi è ovviamente vera per $n=0$. Procediamo per induzione; sia $n>0$. Consideriamo il diagramma
\begin{diagram}
A\rar{g}\dar{j}&E\dar{p}\\
X'\ar[dr,"i'"]\rar{f'}\ar[ru,"g'"]&B\\
&X\ar[u,"f"]\ar[uu,bend right=60,dashed,"h"]
\end{diagram}
dove $X'=A\times I^{n-1}$, $j(x)=(x,s_1\ldots,s_{n-1})$, $i'(x)=(x,s_n)$ e $f'=fi'$; L'esistenza di $g'$ è garantita dall'ipotesi induttiva, e l'esistenza di $g$ deriva dalla definizione di fibrazione (infatti $X=X'\times I$, e si vede facilmente che non serve richiedere $s_n=0$). Poiché $i=i'j$ e il diagramma commuta, si ottiene immediatamente che $ph=f$ e $hi=g$.
\end{proof}
Sia $Y$ il poliedro $X/A$; denotiamo con $\map{q}{X}{Y}$ la proiezione. Consideriamo un'immersione $\map{j}{Y}{I^n}$ per un qualche $n\ge 0$. Sia $s$ il punto di $I^n$ tale che $jq(A)=\{s\}$. Poiché $A$ è contrattile, esiste una retrazione $\map{r}{X}{A}$. Definiamo l'immersione
\Map{l}{X}{A\times I^n}{x}{(r(x),jq(x))}
Poiché $l(X)$ è un poliedro contrattile, esiste una retrazione $A\times I^n\to l(X)$, che permette di estendere $\map{fl^{-1}}{l(X)}{B}$ a $\map{\tilde{f}}{A\times I^n}{B}$. Abbiamo così il diagramma commutativo
\begin{diagram}
A\rar{g}\dar{i}&E\dar{p}\\
X\rar{f}\ar[dr,"l"]\ar[ru,"h",dashed]&B\\
&A\times I^n\uar{\tilde{f}}\ar[uu,bend right=60,"\tilde{h}"]
\end{diagram}
Poiché $li(x)=(x,s)$ per ogni $x\in A$, l'esistenza di $\tilde{h}$ segue dal Lemma. È ora evidente che $h=\tilde{h}l$ soddisfa la tesi.
\end{proof}
\begin{proposition}\thlabel{fibration-homotopy-exact-sequence}
Siano $\map{p}{E}{B}$ una fibrazione, $e\in E$, $b=p(e)$, $F=p^{-1}(e)$.
\begin{enumerate}
\item\label{fibration-homotopy-exact-sequence:1} La mappa $p$ induce un isomorfismo $\map{p_*}{\pi_i(E,F,e)}{\pi_i(B,b)}$ per ogni $i\ge 1$.
\item\label{fibration-homotopy-exact-sequence:2} Esiste una successione esatta lunga di gruppi
\begin{diagram}
\ldots\rar&\pi_{i+1}(E,e)\rar&\pi_{i+1}(B,b)\rar&\pi_i(F,e)\rar&\pi_i(E,e)\rar&\ldots\rar&\pi_1(E,e)\rar&\pi_1(B,b)
\end{diagram}
\end{enumerate}
\end{proposition}
\begin{proof}
La successione esatta del punto \ref{fibration-homotopy-exact-sequence:2} segue immediatamente dalla successione esatta dei gruppi di omotopia della coppia $(E,F)$, una volta dimostrato il punto \ref{fibration-homotopy-exact-sequence:1}. Mostriamo dunque che $p_*$ è un isomorfismo di gruppi.
\begin{itemize}
\item Per verificare che $p_*$ è iniettiva, sia $\bar{f}\in\pi_i(E,F,e)$ tale che $p_*\bar{f}=0$. Consideriamo un rappresentante $\map{f}{(D^n,S^{n-1},s)}{(E,F,e)}$ della classe $\bar{f}$. Ricordiamo che $\bar{f}=0$ se e solo se $f$ è omotopa a un'applicazione con immagine in $F$ mediante un'omotopia che fissa $S^{n-1}$. Poiché $p_*\bar{f}=0$, esiste un'omotopia $\map{H}{D^n\times I}{B}$ stazionaria su $S^{n-1}$ tale che $H_0=pf$ e $H_1(D^n)\subs\{b\}$. Consideriamo il poliedro finito contrattile 
$$
A=(D^n\times\{0\})\cup(S^{n-1}\times I)\subs D^n\times I.
$$
Definiamo un sollevamento di $H$
\Map{\tilde{H}}{A}{E}{(x,t)}{f(x)}
Per la \thref{homotopy-lifting-plus} $\tilde{H}$ si estende un'omotopia definita su tutto a tutto $D^n\times I$. Osserviamo che $\tilde{H}_0=f$, $\tilde{H}_1(D^n)\subs F$ e $\tilde{H}$ è stazionaria su $S^{n-1}$, dunque $\bar{f}=0$.
\item Per verificare che $p_*$ è suriettiva, sia $\bar{g}\in\pi_i(B,b)$. Sia $\map{g}{(D^n,S^n)}{(B,b)}$ un rappresentante della classe $\bar{g}$. Consideriamo il poliedro finito contrattile $\{s\}\subs D^n$ e l'applicazione $\map{f}{\{s\}}{E}$ tale che $f(s)=e$. Per la \thref{homotopy-lifting-plus} $f$ si estende a tutto $D^n$ in modo che $pf=g$. Si vede immediatamente che la classe di omotopia di $f$ appartiene a $\pi_i(E,F,e)$, dunque $p_*\bar{f}=\bar{g}$.
\end{itemize}
\end{proof}

D'ora in poi considereremo solo fibrazioni con spazio base, spazio totale e fibre connessi per archi. Per la \thref{cubic-homology-single-vertex} possiamo limitarci a considerare cubi con vertici in un singolo punto fissato nello studio dell'omologia e della coomologia. Nel seguito supporremo dunque implicitamente che i cubi in $F$ ed $E$ abbiano tutti i vertici in un punto fissato $e$, e che i cubi in $B$ abbiano tutti i vertici in $b=p(e)$. Utilizzeremo inoltre la notazione $\fibration{F}{E}{B}$ per indicare una fibrazione $\map{p}{E}{B}$ di fibra $F$ (i punti $e$ e $b$ saranno spesso irrilevanti).

\section{Azione del gruppo fondamentale di \texorpdfstring{$B$}{B} sull'omologia della fibra}

Ci proponiamo ora di mostrare come il gruppo fondamentale di $B$ agisca sui gruppi di omologia e coomologia di $F$.

\begin{definition}
Siano $\gamma$ un cammino chiuso in $B$ con estremi in $b$, $T$ un'applicazione che a ogni cubo $u$ di dimensione $n$ di $F$ associa un cubo $Tu$ di $E$ di dimensione $n+1$. $T$ si dice costruzione subordinata a $\gamma$ se soddisfa le seguenti proprietà per ogni cubo $u$ di dimensione $n$:
\begin{enumerate}
\item $\lambda^0_1 Tu=u$;
\item $(p\circ Tu)(t,x_1,\ldots,x_n)=\gamma(t)$;
\item $T\lambda^\epsilon_iu=\lambda^\epsilon_{i+1}Tu$ per $1\le i\le n,\epsilon\in\{0,1\}$;
\item se $u$ è degenere, allora anche $Tu$ lo è.
\end{enumerate}
\end{definition}
Ogni costruzione $T$ induce un morfismo di complessi
\Map{S_T}{C_\bullet(F)}{C_\bullet(F)}{u}{\lambda^1_1Tu}
Le proprietà delle costruzioni garantiscono che $S_Tu$ è effettivamente un cubo di $F$, che cubi degeneri vengono mandati in cubi degeneri e che $S_T$ commuta con la mappa di bordo. A sua volta, $S_T$ induce endomorfismi dei gruppi di omologia e coomologia di $F$.
\begin{proposition}\thlabel{subordinate-construction-homotopy-equivalence}
Siano $\gamma_0,\gamma_1$ cammini chiusi in $B$ con estremi in $b$, $T_0,T_1$ costruzioni subordinate rispettivamente a $\gamma_0,\gamma_1$. Supponiamo che $\gamma_0,\gamma_1$ siano omotopi mediante un'omotopia che fissa gli estremi. Allora i morfismi di complessi $S_{T_0}$ e $S_{T_1}$ sono omotopi.
\end{proposition}
\begin{proof}
Per costruire l'omotopia fra $S_{T_0}$ e $S_{T_1}$ faremo uso del seguente lemma.
\begin{restatable}{lemma}{fibrationHconstruction}\thlabel{fibration-H-construction}
Esiste un'applicazione $H$ che associa a ogni cubo $u\in Q_n(F)$ un cubo $Hu\in Q_{n+2}(E)$ e che soddisfa le seguenti proprietà:
\begin{enumerate}
\item\label{fibration-H-construction:pr1} $\lambda^\epsilon_2Hu=T_\epsilon u$ per $\epsilon\in\{0,1\}$;
\item\label{fibration-H-construction:pr2} $H\lambda^\epsilon_iu=\lambda^\epsilon_{i+2}Hu$ per $1\le i\le n,\epsilon\in\{0,1\}$;
\item\label{fibration-H-construction:pr3} se $u$ è degenere, allora anche $Hu$ lo è.
\end{enumerate}
\end{restatable}
\begin{proof}
Per non allungare eccessivamente questa sezione, rimandiamo la dimostrazione alla fine del capitolo (sezione \ref{fibration:lemma-proof}).
\end{proof}
Definiamo allora l'applicazione
\Map{h}{C_n(F)}{C_{n+1}(F)}{u}{\lambda_1^1Hu}
estesa per $\ZZ$-linearità. La proprietà \ref{fibration-H-construction:pr3} garantisce che $h$ è ben definita (ossia è compatibile con il quoziente che definisce $C_n(F)$). Sfruttiamo le altre due proprietà per mostrare che $h$ è un'omotopia fra $S_{T_0}$ e $S_{T_1}$:
\begin{align*}
dhu+hdu&=\sum_{i=1}^{n+1}\sum_\epsilon(-1)^{i+\epsilon}\lambda^\epsilon_i\lambda^1_1Hu+\sum_{i=1}^n\sum_\epsilon(-1)^{i+\epsilon}\lambda^1_1H\lambda^\epsilon_iu\\
&=\sum_{i=1}^{n+1}\sum_\epsilon(-1)^{i+\epsilon}\lambda^\epsilon_i\lambda^1_1Hu+\sum_{i=1}^n\sum_\epsilon(-1)^{i+\epsilon}\lambda^1_1\lambda^\epsilon_{i+2}Hu\\
&=\sum_{i=1}^{n+1}\sum_\epsilon(-1)^{i+\epsilon}\lambda^\epsilon_i\lambda^1_1Hu+\sum_{i=1}^n\sum_\epsilon(-1)^{i+\epsilon}\lambda^\epsilon_{i+1}\lambda^1_1Hu\\
&=\lambda^1_1\lambda^1_1Hu-\lambda^0_1\lambda^1_1Hu\\
&=\lambda^1_1\lambda^1_2Hu-\lambda^1_1\lambda^0_2Hu\\
&=\lambda^1_1T_1u-\lambda^1_1T_0u\\
&=S_{T_1}u-S_{T_0}u.
\end{align*}
\end{proof}
Si potrebbe dimostrare (si veda \citenumber{serre}{2}{3}) che per ogni cammino $\gamma$ esiste una costruzione subordinata a $\gamma$, e che l'applicazione $\pi_1(B,b)\to\Aut(H_n(F))$ è un omomorfismo di gruppi, ma non utilizzeremo questi risultati.
Ci limitiamo a mostrare il seguente.
\begin{corollary}\thlabel{trivial-path-trivial-action}
Siano $\gamma$ un cammino chiuso in $B$ con estremi in $b$, $T$ una costruzione subordinata a $\gamma$. Supponiamo che $\gamma$ sia omotopicamente banale. Allora $S_T$ induce l'identità in omologia e in coomologia.
\end{corollary}
\begin{proof}
In virtù della \thref{subordinate-construction-homotopy-equivalence} è sufficiente dimostrare che esiste una costruzione $T$ subordinata al cammino che vale costantemente $b$ tale che $S_T$ induce l'identità in omologia e in coomologia. Per ogni cubo $u$ di $F$ di dimensione $n$ definiamo
$$
(Tu)(t,t_1,\ldots,t_n)=u(t_1,\ldots,t_n).
$$
È immediato verificare che $T$ è una costruzione subordinata al cammino costante, e che $S_T$ è l'identità, da cui la tesi.
\end{proof}

I risultati principali che ricaveremo in seguito sugli spazi fibrati richiederanno che l'azione del gruppo fondamentale di $B$ sui gruppi di omologia e coomologia di $F$ sia banale (con questa espressione intendiamo che per ogni costruzione $T$ subordinata a un qualche cammino il morfismo $S_T$ induce l'identità in omologia e in coomologia). Per questo motivo faremo spesso l'ipotesi semplificativa che $B$ sia semplicemente connesso, ipotesi che sarà sempre verificata nelle applicazioni che presenteremo.

\begin{corollary}\thlabel{simply-connected-trivial-action}
Se $B$ è semplicemente connesso, allora $\pi_1(B)$ agisce banalmente sui gruppi di omologia e coomologia di $F$.
\end{corollary}

\section{Dimostrazione del \thref{fibration-H-construction}}
\label{fibration:lemma-proof}
Ricordiamo il contesto: abbiamo due cammini chiusi $\gamma_0,\gamma_1$ in $B$ con estremi in $b$ che sono omotopi mediante un'omotopia che fissa gli estremi; $T_0$ e $T_1$ sono costruzioni subordinate rispettivamente a $\gamma_0$ e $\gamma_1$.
\fibrationHconstruction*
\begin{proof}
Per ipotesi esiste un'omotopia $\map{h}{I\times I}{B}$ tale che $h_0=\gamma_0$, $h_1=\gamma_1$ e $h(0,s)=h(1,s)=b$ per ogni $s\in I$. Costruiamo $H$ per induzione su $n$, richiedendo in più che soddisfi anche le seguenti proprietà:
\begin{enumerate}
\setcounter{enumi}{3}
\item\label{fibration-H-construction:pr4} $(p\circ Hu)(t,s,x_1,\ldots,x_n)=h(t,s)$;
\item\label{fibration-H-construction:pr5} $(Hu)(0,s,x_1,\ldots,x_n)=u(x_1,\ldots,x_n)$.
\end{enumerate}
\paragraph{Costruzione per $n=0$.}
In tal caso $u$ è semplicemente il punto $e$. Consideriamo il diagramma
\begin{diagram}
A\dar{i}\rar{g}&E\dar{p}\\
X\rar{f}\ar[ur,dashed,"v"]&B
\end{diagram}
dove
\begin{itemize}
\item $X=I\times I$;
\item $A=(I\times\{0,1\})\cup(\{0\}\times I)\subs X$;
\item $f=h$;
\item $g$ è così definita
$$
\begin{cases}
g(t,\epsilon)=(T_\epsilon u)(t)&\epsilon\in\{0,1\}\\
g(0,s)=e
\end{cases};
$$
\item l'esistenza di $v$ è garantita dalla \thref{homotopy-lifting-plus}.
\end{itemize}
Poniamo $Hu=v$; tutte le proprietà sono di verifica immediata.
\paragraph{Costruzione per $n>0$ e $u$ degenere.}
Se $u$ è degenere, poniamo
$$
(Hu)(t,s,x_1,\ldots,x_n)=(H\lambda^0_nu)(t,s,x_1,\ldots,x_{n-1}).
$$
Verifichiamo le proprietà.
\begin{enumerate}
\eqitem
\begin{align*}
(\lambda^\epsilon_2Hu)(t,x_1,\ldots,x_n)&=(\lambda^\epsilon_2H\lambda^0_nu)(t,x_1,\ldots,x_{n-1})\\
&=(T_\epsilon\lambda^0_nu)(t,x_1,\ldots,x_{n-1})\\
&=(\lambda^0_{n+1}T_\epsilon u)(t,x_1,\ldots,x_{n-1})\\
&=T_\epsilon u(t,x_1,\ldots,x_{n-1},x_n).
\end{align*}
\item Se $i<n$, allora
\begin{align*}
(\lambda^\epsilon_{i+2}Hu)(t,s,x_1,\ldots,x_{n-1})&=(\lambda^\epsilon_{i+2}H\lambda^0_nu)(t,s,x_1,\ldots,x_{n-2})\\
&=(H\lambda^\epsilon_i\lambda^0_nu)(t,s,x_1,\ldots,x_{n-2})\\
&=(H\lambda^0_{n-1}\lambda^i_\epsilon u)(t,s,x_1,\ldots,x_{n-2})\\
&=(\lambda^0_{n+1}H\lambda^\epsilon_iu)(t,s,x_1,\ldots,x_{n-2})\\
&=(H\lambda^\epsilon_iu)(t,s,x_1,\ldots,x_{n-1}).
\end{align*}
Se invece $i=n$, allora
$$
(\lambda^\epsilon_{n+2}Hu)(t,s,x_1,\ldots,x_{n-1})=(H\lambda^0_nu)(t,s,x_1,\ldots,x_{n-1}).
$$
\item È evidente che $Hu$ è degenere.
\eqitem
$$
(p\circ Hu)(t,s,x_1,\ldots,x_n)=(p\circ H\lambda^0_nu)(t,s,x_1,\ldots,x_{n-1})=h(t,s).
$$
\eqitem
\begin{align*}
(Hu)(0,s,x_1,\ldots,x_{n-1})&=(H\lambda^0_nu)(0,s,x_1,\ldots,x_{n-1})\\
&=(\lambda^0_nu)(x_1,\ldots,x_{n-1})\\
&=u(x_1,\ldots,x_n).
\end{align*}
\end{enumerate}
\paragraph{Costruzione per $n>0$ e $u$ non degenere.} Consideriamo il diagramma
\begin{diagram}
A\dar{i}\rar{g}&E\dar{p}\\
X\rar{f}\ar[ur,dashed,"v"]&B
\end{diagram}
dove
\begin{itemize}
\item $X=I\times I\times I^n$;
\item $A=(I\times\{0,1\}\times I^n)\cup(\{0\}\times I\times I^n)\cup(I\times I\times\del I^n)\subs X$;
\item $f(t,s,x_1,\ldots,x_n)=h(t,s)$;
\item $g$ è così definita
$$
\begin{cases}
g(t,\epsilon,x_1,\ldots,x_n)=(T_\epsilon u)(t,x_1,\ldots,x_n)&\epsilon\in\{0,1\}\\
g(0,s,x_1,\ldots,x_n)=u(x_1,\ldots,x_n)\\
g(t,s,x_1,\ldots,x_{i-1},\epsilon,x_i,\ldots,x_{n-1})=(H\lambda^\epsilon_iu)(t,s,x_1,\ldots,x_{n-1})
\end{cases};
$$
\item l'esistenza di $v$ è garantita dalla \thref{homotopy-lifting-plus}, a patto di dimostrare che $g$ è ben definita sulle intersezioni e che $pg=fi$; si tratta in entrambi i casi di facili verifiche, sfruttando l'ipotesi induttiva e le proprietà delle costruzioni subordinate.
\end{itemize}
Poniamo $Hu=v$, e mostriamo che le proprietà richieste sono soddisfatte.
\begin{enumerate}
\eqitem
\begin{align*}
\lambda^\epsilon_2Hu(t,x_1,\ldots,x_n)&=(Hu)(t,\epsilon,x_1,\ldots,x_n)\\
&=g(t,\epsilon,x_1,\ldots,x_n)\\
&=(T_\epsilon u)(t,\epsilon,x_1,\ldots,x_n).
\end{align*}
\eqitem
\begin{align*}
(\lambda^\epsilon_{i+2}Hu)(t,s,x_1,\ldots,x_{n-1})&=(Hu)(t,s,x_1,\ldots,x_{i-1},\epsilon,x_i,x_{n-1})\\
&=g(t,s,x_1,\ldots,x_{i-1},\epsilon,x_i,x_{n-1})\\
&=(H\lambda^\epsilon_iu)(t,s,x_1,\ldots,x_{n-1}).
\end{align*}
\item Per ipotesi $u$ non è degenere.
\eqitem
$$
(p\circ Hu)(t,s,x_1,\ldots,x_n)=f(t,s,x_1,\ldots,x_n)=h(t,s).
$$
\eqitem
$$
(Hu)(0,s,x_1,\ldots,x_n)=g(0,s,x_1,\ldots,x_n)=u(0,s,x_1,\ldots,x_n).
$$
\end{enumerate}
\end{proof}