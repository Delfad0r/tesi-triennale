\chapter{Omologia e coomologia degli spazi fibrati}
\label{chapter-fibration}

\section{Omologia singolare cubica}

L'omologia singolare classica utilizza i simplessi singolari come oggetti fondamentali. Per la teoria degli spazi fibrati dovremo introdurre la nozione di omologia singolare cubica, che impiega cubi in luogo dei simplessi. Come è lecito aspettarsi, i cubi si prestano meglio allo studio degli spazi prodotto, e anche a quello degli spazi fibrati che, come vedremo, ne sono una generalizzazione.

Nel seguito indicheremo con $I$ l'intervallo $[0,1]$ con l'usuale topologia euclidea. Sia inoltre $X$ uno spazio topologico.

\begin{definition}
Sia $n\in\NN$. Si dice cubo singolare (o più semplicemente cubo) di dimensione $n$ un'applicazione continua $\map{u}{I^n}{X}$. Un cubo di dimensione $n\ge 1$ si dice degenere se non dipende dall'ultima coordinata, ossia se $u(x_1,\ldots,x_{n-1},x_n)=u(x_1,\ldots,x_{n-1},x_n')$ per ogni $x_1,\ldots,x_n,x_n'\in I$.
\end{definition}

Denotiamo con $Q_n(X)$ il gruppo abeliano libero avente per base l'insieme dei cubi singolari di dimensione $n$, con $D_n$ il gruppo abeliano libero avente per base l'insieme dei cubi degeneri di dimensione $n$. Per definire il complesso $Q_\bullet(X)$ è necessario costruire mappe di bordo $\map{d_n}{Q_n(X)}{Q_{n-1}(X)}$.

Sia $u$ un cubo di dimensione $n$, $p,q\in\NN$ con $p+q=n$, $H$ un sottoinsieme di $\{1,\ldots,n\}$ di cardinalità $p$, $K$ il complementare di $H$, $\varphi_K$ l'unica applicazione strettamente crescente da $K$ in $\{1,\ldots,q\}$; sia inoltre $\epsilon\in\{0,1\}$. Definiamo allora il cubo singolare $\lambda^\epsilon_Hu$ di dimensione $q$:
$$
\lambda^\epsilon_Hu(x_1,\ldots,x_n)=u(y_1,\ldots,y_n)\qquad\text{, dove }y_i=\begin{cases}\epsilon&\text{se $i\in H$}\\x_{\varphi_K(i)}&\text{se $i\in K$}\end{cases}.
$$
Per snellire la notazione, se $H=\{i\}$ (ossia se $p=1$), scriviamo $\lambda^\epsilon_i$ in luogo di $\lambda^\epsilon_{\{i\}}$. Dato un cubo $u$ di dimensione $n$, definiamo dunque
$$
d_nu=\sum_{i=0}^n(\lambda^0_iu-\lambda^1_iu),
$$
estendendola per $\ZZ$-linearità a tutto $Q_n(X)$. È immediato verificare che $\lambda^\epsilon_i\lambda^{\epsilon'}_j=\lambda^{\epsilon'}_{j-1}\lambda^\epsilon_i$; un semplice conto mostra allora che $d_nd_{n+1}=0$. Abbiamo così definito il complesso $Q_\bullet(X)$. Si vede inoltre che $D_\bullet(X)$ è un sottocomplesso di $Q_\bullet(X)$: se $u$ è un cubo degenere di dimensione $n$, allora anche $\lambda^\epsilon_iu$ è degenere per $0\le i<n$, mentre $\lambda^0_nu=\lambda^1_nu$, pertanto $du$ è degenere.
\begin{definition}
Si dice complesso singolare (cubico) di $X$ il complesso $C_\bullet(X)=Q_\bullet(X)/D_\bullet(X)$. I suoi gruppi di omologia e coomologia a coefficienti in un gruppo abeliano $G$ si dicono gruppi di omologia e coomologia singolare (cubica) di $X$ a coefficienti in $G$.
\end{definition}
Poiché nel seguito faremo uso esclusivamente dell'omologia singolare cubica, impiegheremo le notazioni classiche dell'omologia singolare: 
\begin{align*}
C_\bullet(X;G)&=C_\bullet(X)\tensor G\\
H_n(X;G)&=H_n(C_\bullet(X;G))\\
C^\bullet(X;G)&=\Hom(C_\bullet(X),G)\\
H^n(X;G)&=H^n(C^\bullet(X;G)).
\end{align*}
Osserviamo che $C^n(X;G)$ può essere interpretato come il gruppo delle funzioni $\map{f}{C_n(X)}{G}$ nulle sui cubi degeneri.

Siano inoltre $H(X;G)=\Dirsum_{n\ge 0}H_n(X;G)$, $H^*(X;G)=\Dirsum_{n\ge 0}H^n(X;G)$. Esattamente come nel caso della teoria singolare classica, se $G$ è un anello, $H^*(X;G)$ acquisisce una struttura di anello graduato. Si definisce il \cupproduct{} come segue: se $u$ è un cubo di dimensione $p+q$ e $f,g$ sono cocatene di dimensione $p,q$ rispettivamente, allora
$$
(f\cupp g)u=\sum_H\rho_{H,K}f(\lambda^0_Ku)\cdot g(\lambda^1_Hu),
$$
dove $H$ varia fra i sottoinsiemi di $\{1,\ldots,p+q\}$ di cardinalità $p$, $K$ è il complementare di $H$ e $\rho_{H,K}=(-1)^\nu$ ($\nu$ indica il numero di coppie $(i,j)\in H\times K$ con $j<i$). Il \cupproduct{} è ben definito: poiché $f$ e $g$ sono nulle sui cubi degeneri, si vede anche $f\cupp g$ soddisfa la stessa proprietà (se $u$ è degenere, allora anche uno fra $\lambda^0_Ku$ e $\lambda^1_Hu$ lo è). Si verifica poi che $\cupp$ è associativo, e che
$$
d(f\cupp g)=df\cupp g+(-1)^pf\cupp dg,
$$
da cui segue che il \cupproduct{} passa al quoziente, definendo un prodotto in coomologia $\map{\cupp}{H^*(X;G)\times H^*(X;G)}{H^*(X;G)}$.

Si può dimostrare che l'approccio dell'omologia cubica conduce ai medesimi risultati dell'omologia singolare classica.

\begin{proposition}\thlabel{cubic-homology-isomorphic-classical-homology}
Denotiamo con $H_\Delta(X;G),H^*_\Delta(X;G)$ l'omologia e la coomologia singolare standard. Allora $H(X;G)\iso H_\Delta(X;G)$ come gruppi graduati, e $H^*(X;G)\iso H_\Delta^*(X;G)$ come anelli graduati.
\end{proposition}
\begin{corollary}\thlabel{cohomology-ring-anticommutative}
Siano $f,g\in H^*(X;G)$ rispettivamente di grado $p$ e $q$. Allora $f\cupp g=(-1)^{pq}g\cupp f$.
\end{corollary}
Studiando più esplicitamente l'isomorfismo fra l'omologia (e la coomologia) cubica e quella singolare classica si può dimostrare quanto segue.
\begin{proposition}\thlabel{cubic-homology-single-vertex}
Supponiamo che $X$ sia connesso per archi; sia $x\in X$ un punto fissato. Allora i gruppi di omologia e coomologia (cubica)  di $X$ rimangono inalterati se ci si limita a considerare cubi singolari aventi tutti i vertici in $x$.
\end{proposition}

\section{Spazi fibrati}
\begin{definition}
Un'applicazione continua suriettiva $\map{p}{E}{B}$ si dice fibrazione se soddisfa la seguente proprietà (sollevamento dell'omotopia per poliedri finiti): dati un poliedro finito $P$ e due applicazioni continue $\map{f}{I\times P}{B},\map{g}{P}{E}$ tali che $pg=fi$ (dove $i$ denota l'inclusione $\map{i}{P}{I\times P}$ definita da $i(x)=(0,x)$), esiste un'applicazione continua $\map{h}{I\times P}{E}$ tale che $ph=f$ e $hi=g$.
\begin{diagram}
P\rar{g}\dar{i}&E\dar{p}\\
I\times P\rar{f}\ar[ru,dashed,"h"]&B
\end{diagram}
\end{definition}
Se $\map{p}{E}{B}$ è una fibrazione, chiameremo $E$ spazio totale e $B$ spazio base.
In realtà il sollevamento dell'omotopia per poliedri finiti implica una proprietà più forte.
\begin{proposition}\thlabel{homotopy-lifting-plus}
Sia $\map{p}{E}{B}$ una fibrazione, $A\subs X$ due poliedri finiti; indichiamo con $\map{i}{A}{X}$ l'inclusione. Siano $\map{f}{X}{B},\map{g}{A}{E}$ applicazioni continue tali che $pg=fi$. Allora esiste un'applicazione continua $\map{h}{X}{E}$ tale che $ph=f$ e $hi=g$.
\begin{diagram}
A\rar{g}\dar{i}&E\dar{p}\\
X\rar{f}\ar[ru,dashed,"h"]&B
\end{diagram}
\end{proposition}
\begin{proof}
\begin{lemma}
La proposizione è vera se $X=A\times I^n$ per un qualche $n\ge 0$ e $i(x)=(x,0)$.
\end{lemma}
\begin{proof}

\end{proof}
\end{proof}
\begin{proposition}\thlabel{fibration-homotopy-exact-sequence}
Sia $\map{p}{E}{B}$ una fibrazione, $e\in E,b=p(e),F=p^{-1}(e)$.
\begin{enumerate}
\item La mappa $p$ induce un isomorfismo $\map{p_*}{\pi_i(E,F,e)}{\pi_i(B,b)}$ per ogni $i\ge 1$.
\item Esiste una successione esatta lunga di gruppi
\begin{diagram}
\ldots\rar&\pi_{i+1}(E,e)\rar&\pi_{i+1}(B,b)\rar&\pi_i(F,e)\rar&\pi_i(E,e)\rar&\ldots\rar&\pi_1(E,e)\rar&\pi_1(B,b)
\end{diagram}
\end{enumerate}
\end{proposition}
\begin{proof}

\end{proof}

\section{Un'ipotesi aggiuntiva}
Sia $\map{p}{E}{B}$ una fibrazione, $e\in E,b=p(e),F=p^{-1}(e)$.
\begin{proposition}\thlabel{fibration-path-connected-base-and-fiber}
Supponiamo che $B$ e $F$ siano connesse per archi. Allora anche $E$ e tutte le altre fibre sono connesse per archi.
\end{proposition}
\begin{proof}

\end{proof}
D'ora in poi considereremo solo fibrazioni con spazio base e fibre connessi per archi. Per la \thref{cubic-homology-single-vertex} possiamo limitarci a considerare cubi con vertici in un singolo punto fissato nello studio dell'omologia e della coomologia. Nel seguito supporremo dunque implicitamente che i cubi in $F$ ed $E$ abbiano tutti i vertici in un punto fissato $e$, e che i cubi in $B$ abbiano tutti i vertici in $b=p(e)$.

Ci proponiamo ora di mostrare come il gruppo fondamentale di $B$ agisca sui gruppi di omologia e coomologia di $F$.

\begin{definition}
Sia $\gamma$ un cammino chiuso in $B$ con estremi in $b$, $C$ un'applicazione che a ogni cubo $u$ di dimensione $n$ di $F$ ne associa uno $Cu$ di dimensione $n+1$. $C$ si dice costruzione subordinata a $\gamma$ se soddisfa le seguenti proprietà per ogni cubo $u$ di dimensione $n$:
\begin{enumerate}
\item $\lambda^0_1 Cu=u$;
\item $(p\circ Cu)(t,t_1,\ldots,t_n)=\gamma(t)$ per ogni $t_1,\ldots,t_n\in I$;
\item $C\lambda^\epsilon_iu=\lambda^\epsilon_{i+1}Cu$ per $0\le i\le n,\epsilon\in\{0,1\}$;
\item se $u$ è degenere, allora anche $Cu$ lo è.
\end{enumerate}
\end{definition}