\chapter{Omologia e coomologia degli spazi fibrati}
\label{chapter-fibration}

\section{Omologia singolare cubica}

L'omologia singolare classica utilizza i simplessi singolari come oggetti fondamentali. Per la teoria degli spazi fibrati dovremo introdurre la nozione di omologia singolare cubica, che impiega cubi in luogo dei simplessi. Come è lecito aspettarsi, i cubi si prestano meglio allo studio degli spazi prodotto, e anche a quello degli spazi fibrati, che ne sono, in un certo senso, una generalizzazione.

Nel seguito indicheremo con $I$ l'intervallo $[0,1]$ con l'usuale topologia euclidea. Sia inoltre $X$ uno spazio topologico.

\begin{definition}
Sia $n\in\NN$. Si dice cubo singolare (o più semplicemente cubo) di dimensione $n$ un'applicazione continua $\map{u}{I^n}{X}$. Un cubo di dimensione $n\ge 1$ si dice degenere se non dipende dall'ultima coordinata, ossia se $u(x_1,\ldots,x_{n-1},x_n)=u(x_1,\ldots,x_{n-1},x_n')$ per ogni $x_1,\ldots,x_n,x_n'\in I$.
\end{definition}

Denotiamo con $Q_n(X)$ il gruppo abeliano libero avente per base l'insieme dei cubi singolari di dimensione $n$, con $D_n(X)$ il gruppo abeliano libero avente per base l'insieme dei cubi degeneri di dimensione $n$. Per definire il complesso $Q_\bullet(X)$ è necessario costruire mappe di bordo $\map{d_n}{Q_n(X)}{Q_{n-1}(X)}$.

Sia $u$ un cubo di dimensione $n$, $p,q\in\NN$ con $p+q=n$, $H$ un sottoinsieme di $\{1,\ldots,n\}$ di cardinalità $p$, $K$ il complementare di $H$, $\varphi_K$ l'unica applicazione strettamente crescente da $K$ in $\{1,\ldots,q\}$; sia inoltre $\epsilon\in\{0,1\}$. Definiamo allora il cubo singolare $\lambda^\epsilon_Hu$ di dimensione $q$:
$$
\lambda^\epsilon_Hu(x_1,\ldots,x_n)=u(y_1,\ldots,y_n)\qquad\text{, dove }y_i=\begin{cases}\epsilon&\text{se $i\in H$}\\x_{\varphi_K(i)}&\text{se $i\in K$}\end{cases}.
$$
Per snellire la notazione, se $H=\{i\}$ (ossia se $p=1$), scriviamo $\lambda^\epsilon_i$ in luogo di $\lambda^\epsilon_{\{i\}}$. Dato un cubo $u$ di dimensione $n$, definiamo dunque
$$
d_nu=\sum_{i=1}^n(\lambda^0_iu-\lambda^1_iu),
$$
estendendola per $\ZZ$-linearità a tutto $Q_n(X)$. È immediato verificare che $\lambda^\epsilon_i\lambda^{\epsilon'}_j=\lambda^{\epsilon'}_{j-1}\lambda^\epsilon_i$; un semplice conto mostra allora che $d_nd_{n+1}=0$. Abbiamo così definito il complesso $Q_\bullet(X)$. Si vede inoltre che $D_\bullet(X)$ è un sottocomplesso di $Q_\bullet(X)$: se $u$ è un cubo degenere di dimensione $n$, allora anche $\lambda^\epsilon_iu$ è degenere per $0\le i<n$, mentre $\lambda^0_nu=\lambda^1_nu$, pertanto $du$ è degenere.
\begin{definition}
Si dice complesso singolare (cubico) di $X$ il complesso $C_\bullet(X)=Q_\bullet(X)/D_\bullet(X)$. I suoi gruppi di omologia e coomologia a coefficienti in un gruppo abeliano $G$ si dicono gruppi di omologia e coomologia singolare (cubica) di $X$ a coefficienti in $G$.
\end{definition}
Poiché nel seguito faremo uso esclusivamente dell'omologia singolare cubica, impiegheremo le notazioni classiche dell'omologia singolare: 
\begin{align*}
C_\bullet(X;G)&=C_\bullet(X)\tensor G\\
H_n(X;G)&=H_n(C_\bullet(X;G))\\
C^\bullet(X;G)&=\Hom(C_\bullet(X),G)\\
H^n(X;G)&=H^n(C^\bullet(X;G)).
\end{align*}
Osserviamo che $C^n(X;G)$ può essere interpretato come il gruppo delle funzioni dai cubi di dimensione $n$ in $G$ che sono nulle sui cubi degeneri.

Siano inoltre $H(X;G)=\Dirsum_{n\ge 0}H_n(X;G)$, $H^*(X;G)=\Dirsum_{n\ge 0}H^n(X;G)$. Esattamente come nel caso della teoria singolare classica, se $G$ è un anello, $H^*(X;G)$ acquisisce una struttura di anello graduato. Si definisce il \cupproduct{} come segue: se $u$ è un cubo di dimensione $p+q$ e $f,g$ sono cocatene di dimensione $p,q$ rispettivamente, allora
$$
(f\cupp g)u=\sum_H\rho_{H,K}f(\lambda^0_Ku)\cdot g(\lambda^1_Hu),
$$
dove $H$ varia fra i sottoinsiemi di $\{1,\ldots,p+q\}$ di cardinalità $p$, $K$ è il complementare di $H$ e $\rho_{H,K}=(-1)^\nu$ ($\nu$ indica il numero di coppie $(i,j)\in H\times K$ con $j<i$). Il \cupproduct{} è ben definito: poiché $f$ e $g$ sono nulle sui cubi degeneri, si vede anche $f\cupp g$ soddisfa la stessa proprietà (se $u$ è degenere, allora anche uno fra $\lambda^0_Ku$ e $\lambda^1_Hu$ lo è). Si verifica poi che $\cupp$ è associativo, e che
$$
d(f\cupp g)=df\cupp g+(-1)^pf\cupp dg,
$$
da cui segue che il \cupproduct{} passa al quoziente, definendo un prodotto in coomologia $H^*(X;G)\times H^*(X;G)\to H^*(X;G)$.

Si può dimostrare che l'approccio dell'omologia cubica conduce ai medesimi risultati dell'omologia singolare classica.

\begin{proposition}\thlabel{cubic-homology-isomorphic-classical-homology}
Denotiamo con $H_\Delta(X;G),H^*_\Delta(X;G)$ l'omologia e la coomologia singolare standard. Allora $H(X;G)\iso H_\Delta(X;G)$ come gruppi graduati, e $H^*(X;G)\iso H_\Delta^*(X;G)$ come anelli graduati.
\end{proposition}
\begin{corollary}\thlabel{cohomology-ring-anticommutative}
Siano $f,g\in H^*(X;G)$ rispettivamente di grado $p$ e $q$. Allora $f\cdot g=(-1)^{pq}g\cdot f$.
\end{corollary}
Studiando più esplicitamente l'isomorfismo fra l'omologia (e la coomologia) cubica e quella singolare classica si può dimostrare quanto segue.
\begin{proposition}\thlabel{cubic-homology-single-vertex}
Supponiamo che $X$ sia connesso per archi; sia $x\in X$ un punto fissato. Allora i gruppi di omologia e coomologia (cubica)  di $X$ rimangono inalterati se ci si limita a considerare cubi singolari aventi tutti i vertici in $x$.
\end{proposition}

\section{Spazi fibrati}
\begin{definition}
Un'applicazione continua suriettiva $\map{p}{E}{B}$ si dice fibrazione se soddisfa la seguente proprietà (sollevamento dell'omotopia per poliedri finiti): dati un poliedro finito $P$ e due applicazioni continue $\map{f}{I\times P}{B},\map{g}{P}{E}$ tali che $pg=fi$ (dove $i$ denota l'inclusione $\map{i}{P}{I\times P}$ definita da $i(x)=(0,x)$), esiste un'applicazione continua $\map{h}{I\times P}{E}$ tale che $ph=f$ e $hi=g$.
\begin{diagram}
P\rar{g}\dar{i}&E\dar{p}\\
I\times P\rar{f}\ar[ru,dashed,"h"]&B
\end{diagram}
\end{definition}
Se $\map{p}{E}{B}$ è una fibrazione, chiameremo $E$ spazio totale e $B$ spazio base.
In realtà il sollevamento dell'omotopia per poliedri finiti implica una proprietà più forte.
\begin{proposition}\thlabel{homotopy-lifting-plus}
Sia $\map{p}{E}{B}$ una fibrazione, $A\subs X$ due poliedri finiti; indichiamo con $\map{i}{A}{X}$ l'inclusione. Siano $\map{f}{X}{B},\map{g}{A}{E}$ applicazioni continue tali che $pg=fi$. Allora esiste un'applicazione continua $\map{h}{X}{E}$ tale che $ph=f$ e $hi=g$.
\begin{diagram}
A\rar{g}\dar{i}&E\dar{p}\\
X\rar{f}\ar[ru,dashed,"h"]&B
\end{diagram}
\end{proposition}
\begin{proof}
\begin{lemma}
La proposizione è vera se $X=A\times I^n$ per un qualche $n\ge 0$ e $i(x)=(x,0)$.
\end{lemma}
\begin{proof}

\end{proof}
\end{proof}
\begin{proposition}\thlabel{fibration-homotopy-exact-sequence}
Sia $\map{p}{E}{B}$ una fibrazione, $e\in E,b=p(e),F=p^{-1}(e)$.
\begin{enumerate}
\item La mappa $p$ induce un isomorfismo $\map{p_*}{\pi_i(E,F,e)}{\pi_i(B,b)}$ per ogni $i\ge 1$.
\item Esiste una successione esatta lunga di gruppi
\begin{diagram}
\ldots\rar&\pi_{i+1}(E,e)\rar&\pi_{i+1}(B,b)\rar&\pi_i(F,e)\rar&\pi_i(E,e)\rar&\ldots\rar&\pi_1(E,e)\rar&\pi_1(B,b)
\end{diagram}
\end{enumerate}
\end{proposition}
\begin{proof}

\end{proof}

\begin{proposition}\thlabel{fibration-path-connected-base-and-fiber}
Sia $\map{p}{E}{B}$ una fibrazione, $e\in E,b=p(e),F=p^{-1}(e)$. Supponiamo che $B$ e $F$ siano connessi per archi. Allora anche $E$ e tutte le altre fibre sono connessi per archi.
\end{proposition}
\begin{proof}

\end{proof}
D'ora in poi considereremo solo fibrazioni con spazio base e fibre connessi per archi. Per la \thref{cubic-homology-single-vertex} possiamo limitarci a considerare cubi con vertici in un singolo punto fissato nello studio dell'omologia e della coomologia. Nel seguito supporremo dunque implicitamente che i cubi in $F$ ed $E$ abbiano tutti i vertici in un punto fissato $e$, e che i cubi in $B$ abbiano tutti i vertici in $b=p(e)$.

\section{Azione del gruppo fondamentale di \texorpdfstring{$B$}{B} sull'omologia della fibra}

Ci proponiamo ora di mostrare come il gruppo fondamentale di $B$ agisca sui gruppi di omologia e coomologia di $F$.

\begin{definition}
Sia $\gamma$ un cammino chiuso in $B$ con estremi in $b$, $T$ un'applicazione che a ogni cubo $u$ di dimensione $n$ di $F$ ne associa uno $Tu$ di dimensione $n+1$. $T$ si dice costruzione subordinata a $\gamma$ se soddisfa le seguenti proprietà per ogni cubo $u$ di dimensione $n$:
\begin{enumerate}
\item $\lambda^0_1 Tu=u$;
\item $(p\circ Tu)(t,t_1,\ldots,t_n)=\gamma(t)$ per ogni $t_1,\ldots,t_n\in I$;
\item $T\lambda^\epsilon_iu=\lambda^\epsilon_{i+1}Tu$ per $0\le i\le n,\epsilon\in\{0,1\}$;
\item se $u$ è degenere, allora anche $Tu$ lo è.
\end{enumerate}
\end{definition}
Ogni costruzione $T$ induce un morfismo di complessi $\map{S_T}{C_\bullet(F)}{C_\bullet(F)}$ definito da $(S_Tu)(t)=(Tu)(1,t)$. Le proprietà delle costruzioni garantiscono che $S_Tu$ è effettivamente un cubo di $F$, che cubi degeneri vengono mandati in cubi degeneri e che $S_T$ commuta con la mappa di bordo. A sua volta, $S_T$ induce endomorfismi dei gruppi di omologia e coomologia di $F$.
\begin{proposition}\thlabel{subordinate-construction-homotopy-equivalence}
Siano $\gamma_1,\gamma_2$ cammini chiusi in $B$ con estremi in $b$, $T_1,T_2$ costruzioni subordinate rispettivamente a $\gamma_1,\gamma_2$. Supponiamo che $\gamma_1,\gamma_2$ rappresentino lo stesso elemento del gruppo fondamentale. Allora i morfismi di complessi $S_{T_1}$ e $S_{T_2}$ sono omotopi.
\end{proposition}
\begin{proof}

\end{proof}
Si potrebbe dimostrare che per ogni cammino $\gamma$ esiste una costruzione subordinata a $\gamma$, e che l'applicazione $\pi_1(B,b)\to\Aut(H_n(F))$ è un omomorfismo di gruppi, ma non utilizzeremo questi risultati. \todo{Serre lo fa però.} 
Ci limitiamo a dimostrare quanto segue.
\begin{proposition}\thlabel{trivial-path-trivial-action}
Sia $\gamma$ un cammino chiuso in $B$ con estremi in $b$, $T$ una costruzione subordinata a $\gamma$. Supponiamo che $\gamma$ sia omotopicamente banale. Allora $S_T$ induce l'identità in omologia e in coomologia.
\end{proposition}
\begin{proof}

\end{proof}
Motivati dalla proposizione precedente, ci limiteremo spesso a studiare fibrazioni in cui l'azione di $\pi_1(B)$ sui gruppi di omologia e coomologia di $F$ è banale (con questa espressione intendiamo che per ogni costruzione $T$ subordinata a un qualche cammino il morfismo $S_T$ induce l'identità in omologia e in coomologia).
\begin{corollary}\thlabel{simply-connected-trivial-action}
Se $B$ è semplicemente connesso, allora $\pi_1(B)$ agisce banalmente sui gruppi di omologia e coomologia di $F$.
\end{corollary}

\section{Successione spettrale di uno spazio fibrato}
Per applicare i risultati della Sezione \ref{sec:spectral-sequence-filtered-complex}, è necessario definire una filtrazione crescente sul complesso singolare $C_\bullet(E)$ (d'ora in poi ometteremo la $E$ dell'argomento). Ciò che faremo sarà filtrare il complesso $Q_\bullet$ con dei sottocomplessi $Q^p_\bullet$ e prenderne le immagini nel quoziente $C_\bullet$. Sia dunque $Q^p_n$ il sottogruppo di $Q_n$ generato dai cubi $u\in Q_n$ tali che $p\circ u$ dipende solo dalle prime $p$ coordinate (e $Q^p_n=Q_n$ se $p>n$). Si vede immediatamente che i $Q^p_\bullet$ sono sottocomplessi di $Q_\bullet$ e che soddisfano le proprietà di una filtrazione crescente. Dunque lo stesso vale per $C^p_\bullet=(Q^p_\bullet+D_\bullet)/D_\bullet$, che definiscono una filtrazione crescente per $C_\bullet$. Applicando la \thref{spectral-sequence-of-filtered-complex} otteniamo una successione spettrale $E^{p,q}_r\converges H_{p+q}(E)$. Come vedremo, è possibile calcolare esplicitamente i termini $E^{p,q}_2$ della successione spettrale in funzione dei gruppi di omologia di $B$ e di $F$.

Costruiamo due applicazioni $B^p$ e $F^p$ definite sui cubi di $Q^p_\bullet$ (ed estese per $\ZZ$-linearità a tutto $Q^p_\bullet$). Se $u\in\ Q^p_n$ è un cubo di dimensione $n$ con $n\ge p$, posto $q=n-p$, definiamo
\begin{align*}
(B^pu)(t_1,\ldots,t_p)=pu(t_1,\ldots,t_p,0,\ldots,0);\\
(F^pu)(t_1,\ldots,t_q)=u(0,\ldots,0,t_1,\ldots,t_q).
\end{align*}
$B^pu$ è un cubo di $B$ di dimensione $p$; notiamo che, poiché $u\in Q^p_n$, possiamo sostituire gli zeri nella definizione con qualunque altra $q$-upla di reali fra $0$ e $1$. $F^pu$ è invece un cubo di $F$ di dimensione $q$; la sua immagine è contenuta in $F$ poiché
$$
p(F^pu)(t_1,\ldots,t_q)=pu(0,\ldots,0,t_1,\ldots,t_q)=pu(0,\ldots,0)=b.
$$
Le seguenti proprietà sono di verifica immediata.
\begin{enumerate}
\item\label{spectral-sequence-of-fibration:pr1} Se $u\in Q^{p-1}_n$ allora $B^pu$ è degenere.
\item\label{spectral-sequence-of-fibration:pr2} Se $u$ è degenere e $q>0$ allora $F^pu$ è degenere; se $u$ è degenere e $q=0$ allora $B^pu$ è degenere.
\item\label{spectral-sequence-of-fibration:pr3} Se $i>p,\epsilon\in\{0,1\}$ allora $B^p\lambda^\epsilon_iu=B^pu$ e $F^p\lambda^\epsilon_iu=\lambda^\epsilon_{i-p}F^pu$.
\end{enumerate}

Ricordiamo che $E^{p,q}_0=C^p_{p+q}/C^{p-1}_{p+q}$ e che la mappa $\map{d^{p,q}_0}{E^{p,q}_0}{E^{p,q-1}_0}$ si ottiene dalla mappa di bordo di $C_\bullet$ per passaggio al quoziente. Possiamo dunque raggruppare i termini della successione spettrale aventi $r=0$ in complessi $E^{p,\bullet}_0$, con mappe di bordo $d^p_0$. Osserviamo che
$$
d^p_0u=\sum_{i>p}(-1)^i(\lambda^0_iu-\lambda^1_iu)
$$
in quanto per $i\le p$ vale $\lambda^\epsilon_iu\in Q^p_\bullet$.

Consideriamo il complesso $J^p_\bullet=C_p(B)\tensor C_\bullet(F)$ con mappe di bordo $d_J(b\tensor f)=(-1)^pb\tensor df$, e sia $\varphi^p$ il morfismo di complessi
\Map{\varphi^p}{E^{p,\bullet}_0}{J^p_\bullet}{u}{B^pu\tensor F^pu}
Le proprietà \ref{spectral-sequence-of-fibration:pr1} e \ref{spectral-sequence-of-fibration:pr2} garantiscono che si tratta di una buona definizione (ossia che è compatibile con il quoziente che definisce $E^{p,\bullet}_0$), mentre dalla \ref{spectral-sequence-of-fibration:pr3} segue che $\varphi^p$ è effettivamente un morfismo di complessi:
\begin{align*}
\varphi^pd^p_0u&=\varphi^p\sum_{i>p}(-1)^i(\lambda^0_iu-\lambda^1_iu)\\
&=\sum_{i>p}(-1)^i(B^p\lambda^0_iu\tensor F^p\lambda^0_iu-B^p\lambda^1_iu\tensor F^p\lambda^1_iu)\\
&=\sum_{i>p}(-1)^pB^pu\tensor(-1)^{p-i}(\lambda^0_{i-p}F^pu-\lambda^1_{i-p}F^pu)\\
&=(-1)^pB^pu\tensor dF^pu\\
&=d_J\varphi^pu.
\end{align*}

Ci proponiamo di mostrare che $\varphi^p$ è in realtà un'equivalenza omotopica. Sfrutteremo il seguente lemma per costruire un'inversa omotopica.

\begin{lemma}\thlabel{spectral-sequence-of-fibration-psi-construction}
Esiste un'applicazione $K$ che associa a ogni coppia di cubi $(u,v)$ con $u\in Q_p(B)$ e $v\in Q_q(F)$ un cubo $K(u,v)\in Q^p_{p+q}(E)$ e che soddisfa le seguenti proprietà:
\begin{enumerate}
\item $B^pK(u,v)=u$ e $F^pK(u,v)=v$;
\item per ogni $i\le q,\epsilon\in\{0,1\}$ vale $K(u,\lambda^\epsilon_iv)=\lambda^\epsilon_{i+p}K(u,v)$;
\item se $v$ è degenere allora $K(u,v)$ è degenere.
\end{enumerate}
\end{lemma}
\begin{proof}

\end{proof}