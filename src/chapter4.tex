\chapter{Successione spettrale coomologica di uno spazio fibrato}
\label{ch:cohomological-spectral-sequence-of-fibration}
\section{Costruzione della successione spettrale}
Come è lecito aspettarsi, dato uno spazio fibrato è possibile associarvi anche una successione spettrale coomologica. Sia \(G\) un gruppo abeliano. Partendo dal complesso filtrato \(C_\bullet\) procediamo come mostrato nel paragrafo \ref{cohomological-spectral-sequence:dual-complex}, costruendo il cocomplesso \(C^\bullet=\Hom(C_\bullet, G)\) e filtrandolo mediante i sottocomplessi \(C^\bullet_p=\Ann C^{p-1}_\bullet\). Otteniamo così una successione spettrale coomologica \(E^{\ast p,q}_r \converges H^{p+q}(E;G)\).

Osserviamo che
\[
E^{\ast p,q}_0=\frac{\Ann C^{p-1}_n}{\Ann C^p_n}=\frac{\Hom(C_n/C^{p-1}_n,G)}{\Hom(C_n/C^p_n,G)},
\]
ma, poiché la successione esatta
\begin{diagram}
0\rar&C^p_n/C^{p-1}_n\rar&C_n/C^{p-1}_n\rar&C_n/C^p_n\rar&0
\end{diagram}
spezza, otteniamo
\[
E^{\ast p,q}_0=\Hom(C^p_n/C^{p-1}_n,G)=\Hom(E^{p,q}_0,G).
\]
Possiamo quindi interpretare gli elementi di \(E^{\ast p,q}_0\) come funzioni che associano a ogni cubo di \(Q^p_n\) un elemento di \(G\) e che sono nulle sui cubi degeneri e su quelli di \(Q^{p-1}_n\).

Ripercorriamo brevemente i passi della sezione precedente, applicati questa volta a \(E^{\ast p,q}_r\). Per ogni \(p\) definiamo il complesso \(J^\bullet_p=\Hom(J^p_\bullet, G)\) e i morfismi \(\map{\varphi_p}{J^\bullet_p}{E^{\ast p,\bullet}_0}\), \(\map{\psi_p}{E^{\ast p,\bullet}_0}{J^\bullet_p}\), trasposti di \(\varphi^p\) e \(\psi^p\). Naturalmente \(\varphi_p\) è un'equivalenza omotopica e \(\psi_p\) è una sua inversa omotopica, pertanto il morfismo indotto \(\varphi_*\) è un isomorfismo fra \(E^{\ast p,q}_1\) e \(H^q(J^\bullet_p)\). Poiché \(C_p(B)\) è un gruppo abeliano libero, vale
\[
H^q(J^\bullet_p)=H^q(\Hom(C_p(B),\Hom(C_\bullet(F),G)))=\Hom(C_p(B),H^q(F;G))=C^p(B;H^q(F;G)).
\]
Si mostra poi, in modo del tutto analogo a quanto fatto per la successione spettrale omologica, che le mappe di bordo \(d^{\ast p,q}_1\) vengono mandate da \(\psi^*\) nei morfismi di bordo standard di \(C^\bullet(B;H^q(F;G))\), da cui si ottiene il risultato duale del \thref{spectral-sequence-of-fibration-E-2}.
\begin{theorem}\thlabel{cohomological-spectral-sequence-of-fibration-E-2}
Supponiamo che il gruppo fondamentale di \(B\) agisca banalmente sui gruppi di coomologia di \(F\). Allora \(\psi^*\) induce un isomorfismo \(E^{\ast p,q}_2\iso H^p(B;H^q(F;G))\).
\end{theorem}


\section{Struttura moltiplicativa di \texorpdfstring{\(E^*_2\)}{E*2}}
Supponiamo ora che \(G\) sia anche un anello. Allora i gruppi di coomologia di \(E\) acquistano una struttura di \(G\)-algebra, e lo stesso accade per i gruppi \(E^\ast_r\). Infatti il \cupproduct{} definisce una struttura moltiplicativa su \(C^\bullet(E)\); si verifica facilmente che tale prodotto soddisfa la condizione \(C^n_p\cupp C^m_q\subs C^{m+n}_{p+q}\), dunque come mostrato nel paragrafo \ref{cohomological-spectral-sequence:multiplicative-structure} i termini della successione spettrale \(E^\ast_r\) ereditano una struttura di anello per la quale le mappe di bordo sono antiderivazioni. È naturale domandarsi come questa struttura si trasformi mediante l'isomorfismo del \thref{cohomological-spectral-sequence-of-fibration-E-2}.
Definiamo
\[
H^*(B;H^*(F))=\Dirsum_{p,q}H^p(B;H^q(F)).
\]
La proposizione precedente ci dice che \(\psi^*\) induce un isomorfismo \(E^\ast_2\iso H^*(B;H^*(F))\) di gruppi abeliani.
Osserviamo però che su \(H^*(B;H^*(F))\) è definita anche una struttura di anello, data dal \cupproduct{} su \(B\) e su \(F\):
\[
H^p(B;H^q(F))\times H^{p'}(B;H^{q'}(F))\longrightarrow H^{p+p'}(B;H^{q+q'}(F)).
\]
Vedremo con la proposizione seguente che \(\psi^*\) è anche un isomorfismo di \(G\)-algebre, a meno di un segno.

\begin{proposition}\thlabel{cohomological-spectral-sequence-of-fibration-ring}
Supponiamo che il gruppo fondamentale di \(B\) agisca banalmente sui gruppi di coomologia di \(F\). Siano \(g\in E^{\ast p,q}_2\), \(g'\in E^{\ast p',q'}_2\). Allora \(\psi^*(g\cdot g')=(-1)^{p'q}\psi^*(g)\cdot\psi^*(g')\).
\end{proposition}
\begin{proof}
Cerchiamo di studiare più approfonditamente la struttura di anello indotta dal \cupproduct{} su 
\[
C^\bullet(B;H^*(F;G))=\Dirsum_qC^\bullet(B;H^q(F;G))=H^*(J^\bullet_p).
\]
Siano dunque \(\bar{f}\in H^q(J^\bullet_p)\), \(\bar{f}'\in H^{q'}(J^\bullet_{p'})\), e consideriamo \(f\in J^q_p\), \(f'\in J^{q'}_{p'}\) rappresentanti delle classi di coomologia rispettivamente di \(\bar{f}\), \(\bar{f}'\). Possiamo interpretare \(f\) come una funzione che a ogni coppia di cubi \(u\in Q_p(B)\), \(v\in Q_q(F)\) associa un elemento \(f(u,v)\in G\); inoltre \(f(u,v)=0\) ogniqualvolta uno fra \(u\) e \(v\) è degenere. Allora il prodotto \(f\cupp f'\) appartiene a \(J^{q+q'}_{p+p'}\), e sviluppando il \cupproduct{} su \(B\) e su \(F\) si ottiene che
\[
(f\cupp f')(u,v)=\sum_{H,M}\rho_{H,L}\rho_{M,N}f(\lambda^0_Lu,\lambda^0_Nv)\cdot f'(\lambda^1_Hu,\lambda^1_Mv),
\]
dove \(H\) varia fra i sottoinsiemi di \(\{1,\ldots,p+p'\}\) di cardinalità \(p\), \(M\) fra quelli di \(\{1,\ldots,q+q'\}\) di cardinalità \(p'\), \(L\) e \(N\) sono i complementari di \(H\) e \(M\), e \(\rho_{H,L},\rho_{M,N}\) sono stati definiti insieme al \cupproduct{} nel paragrafo \ref{cubic-homology:cup-product}. Calcoliamo allora
\begin{align*}
\psi_{p+p'}(\varphi_p(f)\cupp\varphi_{p'}(f'))(u,v)&=(\varphi_p(f)\cupp\varphi_{p'}(f'))K(u,v)\\
&=\sum_P\rho_{P,Q}\varphi_p(f)(\lambda^0_QK(u,v))\cdot\varphi_{p'}(f')(\lambda^1_PK(u,v)),
\end{align*}
dove \(P\) varia fra i sottoinsiemi di \(\{1,\ldots,p+p'+q+q'\}\) di cardinalità \(p+q\) e \(Q\) è il complementare di \(P\). Ricordiamo che \(\varphi_p(f)\), come ogni elemento di \(E^{p,q}_0\), è una funzione che a ogni cubo di \(Q^p_n\) associa un elemento di \(G\), ed è nulla sui cubi degeneri e su quelli che appartengono a \(Q^{p-1}_n\), dove al solito \(n=p+q\). Poiché \(K(u,v)\in Q^{p+p'}_{n+n'}\), se \(P\cap\{1,\ldots,p+p'\}\) ha meno di \(p\) elementi allora \(\lambda^0_QK(u,v)\in Q^{p-1}_{n}\), dunque \(\varphi_{p}(f)(\lambda^0_QK(u,v))=0\). Analogamente se \(P\cap\{1,\ldots,p+p'\}\) ha più di \(p\) elementi allora \(\varphi_{p'}(f')(\lambda^1_PK(u,v))=0\). Pertanto possiamo limitarci a sommare sui \(P\) della forma \(P=H\cup(p+p'+M)\), dove \(H\subs\{1,\ldots,p+p'\}\) ha cardinalità \(p\) e \(M\subs\{1,\ldots,q+q'\}\) ha cardinalità \(q\). Denotiamo con \(L\), \(N\) i complementari di \(H\), \(M\). Le seguenti formule sono di verifica immediata (\(w\) denota un qualunque cubo di \(Q^{p+p'}_{n+n'}\)):
\begin{itemize}
\item \(B^p\lambda^0_Qw=\lambda^0_LB^{p+p'}w\);
\item \(B^{p'}\lambda^1_Pw=\lambda^1_HB^{p+p'}w\);
\item \(F^p\lambda^0_Qw=\lambda^0_NF^{p+p'}w\);
\item \(F^{p'}\lambda^1_Pw=\lambda^{0,1}_{L,H}\lambda^1_{p+p'+M}w\), dove l'operazione \(\lambda^{0,1}_{L,H}\) consiste nel rimpiazzare le coordinate con indice in \(L\) con \(0\) e quelle con indice in \(H\) con \(1\).
\end{itemize}
Osserviamo inoltre che
\begin{align*}
\rho_{P,Q}&=\rho_{H,L}\cdot\rho_{H,p+p'+N}\cdot\rho_{p+p'+M,L}\cdot\rho_{p+p'+M,p+p'+N}\\
&=\rho_{H,L}\cdot (-1)^0\cdot(-1)^{qp'}\cdot\rho_{M,N}\\
&=(-1)^{p'q}\rho_{H,L}\rho_{M,N}.
\end{align*}
Otteniamo dunque
\begin{align*}
&\phantomop\psi_{p+p'}(\varphi_p(f)\cupp\varphi_{p'}(f'))(u,v)\\
&=\sum_{H,M}(-1)^{p'q}\rho_{H,L}\rho_{M,N} f(B^p\lambda^0_QK(u,v),F^p\lambda^0_QK(u,v))\cdot f'(B^{p'}\lambda^1_PK(u,v),F^{p'}\lambda^1_PK(u,v))\\
&=\sum_{H,M}(-1)^{p'q}\rho_{H,L}\rho_{M,N}f(\lambda^0_Lu,\lambda^0_Nv)\cdot f'(\lambda^1_Hu,\lambda^{0,1}_{L,H}K(u,\lambda^1_Mv)).
\end{align*}
Consideriamo l'applicazione \(T\) che a ogni cubo \(w\) di \(F\) di dimensione \(q'\) associa il cubo \(Tw\) di \(E\) di dimensione \(q'+1\) definito da
\[
(Tw)(t,x_1,\ldots,x_{q'})=K(u,w)(t\delta_1,\ldots,t\delta_{p+p'},x_1,\ldots,x_{q'}),
\]
dove \(\delta_i=0\) se \(i\in L\) e \(\delta_i=1\) se \(i\in H\). Si vede facilmente che \(T\) è una costruzione subordinata al cammino \(t\mapsto u(t\delta_1,\ldots,t\delta_{p+p'})\) e che \(S_Tw=\lambda^{0,1}_{L,H}K(u,w)\). Se denotiamo con \(S_{H,u}\) il morfismo \(S_T\) associamo all'applicazione \(T\) determinata dai valori fissati di \(H\) e \(u\), possiamo scrivere
\[
\psi_{p+p'}(\varphi_p(f)\cupp\varphi_{p'}(f'))(u,v)=(-1)^{p'q}\sum_{H,M}\rho_{H,L}\rho_{M,N}f(\lambda^0_Lu,\lambda^0_Nv)\cdot f'(\lambda^1_Hu,S_{H,u}\lambda^1_Mv).
\]
Passando al quoziente, ricordando che \(S_{H,u}\) induce l'identità in coomologia, otteniamo
\[
\bar{f}\cdot\bar{f}'=\psi^*(\varphi^*(\bar{f})\cdot\varphi^*(\bar{f}')),
\]
da cui, passando nuovamente al quoziente, si ha la tesi.
\end{proof}

\begin{corollary}\thlabel{cohomological-spectral-sequence-of-fibration-anticommutative}
Supponiamo che il gruppo fondamentale di \(B\) agisca banalmente sui gruppi di coomologia di \(F\). Allora gli anelli \(E^\ast_r\) con \(r\ge 2\) sono anticommutativi rispetto al grado totale \(n=p+q\).
\end{corollary}
\begin{proof}
Sappiamo che l'anello \(H^*(F;-)\) è anticommutativo rispetto al grado di \(F\) e che l'anello \(H^*(B;-)\) è anticommutativo rispetto al grado di \(B\). Siano \(g\in E^{\ast p,q}_2,g'\in E^{\ast p',q'}_2\). Allora
\begin{align*}
g\cdot g'&=\varphi^*(\psi^*(g\cdot g'))\\
&=\varphi^*((-1)^{p'q}\psi^*(g)\cdot\psi^*(g'))\\
&=(-1)^{p'q}\varphi^*((-1)^{pp'+qq'}\psi^*(g')\cdot\psi^*(g))\\
&=(-1)^{p'q}(-1)^{pp'+qq'}(-1)^{pq'}\varphi^*(\psi^*(g'))\cdot\varphi^*(\psi^*(g))\\
&=(-1)^{nn'}g'\cdot g,
\end{align*}
dove abbiamo sfruttato la \thref{cohomological-spectral-sequence-of-fibration-ring} e il risultato analogo per \(\varphi^*\). Dunque la tesi è vera per \(r=2\). Ma allora è vera per ogni \(r\ge 2\), poiché la struttura di anello di \(E^\ast_r\) si ottiene per passaggio al quoziente da quella di \(E^\ast_{r-1}\).
\end{proof}

\section{Semplificazione di \texorpdfstring{\(E^*_2\)}{E*2}}
Continuiamo a supporre che \(G\) sia un anello. Sotto alcune ipotesi risulta possibile semplificare il calcolo del termine \(E^\ast_2\) della successione spettrale. Se \(M\) e \(N\) sono due \(G\)-algebre graduate, possiamo definire su \(M\tensor N\) una struttura di \(G\)-algebra come segue: se \(x,x'\in M\) e \(y,y'\in N\) sono elementi omogenei di gradi, rispettivamente \(p,p',q,q'\), allora
\[
(x\tensor y)(x'\tensor y')=(-1)^{p'q}(xx'\tensor yy').
\]
\begin{proposition}\thlabel{cohomological-spectral-sequence-of-fibration-simplification}
Supponiamo che \(G\) sia un PID, che il gruppo fondamentale di \(B\) agisca banalmente sui gruppi di coomologia di \(F\) e che sia soddisfatta una delle due condizioni seguenti:
\begin{enumerate}
\item \(H_p(B;G)\) è un \(G\)-modulo libero finitamente generato per ogni \(p\ge 0\);
\item \(H^q(F;G)\) è un \(G\)-modulo libero finitamente generato per ogni \(q\ge 0\).
\end{enumerate}
Allora esiste un isomorfismo di \(G\)-algebre \(H^*(B;G)\tensor H^*(F;G)\iso E^\ast_2\).
\end{proposition}
\begin{proof}
Consideriamo l'applicazione
\Map{i}{C^p(B;G)\tensor H^q(F;G)}{C^p(B;H^q(F;G))}{b\tensor f}{(u\mapsto b(u)\cdot f).}
Per passaggio al quoziente otteniamo
\[
\map{i^*}{H^p(B;G)\tensor H^q(F;G)}{H^p(B;H^q(F;G))}.
\]
È evidente che \(\varphi^*\circ i^*\) preserva la struttura moltiplicativa (ricordiamo la \thref{cohomological-spectral-sequence-of-fibration-ring}), dunque è sufficiente mostrare che, sotto le ipotesi dell'enunciato, \(i^*\) è un isomorfismo di \(G\)-moduli.
\begin{enumerate}
\item Per il teorema dei coefficienti universali abbiamo che
\[
H^p(B;H^q(F;G))\iso\Hom(H_p(B;G),H^q(F;G))
\]
e
\[
H^p(B;G)\iso\Hom(H_p(B;G),G),
\]
dunque possiamo interpretare \(i^*\) come la mappa canonica
\[
\map{i^*}{\Hom(H_p(B;G),G)\tensor H^q(F;G)}{\Hom(H_p(B;G),H^q(F;G))}.
\]
Per additività di \(\Hom(-,-)\) e di \(-\tensor-\), essendo \(H_p(B;G)\) libero e finitamente generato, possiamo limitarci a considerare il caso \(H_p(B;G)=G\), in cui \(i^*\) è ovviamente un isomorfismo.
\item Per additività di \(-\tensor-\) e di \(H^p(B;-)\), essendo \(H^q(F;G)\) libero e finitamente generato, possiamo limitarci a considerare il caso \(H^q(F;G)=G\), in cui \(i^*\) è ovviamente un isomorfismo.\qedhere
\end{enumerate}
\end{proof}

\section{Applicazioni}
Presentiamo innanzitutto il risultato duale della \thref{fibration-homology-exact-sequence}; omettiamo la dimostrazione, del tutto analoga.
\begin{proposition}\thlabel{fibration-cohomology-exact-sequence}
Siano \(G\) un PID, \(\fibration{F}{E}{B}\) una fibrazione in cui il gruppo fondamentale di \(B\) agisce banalmente sui moduli di coomologia di \(F\). Supponiamo che \(H^i(B;G)=0\) per \(0<i<p\) e che \(H^i(F;G)=0\) per \(0<i<q\). Allora esiste una successione esatta
\longexactsequence{H^1(B;G)\rar\&H^1(E;G)\rar\&H^1(F;G)\rar{d_2}\&H^2(B;G)\rar\&\ldots}{\ldots\rar\&H^{p+q-2}(F;G)\rar{d_{p+q-1}}\&H^{p+q-1}(B;G)\rar\&H^{p+q-1}(E;G)\rar\&H^{p+q-1}(F;G).}
\end{proposition}

Riportiamo ora un risultato noto come \enquote{Successione esatta di Wang}.
\begin{proposition}\thlabel{wang-exact-sequence}
Siano \(G\) un PID, \(\fibration{F}{E}{B}\) una fibrazione in cui il gruppo fondamentale di \(B\) agisce banalmente sui moduli di coomologia di \(F\). Supponiamo che \(B\) sia semplicemente connesso e abbia la stessa \(G\)-algebra di coomologia della sfera \(S^k\) con \(k\ge 2\).  Allora esiste una successione esatta
\begin{diagram}
\ldots\rar&H^n(E;G)\rar&H^n(F;G)\rar{\theta}&H^{n-k+1}(F;G)\rar&H^{n+1}(E;G)\rar&\ldots
\end{diagram}
Inoltre \(\theta\) è una derivazione se \(k\) è dispari e un'antiderivazione se \(k\) è pari, ossia
\[
\theta(x\cdot y)=\theta x\cdot y+(-1)^{(k+1)\deg x}x\cdot\theta y.
\]
\end{proposition}
\begin{proof}
Denotiamo con \(E^{p,q}_r\) la successione spettrale coomologica associata alla fibrazione. Per ipotesi \(H^i(B;G)\) è un \(G\)-modulo libero finitamente generato per ogni \(i\), dunque per la \thref{cohomological-spectral-sequence-of-fibration-simplification} \(E_2\) è isomorfo come \(G\)-algebra a \(H^*(B;G)\tensor H^*(F;G)\). Pertanto, per ogni grado totale \(n\), \(E_2\) ha al più due termini \(E^{i,j}_2\) non nulli, corrispondenti a \(i=0\) e \(i=k\). Applicando la \thref{cohomological-spectral-exact-sequence} (è evidente che le ipotesi sono soddisfatte) otteniamo la successione esatta
\begin{diagram}
\ldots\rar&H^n(E;G)\rar&E^{0,n}_2\rar{d_k}&E^{k,n-k+1}_2\rar&H^{n+1}(E;G)\rar&\ldots
\end{diagram}
Ricordiamo che \(E^{0,n}_2=H^n(F;G)\) e che \(E^{k,n-k+1}_2=H^k(B;G)\tensor H^{n-k+1}(F;G)\).
Sia \(s\) un generatore di \(H^k(B;G)\); consideriamo l'isomorfismo
\Map{g}{H^{n-k+1}(F;G)}{H^k(B;G)\tensor H^{n-k+1}(F;G)}{x}{s\tensor x.}
Posto \(\theta=g^{-1}d_k\), otteniamo la successione esatta della tesi. Per mostrare la seconda parte, calcoliamo
\begin{align*}
d_k(x\cdot y)&=d_kx\cdot y+(-1)^{\deg x}x\cdot d_ky\\
&=(s\tensor\theta x)\cdot(1\tensor y)+(-1)^{\deg x}(1\tensor x)\cdot (s\tensor\theta y)\\
&=s\tensor(\theta x\cdot y+(-1)^{(k+1)\deg x}x\cdot\theta y),
\end{align*}
ma anche \(d_k(x\cdot y)=s\tensor\theta(x\cdot y)\), da cui la tesi.
\end{proof}