\chapter{Spazi di cammini}
\section{H-spazi}
\begin{definition}
Sia $G$ uno spazio topologico munito di un prodotto $\map{\vee}{G\times G}{G}$ continuo. $G$ si dice H-spazio se esiste $e\in G$ con $e\vee e=e$ tale che le applicazioni da $G$ in $G$ definite da $x\mapsto x\vee e$ e $x\mapsto e\vee x$ siano omotope all'identità mediante omotopie che fissano $e$.
\end{definition}
\begin{proposition}\thlabel{h-space-trivial-automorphisms-of-covering-space}
Sia $G$ un H-spazio connesso, $\map{p}{T}{G}$ un rivestimento. Allora gli automorfismi di rivestimento di $T$ sono omotopi all'identità.
\end{proposition}
\begin{proof}
Sia $\map{f}{T}{T}$ un automorfismo di rivestimento, $\tilde{e}$ un elemento della fibra di $e$, $\tilde{e}'=f(\tilde{e})$. Sia poi $\map{\gamma}{I}{G}$ con $\gamma(0)=\gamma(1)=e$ il cui sollevamento $\tilde\gamma$ soddisfi $\tilde\gamma(0)=\tilde{e},\tilde\gamma(1)=\tilde{e}'$. Sia infine $\map{h}{I\times G}{G}$ un'omotopia con $h_0(x)=x$ e $h_1(x)=e\vee x$. Definiamo l'omotopia
\Map{H}{I\times T}{G}{(x,t)}{
\begin{cases}
h_{3t}(p(x))&t\le\frac{1}{3}\\
\gamma(3t-1)\vee p(x)&\frac{1}{3}\le t\le\frac{2}{3}\\
h_{3-3t}(p(x))&\frac{2}{3}\le t
\end{cases}}
Per la proprietà di sollevamento dell'omotopia, esiste un'omotopia $\map{\tilde{H}}{I\times T}{T}$ con $\tilde{H}_0=\1$ e $p\tilde{H}=H$. Osserviamo che il cammino $t\mapsto H_t(\tilde{e})$ è omotopo a $\gamma$ e $\tilde{H}_0(\tilde{e})=\tilde{e}$, pertanto $\tilde{H}_1(\tilde{e})=\tilde{\gamma}(1)=\tilde{e}'$. Ma allora $\tilde{H}_1$ è un sollevamento dell'identità di $G$ tale che $\tilde{H}_1(\tilde{e})=f(\tilde{e})$. Dalla connessione di $G$ segue che $\tilde{H}_1=f$. Ma $\tilde{H}_1=f$ e $\tilde{H}_0=\1$ sono omotope mediante $\tilde H$.
\end{proof}
\begin{corollary}\thlabel{h-space-trivial-action}
Sia $G$ un H-spazio connesso, $T$ il suo rivestimento universale. Allora il gruppo fondamentale di $G$ agisce banalmente sui gruppi di omotopia, di omologia e di coomologia di $T$.
\end{corollary}

\section{Prime proprietà degli spazi di cammini}
Dati due spazi topologici $X,Y$, denotiamo con $C(X,Y)$ l'insieme delle funzioni continue da $X$ in $Y$. Riportiamo alcune nozioni di base relative alla topologia compatta-aperta.
\begin{definition}
La topologia compatta-aperta su $C(X,Y)$ è la topologia generata da $\{V(K,U):\text{$K\subs X$ compatto, $U\subs Y$ aperto}\}$, dove $V(K,U)$ è l'insieme delle funzioni $f\in C(X,Y)$ tali che $f(K)\subs U$.
\end{definition}
D'ora in poi considereremo sempre su $C(X,Y)$ la topologia compatta-aperta.
\begin{proposition}\thlabel{compact-open-topology-properties}
Siano $X,Y,Z$ spazi topologici con $X$ localmente compatto di Hausdorff.
\begin{enumerate}
\item L'applicazione di valutazione
\Map{\omega}{X\times C(X,Y)}{Y}{(x,f)}{f(x)}
è continua.
\item Una funzione $\map{g}{Z}{C(X,Y)}$ è continua se e solo se l'applicazione
\Map{G}{Z\times X}{Y}{(z,x)}{g(z)(x)}
è continua.
\end{enumerate}
\end{proposition}

Dato uno spazio topologico $X$ e due sottospazi $A,B\subs X$, denotiamo con $E_{A,B}$ il sottospazio di $C(I,X)$ delle funzioni $f$ tali che $f(0)\in A$ e $f(1)\in B$. Con lieve abuso di notazione, scriveremo $E_{x,B}$ in luogo di $E_{\{x\},B}$ se $A=\{x\}$, e analogamente per $B$.

\begin{proposition}\thlabel{path-space-contractible}
Per ogni $x\in X$ lo spazio $E_{x,X}$ è contrattile.
\end{proposition}
\begin{proof}
Definiamo l'applicazione
\Map{H}{I\times E_{x,X}}{E_{x,X}}{(s,f)}{H(s,f)}
dove $H(s,f)(t)=f(st)$. Per la \thref{compact-open-topology-properties}, $H$ è continua. Inoltre $H_1$ è l'identità, mentre per ogni $f$ $H_0(f)$ è il cammino che vale costantemente $x$. Dunque l'identità su $E_{x,X}$ è omotopa a un'applicazione costante, ossia $E_{x,X}$ è contrattile.
\end{proof}

Dati due cammini $f\in E_{x,y},g\in E_{y,z}$ si definisce il cammino $f\ast g\in E_{x,z}$ come
$$
(f\ast g)(t)=
\begin{cases}
f(2t)&t\le\frac{1}{2}\\
g(2t-1)&t\ge\frac{1}{2}
\end{cases}.
$$
Per ogni $x\in X$, definiamo $\Omega_x=E_{x,x}$.
\begin{proposition}\thlabel{omega-h-space}
$\Omega_x$, munito del prodotto $\ast$, è un H-spazio.
\end{proposition}
\begin{proof}
Mostriamo innanzitutto che $\ast$ è continuo. Grazie alla \thref{compact-open-topology-properties}, è sufficiente dimostrare che l'applicazione da $\Omega_x\times\Omega_x\times I$ in $X$ definita da $(f,g,t)\mapsto (f\ast g)(t)$ è continua, e ciò segue dalla continuità di $(f,t)\mapsto f(2t)$ e di $(g,t)\mapsto g(2t-1)$.

Mostriamo poi che l'applicazione
\Map{\varphi_1}{\Omega_x}{\Omega_x}{f}{f\ast e}
è omotopa all'identità su $\Omega_x$ (mediante un'omotopia che fissa $e$), dove $e$ è il cammino che vale costantemente $x$ (è evidente che $e\ast e=e$). È sufficiente considerare, per $s\in I$ e $f\in\Omega_x$, 
$$
\varphi_s(f)(t)=
\begin{cases}
f((s+1)t)&t\le 1-\frac{s}{2}\\
x&t\ge 1-\frac{s}{2}
\end{cases}.
$$
Osserviamo che $\varphi_s(f)(0)=\varphi_s(f)(1)=x$, dunque $\varphi_s(f)\in\Omega_x$; inoltre $\varphi_0(f)=f$, $\varphi_1(f)=f\ast e$ e $\varphi_s(e)=e$. Pertanto è sufficiente mostrare che $\map{\varphi}{I\times\Omega_x}{\Omega_x}$ è continua, ossia, per la \thref{compact-open-topology-properties}, che
\Map{\Phi}{I\times\Omega_x\times I}{X}{(s,f, t)}{\varphi_s(f)(t)}
è continua. Si vede però che $\Phi(s,f, t)=f(\theta(t,s))$, dove
$$
\theta(t,s)=
\begin{cases}
(s+1)t&t\le 1-\frac{s}{2}\\
1&t\ge 1-\frac{s}{2}
\end{cases},
$$
perciò $\Phi$ è continua. In modo del tutto analogo si mostra che $f\mapsto e\ast f$ è omotopa all'identità.
\end{proof}
\begin{proposition}\thlabel{path-space-on-contractile-base}
Supponiamo che $A$ si contragga a un punto $x\in X$. Allora $E_{A,B}$ è omotopicamente equivalente a $A\times E_{x,B}$
\end{proposition}
\begin{proof}
Per ipotesi esiste un'applicazione $\map{D}{I\times A}{X}$ tale che $D(0,a)=a$ e $D(1,a)=x$ per ogni $a\in A$. Denotiamo con $f_a\in E_{a,x}$ il cammino $f_a(t)=D(t,a)$ e con $g_a\in\ E_{x,a}$ il cammino $g_a(t)=D(1-t,a)$. Definiamo le applicazioni continue
\Map{\varphi}{A\times E_{x,B}}{E_{A,B}}{(a,h)}{f_a\ast h}
\Map{\psi}{E_{A,B}}{A\times E_{x,B}}{h}{(h(0),g_{h(0)}\ast h)}
Abbiamo
\begin{align*}
\varphi\psi(h)&=f_{h(0)}\ast(g_{h(0)}\ast h)\\
\psi\varphi(a,h)&=(a,g_a\ast(f_a\ast h)).
\end{align*}
\todo{Concludere la dimostrazione noiosa}
\end{proof}
\begin{corollary}
Se $A$ e $B$ si contraggono rispettivamente a $x,y$, allora $E_{A,B}$ è omotopicamente equivalente a $A\times B\times E_{x,y}$.
\end{corollary}
\begin{corollary}
Supponiamo che $X$ sia connesso per archi. Allora il tipo di omotopia di $E_{x,y}$ non dipende dalla scelta di $x$ e $y$.
\end{corollary}
In particolare, se $X$ è connesso per archi il tipo di omotopia di $\Omega_x$ è indipendente da $x$. Indicheremo allora con $\Omega X$ (o semplicemente con $\Omega$) lo spazio dei cammini chiusi aventi estremi in un punto $x\in X$ fissato, ma irrilevante. Dalla \thref{compact-open-topology-properties} segue facilmente che $\Omega$ è connesso per archi se e solo se $X$ è semplicemente connesso.

\section{Fibrazione degli spazi di cammini}
\begin{proposition}\thlabel{path-space-fibration}
Sia $X$ uno spazio topologico connesso per archi, e siano $A,B\subs X$ due sottospazi. Allora l'applicazione
\Map{p}{E_{A,B}}{A\times B}{f}{(f(0),f(1))}
è una fibrazione.
\end{proposition}
\begin{proof}
Notiamo subito che $p$ è suriettiva, in quanto $X$ è connesso per archi. Mostreremo ora che $p$ soddisfa la proprietà di sollevamento dell'omotopia per tutti gli spazi topologici (e non solo per i poliedri finiti). Sia $Y$ uno spazio topologico, $\map{f=(f_A,f_B)}{I\times Y}{A\times B}$ un'applicazione continua, $\map{g}{Y}{E_{A,B}}$ tale che $pg(y)=f(0,y)$ per ogni $y\in Y$. Per la \thref{compact-open-topology-properties}, l'applicazione
\Map{G}{Y\times I}{X}{(y,t)}{g(y)(t)}
è continua. Dobbiamo trovare una mappa continua $\map{h}{I\times Y}{E_{A,B}}$ tale che $h(0,y)=g(y)$ e $ph=f$ o, equivalentemente, $\map{H}{I\times Y\times I}{X}$ tale che $H(0,y,t)=G(y,t),H(s,y,0)=f_A(s,y),H(s,y,1)=f_B(s,y)$. Dobbiamo dunque estendere a tutto $I\times Y\times I$ una funzione definita su
$$
(\{0\}\times Y\times I)\cup(I\times Y\times \{0\})\cup(I\times Y\times \{1\}),
$$
e ciò è reso possibile dal fatto che $(\{0\}\times I)\cup(I\times\{0\})\cup(I\times\{1\})$ è un retratto di $I\times I$.
\end{proof}

\begin{proposition}
Sia $X$ uno spazio topologico connesso per archi e semplicemente connesso, $x\in X$. Allora esiste una successione spettrale $E^{p,q}_r\converges H_{p+q}(E_{x,X})$ tale che $E^{p,q}_2=H_p(X;H_q(\Omega))$.
\end{proposition}
\begin{proof}
Consideriamo la fibrazione $E_{x,X}\to X$ della \thref{path-space-fibration} (dove abbiamo identificato $\{x\}\times X$ con $X$). Le fibre sono spazi del tipo $E_{x,y}$, dunque omotopicamente equivalenti a $\Omega$. Poiché $X$ è semplicemente connesso, $\Omega$ è connesso per archi, e inoltre l'azione di $\pi_1(X)$ sui gruppi di omologia di $\Omega$ è banale. Siamo dunque nelle condizioni di applicare \missing{}, da cui segue immediatamente la tesi.
\end{proof}
Naturalmente esiste la successione spettrale duale in coomologia.

\begin{proposition}\thlabel{loop-space-finitely-generated-homology}
Siano $A$ un PID, $X$ uno spazio topologico connesso per archi e semplicemente connesso. Supponiamo che tutti i moduli di omologia di $X$ a coefficienti in $A$ siano finitamente generati. Allora vale lo stesso per $\Omega$.
\end{proposition}
\begin{proof}
Sia $x\in X$ un punto. Poiché $E_{x,X}$ è contrattile, tutti i suoi moduli di omologia sono finitamente generati (in particolare $H_0(E_{x,X};A)=A$ e $H_i(E_{x,X};A)=0$ per $i>0$). Applicando la \thref{finitely-generated-homology-groups} alla fibrazione $\fibration{\Omega}{E_{x,X}}{X}$ si ottiene immediatamente la tesi.
\end{proof}

\begin{proposition}\thlabel{loop-space-suspension-isomorphism}
Sia $A$ un PID, $X$ uno spazio topologico connesso per archi e semplicemente connesso. Supponiamo che $H_i(X;A)=0$ per $0<i<p$. Allora la sospensione $\map{\Sigma}{H_i(\Omega;A)}{H_{i+1}(X;A)}$ è un isomorfismo per $0<i<2p-2$ ed è suriettiva per $i=2p-2$.
\end{proposition}
\begin{proof}
La tesi segue immediatamente dal \thref{fibration-suspension-isomorphism} applicato alla fibrazione $\fibration{\Omega}{E_{x,X}}{X}$, ricordando che $E_{x,X}$ è contrattile.
\end{proof}