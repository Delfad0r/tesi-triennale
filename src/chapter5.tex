\chapter{Spazi di cammini}
\section{H-spazi}
\begin{definition}
Sia $\Omega$ uno spazio topologico munito di un prodotto $\map{\vee}{\Omega\times \Omega}{\Omega}$ continuo. $\Omega$ si dice H-spazio se esiste $e\in \Omega$ con $e\vee e=e$ tale che le applicazioni da $\Omega$ in $\Omega$ definite da $x\mapsto x\vee e$ e $x\mapsto e\vee x$ siano omotope all'identità mediante omotopie che fissano $e$.
\end{definition}
\begin{proposition}\thlabel{h-space-trivial-automorphisms-of-covering-space}
Siano $\Omega$ un H-spazio, $\map{p}{T}{\Omega}$ un rivestimento connesso. Allora gli automorfismi di rivestimento di $T$ sono omotopi all'identità.
\end{proposition}
\begin{proof}
Siano $\map{f}{T}{T}$ un automorfismo di rivestimento, $\tilde{e}$ un elemento della fibra di $e$, $\tilde{e}'=f(\tilde{e})$. Sia poi $\map{\gamma}{I}{\Omega}$ con $\gamma(0)=\gamma(1)=e$ il cui sollevamento $\tilde\gamma$ soddisfi $\tilde\gamma(0)=\tilde{e},\tilde\gamma(1)=\tilde{e}'$. Sia infine $\map{h}{\Omega\times I}{\Omega}$ un'omotopia con $h_0(x)=x$ e $h_1(x)=e\vee x$. Definiamo l'omotopia
\Map{H}{T\times I}{\Omega}{(x,t)}{
\begin{cases}
h_{3t}(p(x))&t\le\frac{1}{3}\\
\gamma(3t-1)\vee p(x)&\frac{1}{3}\le t\le\frac{2}{3}\\
h_{3-3t}(p(x))&\frac{2}{3}\le t
\end{cases}}
Per la proprietà di sollevamento dell'omotopia, esiste un'omotopia $\map{\tilde{H}}{T\times I}{T}$ con $\tilde{H}_0=\1$ e $p\tilde{H}=H$. Osserviamo che il cammino $t\mapsto H_t(\tilde{e})$ è omotopo a $\gamma$ e $\tilde{H}_0(\tilde{e})=\tilde{e}$, pertanto $\tilde{H}_1(\tilde{e})=\tilde{\gamma}(1)=\tilde{e}'$. Ma allora $\tilde{H}_1$ è un automorfismo di rivestimento tale che $\tilde{H}_1(\tilde{e})=f(\tilde{e})$. Dalla connessione di $T$ segue che $\tilde{H}_1=f$. Pertanto $\tilde{H}_1=f$ e $\tilde{H}_0=\1$ sono omotope mediante $\tilde H$, ossia la tesi.
\end{proof}
\begin{corollary}\thlabel{h-space-trivial-action}
Siano $\Omega$ un H-spazio connesso, $T$ il suo rivestimento universale. Allora il gruppo fondamentale di $\Omega$ agisce banalmente sui moduli di omologia e di coomologia di $T$.
\end{corollary}

\section{Prime proprietà degli spazi di cammini}
\subsection{Topologia compatta-aperta}
Dati due spazi topologici $X,Y$, denotiamo con $C(X,Y)$ l'insieme delle funzioni continue da $X$ in $Y$. Riportiamo alcune nozioni di base relative alla topologia compatta-aperta.
\begin{definition}
La topologia compatta-aperta su $C(X,Y)$ è la topologia generata da 
$$
\{V(K,U):\text{$K\subs X$ compatto, $U\subs Y$ aperto}\},
$$
dove $V(K,U)$ è l'insieme delle funzioni $f\in C(X,Y)$ tali che $f(K)\subs U$.
\end{definition}
D'ora in poi considereremo sempre $C(X,Y)$ come spazio topologico munito della topologia compatta-aperta.
\begin{proposition}\thlabel{compact-open-topology-properties}
Siano $X,Y,Z$ spazi topologici con $X$ localmente compatto di Hausdorff.
\begin{enumerate}
\item L'applicazione di valutazione
\uMap{X\times C(X,Y)}{Y}{(x,f)}{f(x)}
è continua.
\item Una funzione $\map{g}{Z}{C(X,Y)}$ è continua se e solo se l'applicazione
\uMap{Z\times X}{Y}{(z,x)}{g(z)(x)}
è continua.
\end{enumerate}
\end{proposition}

\subsection{H-spazio dei cammini chiusi}

Dato uno spazio topologico $X$ e due sottospazi $A,B\subs X$, denotiamo con $E_{A,B}$ il sottospazio di $C(I,X)$ delle funzioni $f$ tali che $f(0)\in A$ e $f(1)\in B$. Con lieve abuso di notazione, scriveremo $E_{x,B}$ in luogo di $E_{\{x\},B}$ se $A=\{x\}$, e analogamente per $B$. Per ogni $x\in X$, definiamo $\Omega_x=E_{x,x}$.

Dati due cammini $f\in E_{x,y},g\in E_{y,z}$ si definisce il cammino $f\ast g\in E_{x,z}$ come
$$
(f\ast g)(t)=
\begin{cases}
f(2t)&t\le\frac{1}{2}\\
g(2t-1)&t\ge\frac{1}{2}
\end{cases}.
$$

\begin{proposition}\thlabel{omega-h-space}
$\Omega_x$, munito del prodotto $\ast$, è un H-spazio.
\end{proposition}
\begin{proof}
Mostriamo innanzitutto che $\ast$ è continuo. Grazie alla \thref{compact-open-topology-properties}, è sufficiente dimostrare che l'applicazione da $\Omega_x\times\Omega_x\times I$ in $X$ definita da $(f,g,t)\mapsto (f\ast g)(t)$ è continua, e ciò segue dalla continuità di $(f,t)\mapsto f(2t)$ e di $(g,t)\mapsto g(2t-1)$.

Mostriamo poi che l'applicazione $(-\ast e)$
è omotopa all'identità su $\Omega_x$ (mediante un'omotopia che fissa $e$), dove $e$ è il cammino che vale costantemente $x$ (è evidente che $e\ast e=e$). È sufficiente considerare, per $s\in I$ e $f\in\Omega_x$, 
$$
\varphi_s(f)(t)=
\begin{cases}
f((s+1)t)&t\le 1-\frac{s}{2}\\
x&t\ge 1-\frac{s}{2}
\end{cases}.
$$
Osserviamo che $\varphi_s(f)(0)=\varphi_s(f)(1)=x$, dunque $\varphi_s(f)\in\Omega_x$; inoltre $\varphi_0=\1$, $\varphi_1=(-\ast e)$ e $\varphi_s(e)=e$. Pertanto è sufficiente mostrare che $\map{\varphi}{\Omega_x\times I}{\Omega_x}$ è continua, ossia, per la \thref{compact-open-topology-properties}, che
\Map{\Phi}{\Omega_x\times I\times I}{X}{(f,s,t)}{\varphi_s(f)(t)}
è continua. Si vede però che $\Phi(f,s,t)=f(\theta(s,t))$, dove
$$
\theta(s,t)=
\begin{cases}
(s+1)t&t\le 1-\frac{s}{2}\\
1&t\ge 1-\frac{s}{2}
\end{cases},
$$
perciò $\Phi$ è continua. In modo del tutto analogo si mostra che $(e\ast-)$ è omotopa all'identità.
\end{proof}

\subsection{Tipi di omotopia degli spazi di cammini}

Fissiamo un punto $x\in X$, e denotiamo con $e$ il cammino che vale costantemente $x$.

\begin{proposition}\thlabel{path-space-contractible}
Lo spazio $E_{x,X}$ è contrattile.
\end{proposition}
\begin{proof}
Definiamo l'omotopia
\Map{H}{E_{x,X}\times I}{E_{x,X}}{(f,s)}{H_s(f)}
dove $H_s(f)(t)=f(st)$. Per la \thref{compact-open-topology-properties}, $H$ è continua. Inoltre $H_1$ è l'identità, mentre per ogni $f$ vale $H_0(f)=e$. Dunque l'identità su $E_{x,X}$ è omotopa a un'applicazione costante, ossia $E_{x,X}$ è contrattile.
\end{proof}

\begin{proposition}\thlabel{path-space-independent-of-point}
Supponiamo che $X$ sia connesso per archi. Siano $y\in X$ un punto, $B\subs X$ un sottospazio. Allora $E_{x,B}$ e $E_{y,B}$ sono omotopicamente equivalenti.
\end{proposition}
\begin{proof}
Sia $\gamma\in E_{x,y}$, e sia $\delta\in E_{y,x}$ il cammino inverso (ossia $\delta(t)=\gamma(1-t)$). Consideriamo le applicazioni $\map{(\gamma\ast-)}{E_{y,B}}{E_{x,B}}$ e $\map{(\delta\ast-)}{E_{x,B}}{E_{y,B}}$ (si vede facilmente che sono continue). Mostriamo che sono una l'inversa omotopica dell'altra. Non è difficile vedere che $(\delta\ast(\gamma\ast-))$ è omotopa a $((\delta\ast\gamma)\ast-)$. Presentiamo ora un lemma che ci permetterà di concludere.
\begin{lemma}
Sia $\alpha\in\Omega_x$ un cammino omotopicamente banale (ossia omotopo a $e$ mediante un'omotopia che fissa gli estremi). Allora l'applicazione $\map{(\alpha\ast-)}{E_{x,B}}{E_{x,B}}$ è omotopa all'identità.
\end{lemma}
\begin{proof}
Per ipotesi esiste un'omotopia $\map{H}{I\times I}{X}$ con $H_0=\alpha$ e $H_1=e$. Definiamo allora l'omotopia $\map{K}{E_{x,B}\times I}{E_{x,B}}$ come $K_t=(H_t\ast-)$. Si verifica immediatamente che $K$ è effettivamente un'omotopia fra $(\alpha\ast-)$ e $(e\ast-)$. Procedendo come nella dimostrazione della \thref{omega-h-space} si ottiene che $(e\ast-)$ è omotopa all'identità, da cui la tesi.
\end{proof}
Applicando il Lemma a $\alpha=\delta\ast\gamma$ otteniamo che $(\delta\ast(\gamma\ast-))$ è omotopa all'identità. In modo del tutto analogo si mostra che $(\gamma\ast(\delta\ast-))$ è omotopa all'identità.
\end{proof}
\begin{corollary}\thlabel{path-space-independent-of-points}
Supponiamo che $X$ sia connesso per archi. Allora il tipo di omotopia di $E_{x,y}$ non dipende dalla scelta di $x$ e $y$.
\end{corollary}
In particolare, se $X$ è connesso per archi il tipo di omotopia di $\Omega_x$ è indipendente da $x$. Indicheremo allora con $\Omega X$ (o semplicemente con $\Omega$) lo spazio dei cammini chiusi aventi estremi in un punto $x\in X$ fissato, ma irrilevante. Dalla \thref{compact-open-topology-properties} segue facilmente che $\Omega$ è connesso per archi se e solo se $X$ è semplicemente connesso.

\section{Fibrazione degli spazi di cammini}
\begin{proposition}\thlabel{path-space-fibration}
Sia $X$ uno spazio topologico connesso per archi, e siano $A,B\subs X$ due sottospazi. Allora l'applicazione
\Map{p}{E_{A,B}}{A\times B}{f}{(f(0),f(1))}
è una fibrazione (con fibra non necessariamente connessa).
\end{proposition}
\begin{proof}
Notiamo subito che $p$ è suriettiva, in quanto $X$ è connesso per archi. Mostreremo ora che $p$ soddisfa la proprietà di sollevamento dell'omotopia per tutti gli spazi topologici (e non solo per i poliedri finiti). Siano $Y$ uno spazio topologico, $\map{f=(f_A,f_B)}{Y\times I}{A\times B}$ un'applicazione continua, $\map{g}{Y}{E_{A,B}}$ tale che $pg(y)=f(y,0)$ per ogni $y\in Y$. Per la \thref{compact-open-topology-properties}, l'applicazione
\Map{G}{Y\times I}{X}{(y,t)}{g(y)(t)}
è continua. Dobbiamo trovare una mappa continua $\map{h}{Y\times I}{E_{A,B}}$ tale che $h(y,0)=g(y)$ e $ph=f$ o, equivalentemente, $\map{H}{Y\times I\times I}{X}$ tale che $H(y,0,t)=G(y,t)$, $H(y,s,0)=f_A(y,s)$ e $H(y,s,1)=f_B(y,s)$. Dobbiamo dunque estendere a tutto $Y\times I\times I$ una funzione definita su
$$
(Y\times\{0\}\times I)\cup(Y\times I\times \{0,1\}),
$$
e ciò è reso possibile dal fatto che questo spazio è un retratto di $I\times I$.
\end{proof}

Consideriamo ora uno spazio topologico $X$ connesso per archi e semplicemente connesso.

\begin{corollary}\thlabel{loop-space-fibration}
Esistono uno spazio contrattile $E$ e una fibrazione $\map{p}{E}{X}$ con fibra connessa per archi e omotopicamente equivalente a $\Omega$.
\end{corollary}
\begin{proof}
Fissiamo un punto $x\in X$ e consideriamo la fibrazione $\map{p}{E_{x,X}}{\{x\}\times X}$ della \thref{path-space-fibration}. Naturalmente $\{x\}\times X$ è omeomorfo a $X$. Inoltre le fibre sono della forma $E_{x,y}$, dunque per il \thref{path-space-independent-of-points} sono omotopicamente equivalenti a $\Omega$.
\end{proof}

Possiamo così applicare agli spazi di cammini (almeno nel caso in cui $X$ è semplicemente connesso) i risultati generali sviluppati per gli spazi fibrati, sfruttando in più il fatto che lo spazio totale è contrattile. Con lieve abuso di notazione indicheremo la fibrazione del \thref{loop-space-fibration} con $\fibration{\Omega}{E}{X}$, anche se la fibra è solo omotopicamente equivalente a $\Omega$.

\begin{corollary}\thlabel{loop-space-finitely-generated-homology}
Supponiamo che i gruppi $H_i(X)$ siano finitamente generati. Allora anche i gruppi $H_i(\Omega)$ lo sono.
\end{corollary}
\begin{proof}
Consideriamo la fibrazione $\fibration{\Omega}{E}{X}$ del \thref{loop-space-fibration}. Poiché $E$ è contrattile, i suoi gruppi di omologia sono finitamente generati. Applicando la \thref{fibration-finitely-generated-homology-groups} si ottiene immediatamente la tesi.
\end{proof}

\begin{corollary}\thlabel{loop-space-homology-isomorphism}
Sia $G$ un PID. Supponiamo che $H_i(X;G)=0$ per $0<i<p$. Allora $\map{d_i}{H_i(X;G)}{H_{i-1}(\Omega;G)}$ è un isomorfismo per $2\le i\le p$ (in particolare $H_i(\Omega;G)=0$ per $0<i<p-1$).
\end{corollary}
\begin{proof}
Consideriamo la fibrazione $\fibration{\Omega}{E}{X}$ del \thref{loop-space-fibration}. Dalla \thref{fibration-homology-exact-sequence} (applicata con $q=1$) otteniamo le successioni esatte
\begin{diagram}
H_i(E;G)\rar&H_i(X;G)\rar{d_i}&H_{i-1}(\Omega;G)\rar&H_{i-1}(E;G)
\end{diagram}
per $2\le i\le p$. Poiché $E$ è contrattile, $H_i(E;G)=0$ per ogni $i>0$, da cui la tesi.
\end{proof}

Naturalmente vale il risultato duale in coomologia.
\begin{corollary}\thlabel{loop-space-cohomology-isomorphism}
Sia $G$ un PID. Supponiamo che $H^i(X;G)=0$ per $0<i<p$. Allora $\map{d_i}{H^{i-1}(\Omega;G)}{H^i(X;G)}$ è un isomorfismo per $2\le i\le p$ (in particolare $H^i(\Omega;G)=0$ per $0<i<p-1$).
\end{corollary}