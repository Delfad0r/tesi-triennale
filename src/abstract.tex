\chapter*{Introduzione}
\markboth{Introduzione}{Introduzione}
\addcontentsline{toc}{chapter}{Introduzione}
Lo scopo di questa trattazione è lo studio dei gruppi di omotopia superiori delle sfere di dimensione dispari. Più precisamente, ci proponiamo di dimostrare che per ogni intero \(n\ge 3\) dispari e per ogni \(i>n\) l'\(i\)-esimo gruppo di omotopia della \(n\)-sfera è finito. Seguiremo un approccio \enquote{storico} nell'affrontare questo problema, mantenendoci perlopiù fedeli alla tesi di dottorato di Jean-Pierre Serre, \emph{Homologie singulière des espaces fibrés}, risalente al 1951. Ci discosteremo da essa solo per alcuni aspetti marginali (utilizzando ad esempio una teoria dei rivestimenti più moderna), lasciando però inalterato l'impianto generale.

Per arrivare a dimostrare il risultato sopra citato, la cui formulazione è praticamente elementare, sarà necessario sviluppare strumenti concettualmente più avanzati, alcuni dei quali nascevano proprio negli anni in cui Serre stava lavorando alla sua tesi, o sono addirittura dovuti allo stesso Serre.

Inizieremo la trattazione presentando in modo puramente algebrico, e quindi \enquote{astratto}, il concetto di successione spettrale (omologica e coomologica), cui è dedicato il capitolo \ref{ch:spectral-sequences}. Mostreremo alcune proprietà di base delle successioni spettrali, in particolare come costruirne a partire da filtrazioni di complessi. Questo strumento, pur avendo una natura prettamente algebrica, sarà di fondamentale importanza nelle circostanze in cui emergerà da contesti topologici. Ciò accade, ad esempio, con la successione spettrale associata al rivestimento universale, di cui ci occuperemo brevemente alla fine di questo capitolo.

Ci soffermeremo più approfonditamente, invece, sulla successione spettrale associata a uno spazio fibrato, che sarà l'oggetto dei tre capitoli successivi. Nel capitolo \ref{ch:homology-and-cohomology-of-fibrations} introdurremo gli strumenti topologici fondamentali: esporremo innanzitutto una teoria omologica (l'omologia singolare cubica) che sostituirà l'omologia singolare standard nella nostra trattazione, per poi passare alla definizione di spazi fibrati e alle loro proprietà. È opportuno osservare che la classe degli spazi fibrati secondo la definizione di Serre è molto più ampia di quella dei fibrati in senso classico: questa maggiore generalità ci consentirà di applicare i risultati sulle fibrazioni a spazi topologici di grande interesse, ossia gli spazi di cammini. Il capitolo \ref{ch:homological-spectral-sequence-of-fibration} sarà dedicato all'applicazione degli strumenti sviluppati nel capitolo \ref{ch:spectral-sequences} al complesso singolare (opportunamente filtrato) di uno spazio fibrato, in modo da costruire una successione spettrale che metta in relazione i gruppi di omologia della base, della fibra e dello spazio totale. La versione \enquote{duale}, ossia coomologica, di questa costruzione verrà studiata nel capitolo \ref{ch:cohomological-spectral-sequence-of-fibration}, prestando particolare attenzione all'aspetto che maggiormente la distingue dalla sua controparte omologica, ossia la struttura moltiplicativa.

Avvicinandoci alla conclusione, nel capitolo \ref{ch:path-spaces} presenteremo alcune proprietà degli spazi di cammini. Sarà di particolare interesse osservare che ogni spazio topologico semplicemente connesso \(X\) è base di una fibrazione avente per fibra il suo spazio di cammini chiusi \(\Omega\) e per spazio totale uno spazio contrattile: grazie ai risultati presentati nei capitoli precedenti, per mezzo di una successione spettrale potremo mettere in relazione i gruppi di omologia e coomologia di \(X\) con quelli di \(\Omega\). Il capitolo \ref{ch:homotopy-groups-of-spheres}, infine, avrà il compito di raccogliere gli ultimi risultati che ci porteranno a dimostrare quanto ci eravamo prefissati sui gruppi di omotopia delle sfere di dimensione dispari: dopo aver precisato una condizione tecnica di tipo topologico che giustifichi le costruzioni che ci serviranno, una serie di passaggi sostanzialmente algebrici ci condurrà alla conclusione, sfruttando tutti gli strumenti sviluppati nei precedenti capitoli.

Come dimostra Serre nella sua tesi, è possibile spingersi oltre, impiegando gli stessi metodi per studiare anche i gruppi di omotopia delle sfere di dimensione pari. Più precisamente, Serre mostra che per \(n\) pari e \(i>n\) l'\(i\)-esimo gruppo di omotopia della \(n\)-sfera è finito, eccezion fatta per \(i=2n-1\); in questo caso l'\(i\)-esimo gruppo di omotopia è somma diretta di \(\ZZ\) e di un gruppo finito. Nonostante l'indiscutibile importanza del risultato, non ce ne occuperemo in questa trattazione: abbiamo preferito concentrarci sui concetti innovativi introdotti da Serre (in particolare sulle successioni spettrali associate agli spazi fibrati) piuttosto che sui numerosi tecnicismi algebrici che sarebbero stati necessari per lo studio delle sfere di dimensione pari.
