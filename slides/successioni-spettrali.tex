\section{Successioni spettrali}
\newcommand*{\sseqOne}{\(E^{p,q}_r=0\) per \(p<0\) o \(q<0\)}
\newcommand*{\sseqTwo}{\(d^{p+r,q-r+1}_r\circ d^{p,q}_r=0\)}
\newcommand*{\sseqThree}{\(E^{p,q}_{r+1}=\ker d^{p,q}_r/\im d^{p+r,q-r+1}_r\)}
\begin{frame}
\frametitle{\secname}
\begin{block}{Definizione}
Si dice \emph{successione spettrale} una famiglia di gruppi abeliani \(E^{p,q}_r\) indicizzata da \(p,q\in\ZZ\), \(r\in\NN\) dotata di mappe di bordo \(\map{d^{p,q}_r}{E^{p,q}_r}{E^{p-r,q+r-1}_r}\) che soddisfa le seguenti proprietà:
\begin{enumerate}
\item \sseqOne;
\item \(d^{p+r,q-r+1}_r\circ d^{p,q}_r=0\);
\item \(E^{p,q}_{r+1}=\ker d^{p,q}_r/\im d^{p+r,q-r+1}_r\).
\end{enumerate}
\end{block}
\end{frame}
\begin{frame}
\frametitle{\secname}
\begin{overprint}
\onslide<1>
\begin{enumerate}
\item \sseqOne\vphantom{\sseqTwo}
\end{enumerate}
\onslide<2>
\begin{enumerate}
\setcounter{enumi}{1}
\item \sseqTwo
\end{enumerate}
\onslide<3>
\begin{enumerate}
\setcounter{enumi}{2}
\item \sseqThree
\end{enumerate}
\onslide<4>Fissati \(p\) e \(q\), per \(r\) sufficientemente grande vale
\vspace{-10pt}
\[
E^{p,q}_r=E^{p,q}_{r+1}=\ldots=E^{p,q}_\infty.
\]
\end{overprint}
\vfill
\begin{overprint}
\onslide<1>\centering\drawsseq{homological}{0}
\onslide<2>\centering\drawsseq{homological}{1}
\onslide<3>\centering\drawsseq{homological}{2}
\onslide<4>\centering\drawsseq{homological}{3}
\end{overprint}
\end{frame}
\begin{frame}
\frametitle{\secname}
I gruppi $E^{p,q}_\infty$ sono detti \emph{gruppi terminali}.
\vspace{0.5cm}
\visible<2->{
\begin{block}{Definizione}
Si dice che \(E^{p,q}_r\converges H_{p+q}\) se esistono filtrazioni
\[
0=H^{-1}_n\subs H^0_n\subs H^1_n\subs\ldots\subs H^{n-1}_n\subs H^n_n=H_n
\]
tali che \(E^{p,q}_\infty=H^p_{p+q}/H^{p-1}_{p+q}\).
\end{block}
}
\end{frame}
\begin{frame}
\frametitle{\secname}
\begin{block}{Teorema}
Siano \(\map{p}{T}{X}\) un rivestimento universale, \(\pi=\pi_1(X)\). Allora esiste una successione spettrale \(E^{p,q}_r\converges H_{p+q}(X)\) con
\[
E^{p,q}_2=H_p(\pi;H_q(T)).
\]
\end{block}
\pause
\begin{block}{Corollario}
Se \(\pi\) è abeliano e finitamente generato e i gruppi \(H_i(X)\) sono finitamente generati, allora anche i gruppi \(H_i(T)\) lo sono.
\end{block}
\end{frame}