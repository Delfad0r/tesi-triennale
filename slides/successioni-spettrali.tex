\section{Successioni spettrali}
\begin{frame}
\begin{block}{Definizione}
Si dice \emph{successione spettrale} una famiglia di gruppi abeliani \(E^{p,q}_r\) indicizzata da \(p,q\in\ZZ\), \(r\in\NN\) dotata di mappe di bordo \(\map{d^{p,q}_r}{E^{p,q}_r}{E^{p-r,q+r-1}}\) che soddisfa le seguenti proprietà:
\begin{enumerate}
\item \(E^{p,q}_r=0\) per \(p<0\) o \(q<0\);
\item \(d^{p+r,q-r+1}_r\circ d^{p,q}_r=0\);
\item \(E^{p,q}_{r+1}=\ker d^{p,q}_r/\im d^{p+r,q-r+1}_r\).
\end{enumerate}
\end{block}
\end{frame}
\begin{frame}
\begin{center}
\only<1>{\drawsseq{homological}{0}}
\only<2>{\drawsseq{homological}{1}}
\only<3>{\drawsseq{homological}{2}}
\only<4>{\drawsseq{homological}{3}}
\end{center}
\end{frame}