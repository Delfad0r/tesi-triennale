\section{Spazi fibrati}

\begin{frame}
\frametitle{.}
\begin{block}{Definizione}
Si dice \emph{fibrazione} un'applicazione suriettiva \(\map{p}{E}{B}\) che soddisfa la proprietà di sollevamento dell'omotopia per poliedri finiti:
\visible<2->{
\begin{diagram}[ampersand replacement=\&]
P\times\{0\}\rar[visible on=<4->]{\tilde{f}_0}\dar[hook]\&E\dar{p}\\
P\times[0,1]\rar[visible on=<3->]{f}\ar[visible on=<5->,ur,"\tilde{f}",dashed]\&B
\end{diagram}
}
\end{block}
\visible<6->{
\(E\) è detto \emph{spazio totale}, \(B\) \emph{spazio base}, \(F=p^{-1}(b)\) \emph{fibra}.
}
\end{frame}
\begin{frame}
\frametitle{.}
\begin{block}{Teorema}
Esiste una successione esatta lunga di gruppi
\begin{diagram}[ampersand replacement=\&]
\pi_n(F)\rar\&\pi_n(E)\rar\&\pi_n(B)\rar\&\pi_{n-1}(F)\rar\&\ldots
\end{diagram}
\end{block}
\pause
\begin{block}{Teorema}
Esiste una successione spettrale \(E^{p,q}_r\converges H_{p+q}(E)\) con
\[
E^{p,q}_2=H_p(B;H_q(F)).
\]
\end{block}
\end{frame}