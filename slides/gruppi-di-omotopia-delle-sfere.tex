\section{Gruppi di omotopia delle sfere}
\newcommand*{\sphereheader}[2]{%
\begin{columns}
\checklistsize
\begin{column}{0.5\textwidth}
\begin{itemize}
\item[\ifnum #1=1 \done\else\notdone\fi] \(\pi_i(S^n)\) f.g.
\end{itemize}
\end{column}
\begin{column}{0.5\textwidth}
\begin{itemize}
\item[\ifnum #2=1 \done\else\notdone\fi] \(\pi_i(S^n)\tensor K=0\)
\end{itemize}
\end{column}
\end{columns}
\rule{\textwidth}{0.4pt}
}
\newenvironment*{sphereframe}[3][]{%
\begin{frame}[t,#1]
\frametitle{.}
\sphereheader{#2}{#3}
\vfill
}{%
\vfill
\end{frame}
}
\newcommand<>{\highlighton}[2]{\alt<#1>{\invboxx{#2}}{\alt#3{\boxx{#2}}{{\color{white}\fbox{\color{gray!50}#2}}}}}
\begin{frame}
\frametitle{.}
Vogliamo mostrare che per \(n\ge 3\) dispari e \(i>n\) il gruppo di omotopia \(\pi_i(S^n)\) è finito.
\pause
\begin{itemize}[<+->]
\item[\notdone] \(\pi_i(S^n)\) è finitamente generato.
\item[\notdone] \(\pi_i(S^n)\tensor K=0\) (dove \(K\) denota un campo di caratteristica \(0\)).
\end{itemize}
\end{frame}
\begin{sphereframe}{0}{0}
Sia \(X\) uno spazio topologico semplicemente connesso con gruppi di omologia finitamente generati.
\pause
\begin{block}{Costruzione}
\[
\begin{cases}
X_0=X\\
\visible<+(1)->{T_n=\text{rivestimento universale di \(X_n\)}}\\
\visible<+(1)->{X_{n+1}=\Omega T_n}
\end{cases}
\]
\end{block}
\end{sphereframe}
\begin{sphereframe}{0}{0}
\begin{itemize}
\item \(T_n\to X_n\quad\rightsquigarrow\quad\begin{cases}\pause\pi_i(T_n)=\pi_i(X_n)&i\ge 2\vspace{0.2cm}\pause\\\text{\(E^{p,q}_r\converges H_{p+q}(X_n)\) con}\\ \qquad E^{p,q}_2=H_p(\pi_1(X_n);H_q(T_n))\end{cases}\)
\vspace{0.2cm}
\pause
\item \(\fibration{X_{n+1}}{E}{T_n}\quad\rightsquigarrow\quad\begin{cases}\pause\pi_i(X_{n+1})=\pi_{i+1}(T_n)\vspace{0.2cm}\pause\\\text{\(E^{p,q}_r\converges H_{p+q}(E)\) con}\\ \qquad E^{p,q}_2=H_p(T_n;H_q(X_{n+1}))\end{cases}\)
\end{itemize}
\end{sphereframe}
\begin{frame}[t]
\frametitle{.}
\only<-8>{\sphereheader{0}{0}}
\only<9-31>{\sphereheader{1}{0}}
\only<32->{\sphereheader{1}{1}}
\vfill
\begin{columns}
\begin{column}{0.3\textwidth}
\begin{minipage}[t][.6\textheight]{2\textwidth}
\checklistsize
\begin{itemize}
\item<2->[\(\blacktriangleright\)]
\highlighton<16,30>{2}{\(\pi_{n+i}(X)=\pi_i(X_n)\)}
\item<11->[\(\blacktriangleright\)] \highlighton<28>{11}{\(\pi_i(X)\tensor K=0\)}
\item<12->[\(\blacktriangleright\)] \highlighton<20,22>{12}{\(\Exterior\langle u_n\rangle\Rightarrow K[u_{n-1}]\)}
\item<13->[\(\blacktriangleright\)] \highlighton<21-23>{13}{\(K[u_n]\Rightarrow\Exterior\langle u_{n-1}\rangle\)}
\item<17->[\(\blacktriangleright\)] \highlighton<20-23>{17}{\(\pi_1(X_i)=0\)}
\item<17->[\(\blacktriangleright\)] \highlighton<27>{17}{\(\pi_1(X_{n-1})=\ZZ\)}
\end{itemize}
\end{minipage}
\end{column}
{\vrule{}}
\begin{column}{0.7\textwidth}
%\begin{minipage}{\textwidth}
\only<-9>{
\begin{itemize}
\addtolength{\itemsep}{8pt}
\item<+-> \(\pi_i(X_{n+1})=\pi_{i+1}(T_n)=\pi_{i+1}(X_n)\)\\\visible<+->{
\(\qquad\implies\)\boxxon<.>{\(\pi_{n+i}(X)=\pi_i(X_n)\)}}
\item<+-> \(\pi_{n+1}(X)=\pi_1(X_n)\visible<+->{\stackrel{\textcolor{blue}{\text{Hurewicz}}}{=}H_1(X_n)}\)
\item<+-> \(H_i(X)\) f.g. \visible<+->{\(\implies H_i(T_n)\), \(H_i(X_n)\) f.g.}\\ \visible<+->{\(\qquad\implies H_1(X_{i-1})\visible<+->{=\pi_i(X)}\) f.g.}
\end{itemize}
}
\only<10-13>{
\addtocounter{beamerpauses}{10}
\begin{itemize}
\item<+-> Se \(H_i(X;K)=0\) per \(i>0\), allora \boxxon<.>{\(\pi_i(X)\tensor K=0\)} per \(i\ge 2\).
\item<+-> Se \(H^*(X;K)=\Exterior\langle u_n\rangle\) con \(n\ge3\) dispari, allora \boxxon<.>{\(H^*(\Omega X;K)=K[u_{n-1}]\)}.
\item<+-> Se \(H^*(X;K)=K[u_n]\) con \(n\ge 2\) pari, allora \boxxon<.>{\(H^*(\Omega X;K)=\Exterior\langle u_{n-1}\rangle\)}.
\end{itemize}
}
\only<14-17>{
\addtocounter{beamerpauses}{14}
\begin{itemize}
\item<+-> Poniamo \(X=S^n\).
\item<+-> \(\pi_1(X_i)=\boxxon<.(1)>{\(\pi_{i+1}(S^n)\visible<+->{=\begin{cases}
0&i\le n-2\\
\ZZ&i=n-1
\end{cases}}\)}\)
\end{itemize}
}
\only<18-25>{
\addtocounter{beamerpauses}{18}
\begin{itemize}
\eqitem<+-24>\begin{align*}
&H^*(X_0;K)=H^*(S^n)=\Exterior\langle u_n\rangle\\
\visible<+->{&\qquad\implies H^*(X_1;K)=K[u_{n-1}]}\\
\visible<+->{&\qquad\implies H^*(X_2;K)=\Exterior\langle u_{n-2}\rangle}\\
\visible<+->{&\qquad\implies\ldots}\\
\visible<+->{&\qquad\implies H^*(X_{n-1};K)=\Exterior\langle u_1\rangle}
\end{align*}
\item<+-> In particolare, \(H^i(X_{n-1};K)=0\) per \(i>1\).
\end{itemize}
}
\only<26->{
\addtocounter{beamerpauses}{26}
\begin{itemize}
\item<.-> In particolare, \(H^i(X_{n-1};K)=0\) per \(i>1\).
\item<+-> Si può dimostrare che allora \(H_i(T_{n-1};K)=0\) per \(i>0\).
\item<+-> \hspace{-3pt}\mbox{\begin{overprint}[50pt]\onslide<.>\(\pi_i(T_{n-1})\)
\onslide<+>\color{blue}\(\pi_i(X_{n-1})\)\onslide<+>\color{blue}\(\pi_{i+n-1}(X)\)\onslide<+->\(\pi_{i+n-1}(X)\)\end{overprint}}\(\tensor K=0\) per \(i\ge 2\)\\
\visible<.->{\(\qquad\implies\pi_i(S^n)\tensor K=0\) per \(i\ge n-1\)}
\end{itemize}
}
%\end{minipage}
\end{column}
\end{columns}
\vfill
\end{frame}