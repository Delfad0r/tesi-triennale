\section{Gruppi di omotopia delle sfere}
\newcommand{\sphereheader}[2]{%
\begin{columns}
\begin{column}{0.5\textwidth}
\begin{itemize}
\item[\ifnum #1=1 \done\else \(\square\)\fi] \(\pi_i(S^n)\) f.g.
\end{itemize}
\end{column}
\begin{column}{0.5\textwidth}
\begin{itemize}
\item[\ifnum #2=1 \done\else \(\square\)\fi] \(\pi_i(S^n)\tensor K=0\)
\end{itemize}
\end{column}
\end{columns}
\rule{\textwidth}{0.4pt}
}
\begin{frame}
\frametitle{.}
Vogliamo mostrare che per \(n\ge 3\) dispari e \(i>n\) il gruppo di omotopia \(\pi_i(S^n)\) è finito.
\pause
\begin{itemize}[<+->]
\item[\(\square\)] \(\pi_i(S^n)\) è finitamente generato.
\item[\(\square\)] \(\pi_i(S^n)\tensor K=0\) (dove \(K\) denota un campo di caratteristica \(0\)).
\end{itemize}
\end{frame}
\begin{frame}[t]
\frametitle{.}
\sphereheader{0}{0}
\end{frame}