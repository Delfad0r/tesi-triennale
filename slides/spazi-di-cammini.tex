\section{Spazi di cammini}
\begin{frame}
\frametitle{\secname}
\begin{block}{Definizione}
Siano \(X\) uno spazio topologico connesso per archi, \(A,B\subs X\) sottospazi. Si definisce
\[
E_{A,B}=\{f\in C([0,1],X):f(0)\in A, f(1)\in B\}
\]
con la topologia compatta-aperta.
\end{block}
\end{frame}
\begin{frame}
\frametitle{\secname}
Supponiamo che \(X\) sia semplicemente connesso.
\pause
\begin{block}{Teorema}
Sia \(y\in X\). Allora l'applicazione
\Map{p}{E_{y,X}}{X}{f}{f(1)}
è una fibrazione.
\end{block}
\pause
\begin{itemize}[<+->]
\item Lo spazio totale \(E_{y,X}\) è contrattile.
\item Il tipo di omotopia delle fibre \(E_{y,z}\) non dipende da \(y\) e \(z\).
\item Le fibre sono omotopicamente equivalenti a \(\Omega X=E_{y,y}\).
\end{itemize}
\end{frame}
\begin{frame}
\frametitle{\secname}
Abbiamo dunque una fibrazione \(\fibration{\Omega X}{E}{X}\) con \(E\) contrattile.
\begin{itemize}
\addtolength{\itemsep}{10pt}
\item<2->\(
\begin{tikzcd}[ampersand replacement=\&]
\cancelto<3->{0}{\pi_{n+1}(E)}\rar\&\pi_{n+1}(X)\rar[visible on=<2-3>]\rar[visible on=<4->,text=blue]{\iso}\&\pi_n(\Omega X)\rar\&\cancelto<3->{0}{\pi_n(E)}
\end{tikzcd}
\)
\item<5-> \(E^{p,q}_r\converges H_{p+q}(E)\) con \(E^{p,q}_2=H_p(X;H_q(\Omega X))\).
\item<6-> Se i gruppi \(H_i(X)\) sono finitamente generati, allora anche i gruppi \(H_i(\Omega X)\) lo sono.
\end{itemize}
\end{frame}